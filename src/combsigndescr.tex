\section{The Combined Signature primitive}
\label{sec:ov-combined-ds}
  As previously mentioned, we define a new primitive for combining keys and
  generating signatures, which is leveraged in the revocation and punishment
  mechanism of channel updates. Furthermore, we prove that the construction
  designed by the creators of LN realizes this primitive. We provide here the
  concrete syntax and correctness definitions, along with the intuition behind
  it, the exact security definitions, a concrete construction and proof of
  \redden{its} security.

  Previous work on the subject of multi-party
signatures~\cite{DBLP:conf/ndss/NicolosiKDM03,DBLP:journals/iacr/BellareS01,boyd1986digital,DBLP:conf/ndss/Ganesan95a,DBLP:conf/crypto/MacKenzieR01,ganesan1994secure}
  focuses on use-cases where some parties desire to generate a signature without
  revealing their private information; the latter is created using an
  interactive protocol. The resulting signatures can be verified by a single
  verification key, which is also included in the output of the key generation
  protocol. As we will see however, the primitive defined here has different
  aims and limitations and, to our knowledge, has not been formalized yet.

  A combined signature is a two-party primitive, say between \alice{}
  and \bob, with \bob{} being the signer and \alice{} the holder of a
  share of the secret key. This share is essential for issuing
  signatures, which in turn are verifiable with the ``combined''
  verification key. The verification key is generated using public
  information drawn from \alice{} and \bob{} and is feasible without any party
  knowing the corresponding signing key. \bob{} will be able to
  construct the signing key only if \alice{} shares her secret information with
  him.

  More specifically, the seven algorithms used by a Combined Signatures
  scheme are:
  \begin{itemize}
    \item $\left(mpk, msk\right) \gets \textsc{MasterKeyGen}\left(1^k\right)$
    \item $\left(pk, sk\right) \gets \textsc{KeyShareGen}\left(1^k\right)$
    \item $cpk_l \gets \textsc{CombinePubKey}\left(mpk, pk\right)$
    \item $\left(cpk_l, csk_l\right) \gets \textsc{CombineKey}\left(mpk,
    msk, pk, sk\right)$
    \item $\{0, 1\} \gets \textsc{TestKey}\left(pk, sk\right)$
    \item $\sigma \gets \textsc{SignCS}\left(m, csk\right)$
    \item $\left\{0, 1\right\} \gets \textsc{VerifyCS}\left(\sigma, m,
    cpk\right)$
  \end{itemize}
  We demand that these three properties hold for a scheme to have
  correctness:
  \begin{itemize}
    \item $\forall k \in \mathbb{N},$ \\
    $\Pr[\left(pk, sk\right) \gets \textsc{KeyShareGen}\left(1^k\right),$ \\
    $\textsc{TestKey}(pk, sk) = 1] = 1$

    I.e. \textsc{KeyShareGen}() must always generate a valid keypair.

    \item $\forall k \in \mathbb{N},$ \\
    $\Pr[(mpk, msk) \gets \textsc{MasterKeyGen}\left(1^k\right),$ \\
    $\left(pk, sk\right) \gets \textsc{KeyShareGen}\left(1^k\right),$ \\
    $\left(cpk_1, csk_1\right) \gets \textsc{CombineKey}\left(mpk, msk, pk,
    sk\right),$ \\
    $cpk_2 \gets \textsc{CombinePubKey}\left(mpk, pk\right),$ \\
    $cpk_1 = cpk_2] = 1$

    I.e. for suitable input, \textsc{CombinePubKey}() and
    \textsc{CombineKey}() produce the same public key.

    \item $\forall k \in \mathbb{N}, m \in \mathcal{M},$ \\
    $\Pr[(mpk, msk) \gets \textsc{MasterKeyGen}\left(1^k\right),$ \\
    $\left(pk, sk\right) \gets \textsc{KeyShareGen}\left(1^k\right),$ \\
    $\left(cpk, csk\right) \gets \textsc{CombineKey}\left(mpk, msk, pk,
    sk\right),$ \\
    $\textsc{VerifyCS}(\textsc{SignCS}(m, csk), m, cpk) = 1] = 1$

    I.e. for suitable input, honestly generated signatures always verify
    correctly.
  \end{itemize}

  Beyond correctness, combined signatures have two security properties expressed
  as follows. \textsf{Share-EUF} security expresses security from the
  point of view of \alice{}, and establishes that \bob{} cannot issue a
  valid combined signature if he does not possess \alice's corresponding secret
  share. \redden{Formally:}

  \redden{
  \begin{gamebox}{$\mathsf{share \mhyphen
  EUF}^{\adversary}\left(1^k\right)$}
    \begin{algorithmic}[1]
      \State $\left(\mathtt{aux}, mpk, n\right) \gets
      \adversary\left(\textsc{init}\right)$
      \For{$i \gets 1$ to $n$}
        \State $\left(pk_i, sk_i\right) \gets
        \textsc{KeyShareGen}\left(1^k\right)$
      \EndFor
      \State $\left(cpk^*, pk^*, m^*, \sigma^*\right) \gets
      \adversary\left(\textsc{keys}, \mathtt{aux}, pk_1, \dots, pk_n\right)$
      \If{$pk^* \in \left\{pk_1, \dots, pk_n\right\} \wedge$
      $cpk^* = \textsc{CombinePubKey}\left(mpk, pk^*\right) \wedge$
      $\textsc{VerifyCS}\left(\sigma^*, m^*, cpk^*\right) = 1$}
        \State \Return 1
      \Else
        \State \Return 0
      \EndIf
    \end{algorithmic}
  \end{gamebox}
  \begin{definition}
    \label{def:share:secure}
    A Combined Signatures scheme is \emph{\textsf{share-EUF}-secure} if
    \begin{gather*}
      \forall \adversary \in \mathtt{PPT}, \Pr\left[\mathsf{share \mhyphen
      EUF}^{\adversary}\left(1^k\right) = 1\right] = \mathrm{negl}\left(k\right)
      \ifelseieee{\\}{}
      \text{or equivalently} \\
      \mathrm{E \mhyphen share}(k) = \mathrm{negl}\left(k\right) \enspace, \\
      \text{where } \mathrm{E \mhyphen share}(k) = \underset{\adversary \in
      \mathtt{PPT}}{\sup}\{\Pr[\mathsf{share \mhyphen
      EUF}^{\adversary}\left(1^k\right) = 1]\}
    \end{gather*}
  \end{definition}
  }

  On the other hand, \textsf{master-EUF-CMA} security is modeled very
  similarly to standard \textsf{EUF-CMA} security, with the difference
  that \bob{} (the signer) combines malicious shares into his public key and
  issues signatures with respect to such combined keys. The security
  property ensures that these signatures provide no advantage to the adversary
  in terms of producing a forged message for a combined key of its choice.
  \redden{Formally:}

  \begin{gamebox}{$\mathsf{master \mhyphen EUF \mhyphen
  CMA}^{\adversary}\left(1^k\right)$}
    \begin{algorithmic}[1]
      \State $\left(mpk, msk\right) \gets
      \textsc{MasterKeyGen}\left(1^k\right)$
      \State $i \gets 0$
      \State $\left(\mathtt{aux}_i, \mathrm{response}\right) \gets
      \adversary\left(\textsc{init}, mpk\right)$
      \While{response can be parsed as $\left(pk, sk, m\right)$}
        \State $i \gets i + 1$
        \State store $pk, sk, m$ as $pk_i, sk_i, m_i$
        \State $\left(cpk_i, csk_i\right) \gets
        \textsc{CombineKey}\left(mpk, msk, pk_i, sk_i\right)$
        \State $\sigma_i \gets \textsc{SignCS}\left(m_i, csk_i\right)$
        \State $\left(\mathtt{aux}_i, \mathrm{response}\right) \gets
        \adversary\left(\textsc{signature}, \mathtt{aux}_{i-1},
        \sigma_i\right)$
      \EndWhile
      \State parse response as $\left(cpk^*, pk^*, m^*, \sigma^*\right)$
      \If{$m^* \notin \left\{m_1, \dots, m_i\right\} \wedge$
      $cpk^* = \textsc{CombinePubKey}\left(mpk, pk^*\right) \wedge$
      $\textsc{VerifyCS}\left(\sigma^*, m^*, cpk^*\right) = 1$}
        \State \Return 1
      \Else
        \State \Return 0
      \EndIf
    \end{algorithmic}
  \end{gamebox}
  \begin{definition}
    \label{def:master:secure}
    A Combined Signatures scheme is \emph{\textsf{master-EUF-CMA}-secure} if
    \begin{gather*}
      \forall \adversary \in \mathtt{PPT}, \ifelseieee{\\}{}
      \Pr\left[\mathsf{master \mhyphen EUF \mhyphen
      CMA}^{\adversary}\left(1^k\right) = 1\right] =
      \mathrm{negl}\left(k\right)
      \ifelseieee{\\}{}
      \text{or equivalently} \\
      \mathrm{E \mhyphen master}(k) = \mathrm{negl}\left(k\right) \enspace, \\
      \text{where }
      \ifelseieee{\\}{}
      \mathrm{E \mhyphen master}(k) = \underset{\adversary \in
      \mathtt{PPT}}{\sup}\{\Pr[\mathsf{master \mhyphen
      EUF-CMA}^{\adversary}\left(1^k\right) = 1]\}
    \end{gather*}
  \end{definition}

  \begin{definition}
    \redden{A Combined Signatures scheme is \emph{\textsf{combine-EUF}-secure}
    if it is both \textsf{share-EUF}-secure and \textsf{master-EUF-CMA}-secure.}
  \end{definition}

  \redden{In conclusion, a collection of algoritms is said to be a secure
  Combined Signatures scheme if it conforms to the syntax of the seven
  aforementioned algorithms, it satisfies the three correctness properties and
  provides existential unforgeability against key-only attacks with respect to
  key shares and existential unforgeability against chosen message attacks with
  respect to master keys.}

  \redden{We here define the particular construction for Combined Signatures
  used in LN and prove its security.}

  \redden{Parameters: hash function $\mathcal{H}$, group generator $G$
  \begin{algorithmic}[0]
    \State \textsc{MasterKeyGen}($1^k$, rand):
    \Indent
      \State \Return ($G \cdot \mathrm{rand}$, rand)
    \EndIndent
  \end{algorithmic}}

  \redden{\begin{algorithmic}[0]
    \State \textsc{KeyShareGen}($1^k$, rand):
    \Indent
      \State \Return ($G \cdot \mathrm{rand}$, rand)
    \EndIndent
  \end{algorithmic}}

  \redden{\begin{algorithmic}[0]
    \State \textsc{CombinePubKey}($mpk, pk$):
    \Indent
      \State \Return $mpk \cdot \mathcal{H}\left(mpk \concat pk\right) + pk
      \cdot \mathcal{H}\left(pk \concat mpk\right)$
    \EndIndent
  \end{algorithmic}}

  \redden{\begin{algorithmic}[0]
    \State \textsc{CombineKey}($mpk, msk, pk, sk$):
    \Indent
      \State \Return $(\textsc{CombinePubKey}(mpk, pk), msk \cdot
      \mathcal{H}\left(mpk \concat pk\right) + sk \cdot \mathcal{H}\left(pk
      \concat mpk\right))$
    \EndIndent
  \end{algorithmic}}

  \redden{\begin{algorithmic}[0]
    \State \textsc{TestKey}($pk, sk$):
    \Indent
      \If{$pk = G \cdot sk$}
        \State \Return 1
      \Else
        \State \Return 0
      \EndIf
    \EndIndent
  \end{algorithmic}}

  \redden{\begin{algorithmic}[0]
    \State \textsc{SignCS}($m, csk$):
    \Indent
      \State \Return \textsc{SignDS}($m, csk$)
    \EndIndent
  \end{algorithmic}}

  \redden{\begin{algorithmic}[0]
    \State \textsc{VerifyCS}($\sigma, m, cpk$):
    \Indent
      \State \Return \textsc{VerifyDS}($\sigma, m, cpk$)
    \EndIndent
  \end{algorithmic}}

  \redden{One can check by inspection that the syntax above matches the one
  required by the Combined Signatures scheme definition. Furthermore, assuming
  that \textsc{SignDS}() and \textsc{VerifyDS}() are provided by a correct
  Digital Signature construction, it is straightforward to confirm that the
  construction here satisfies the Combined Signatures correctness properties.}

  \redden{We now move on to proving that the construction is also secure.}

  \begin{lemma}
    \label{lemma:comb:share}
    The construction defined \redden{above} is \textsf{share-EUF}-secure in the
    Random Oracle model under the assumption that the underlying signature
    scheme is strongly \textsf{EUF-CMA}-secure and the range of the Random
    Oracle coincides with that of the underlying signature scheme signing keys.
  \end{lemma}

  \begin{proof}
    \redden{Let $k \in \mathbb{N}, \mathcal{B}$ PPT algorithm such that
    \begin{equation*}
      \Pr\left[\mathsf{share \mhyphen EUF}^{\mathcal{B}}\left(1^k\right) =
      1\right] = a = \mathrm{non-negl}\left(k\right) \enspace.
    \end{equation*}
    We construct a PPT distinguisher \adversary{}
    (Fig.~\ref{proof:comb:share:distinguisher}) such that
    \begin{equation*}
      \Pr\left[\mathsf{EUF \mhyphen CMA}^{\adversary}\left(1^k\right) =
      1\right] = \mathrm{non-negl}\left(k\right)
    \end{equation*}
    that breaks the assumption, thus proving Lemma~\ref{lemma:comb:share}.}

    \begin{figure}[!htbp]
      \begin{algobox}{$\adversary\left(vk\right)$}
        \begin{algorithmic}[1]
          \State $j \overset{\$}{\gets} U\left[1,
          T\left(\mathcal{B}\right)\right]$
          \Comment{$T\left(M\right)$ is the maximum running time of $M$}
          \Indent
            \State Random Oracle: for every first-seen query $q$ from
            $\mathcal{B}$ set $\mathcal{H}\left(q\right)$ to a random value
            \State \Return $\mathcal{H}\left(q\right)$ to $\mathcal{B}$
          \EndIndent
          \State $\left(\mathtt{aux}, mpk, n\right) \gets
          \adversary\left(\textsc{init}\right)$
          \For{$i \gets 1$ to $n$}
            \State $\left(pk_i, sk_i\right) \gets
            \textsc{KeyShareGen}\left(1^k\right)$
          \EndFor
          \Indent
            \State Random Oracle: Let $q$ be the $j$th first-seen query from
            $\mathcal{B}$:
            \If{$q = \left(mpk \concat x\right)$}
            \label{proof:comb:share:ro:start}
              \If{$\mathcal{H}\left(x \concat mpk\right)$ unset}
                \State set $\mathcal{H}\left(x \concat mpk\right)$ to a random
                value
              \EndIf
              \State set $\mathcal{H}\left(mpk \concat x\right)$ to $\left(vk
              - x \cdot \mathcal{H}\left(x \concat mpk\right)\right) \cdot
              mpk^{-1}$
            \ElsIf{$q = \left(x \concat mpk\right)$}
              \If{$\mathcal{H}\left(mpk \concat x\right)$ unset}
                \State set $\mathcal{H}\left(mpk \concat x\right)$ to a random
                value
              \EndIf
              \State set $\mathcal{H}\left(x \concat mpk\right)$ to $\left(vk
              - mpk \cdot \mathcal{H}\left(mpk \concat x\right)\right) \cdot
              x^{-1}$
              \label{proof:comb:share:ro:end}
            \Else
              \State set $\mathcal{H}\left(q\right)$ to a random value
            \EndIf
            \State \Return $\mathcal{H}\left(q\right)$ to $\mathcal{B}$
          \EndIndent
          \State $\left(cpk^*, pk^*, m^*, \sigma^*\right) \gets
          \mathcal{B}\left(\textsc{keys}, \mathtt{aux}, pk_1, \dots,
          pk_n\right)$
          \If{$vk = cpk^* \wedge \mathcal{B}$ wins the \textsf{share-EUF}
          game} \Comment{\adversary{} won the \textsf{EUF-CMA} game}
          \label{proof:comb:share:distinguisher:won}
            \State \Return $\left(m^*, \sigma^*\right)$
          \Else
            \State \Return \textsc{fail}
          \EndIf
        \end{algorithmic}
      \end{algobox}
      \caption{}
      \label{proof:comb:share:distinguisher}
    \end{figure}

    \redden{Let $Y$ be the range of the random oracle. The modified random
    oracle used in Fig.~\ref{proof:comb:share:distinguisher} is
    indistinguishable from the standard random oracle by PPT algorithms since
    the statistical distance of the standard random oracle from the modified one
    is at most $\frac{1}{2|Y|} = \mathrm{negl}\left(k\right)$ as they differ in
    at most one element.}

    \redden{Let $E$ denote the event in which $\mathcal{B}$ does not invoke
    \textsc{CombinePubKey} to produce $cpk^*$. In that case the values
    $\mathcal{H}\left(pk^* \concat mpk\right)$ and $\mathcal{H}\left(mpk
    \concat pk^*\right)$ are decided after $\mathcal{B}$ terminates
    (Fig.~\ref{proof:comb:share:distinguisher},
    line~\ref{proof:comb:share:distinguisher:won}) and thus
    \begin{equation}
      \begin{gathered}
        \Pr\left[cpk^* = \textsc{CombinePubKey}\left(mpk, pk^*\right) |
        E\right] =
        \ifelseieee{\\ =}{}
        \frac{1}{|Y|} = \mathrm{negl}\left(k\right) \Rightarrow \\
        \Pr\left[cpk^* = \textsc{CombinePubKey}\left(mpk, pk^*\right)
        \wedge E\right] =
        \ifelseieee{\\ =}{}
        \mathrm{negl}\left(k\right) \enspace.
      \end{gathered}
      \label{proof:comb:share:nocomb}
    \end{equation}
    It is
    \begin{gather*}
      \left(\mathcal{B} \text{ wins}\right) \rightarrow \left(cpk^* =
      \textsc{CombinePubKey}\left(mpk, pk^*\right)\right) \Rightarrow \\
      \Pr\left[\mathcal{B} \text{ wins}\right] \leq
      \ifelseieee{\\ \leq}{}
      \Pr\left[cpk^* = \textsc{CombinePubKey}\left(mpk, pk^*\right)\right]
      \Rightarrow \\
      \Pr\left[\mathcal{B} \text{ wins} \wedge E\right] \leq
      \ifelseieee{\\ \leq}{}
      \Pr\left[cpk^* = \textsc{CombinePubKey}\left(mpk, pk^*\right) \wedge
      E\right]
      \overset{\left(\ref{proof:comb:share:nocomb}\right)}{\Rightarrow} \\
      \Pr\left[\mathcal{B} \text{ wins} \wedge E\right] =
      \mathrm{negl}\left(k\right) \enspace.
    \end{gather*}
    }

    \redden{But we know that $\Pr\left[\mathcal{B} \text{ wins}\right] =
    \Pr\left[\mathcal{B} \text{ wins} \wedge E\right] + \Pr\left[\mathcal{B}
    \text{ wins} \wedge \neg E\right]$ and $\Pr\left[\mathcal{B} \text{
    wins}\right] = a$ by the assumption, thus
    \begin{equation}
      \label{proof:comb:share:nohash}
      \Pr\left[\mathcal{B} \text{ wins} \wedge \neg E\right] > a -
      \mathrm{negl}\left(k\right) \enspace.
    \end{equation}
    }

    \redden{We now focus at the event $\neg E$. Let $F$ the event in which the
    call of $\mathcal{B}$ to \textsc{CombinePubKey} to produce $cpk^*$ results
    in the $j$th invocation of the Random Oracle. Since $j$ is chosen uniformly
    at random and using Proposition~1 of~\cite{full-version}, $\Pr\left[F | \neg
    E\right] = \frac{1}{T\left(\mathcal{B}\right)}$. Observe that $\Pr\left[F |
    E\right] = 0 \Rightarrow \Pr\left[F\right] = \Pr\left[F | \neg E\right] =
    \frac{1}{T\left(\mathcal{B}\right)}$.}

    \redden{In the case where the event $\left(F \wedge \mathcal{B} \text{ wins}
    \wedge \neg E\right)$ holds, it is
    \begin{gather*}
      cpk^* = \textsc{CombinePubKey}\left(mpk, pk^*\right) = \\
      mpk \cdot \mathcal{H}\left(mpk \concat pk^*\right) + pk^* \cdot
      \mathcal{H}\left(pk^* \concat mpk\right)
    \end{gather*}
    Since $F$ holds, the $j$th invocation of the Random Oracle queried either
    the value $\mathcal{H}\left(mpk \concat pk^*\right)$ or
    $\mathcal{H}\left(pk^* \concat mpk\right)$. In either case
    (Fig.~\ref{proof:comb:share:distinguisher},
    lines~\ref{proof:comb:share:ro:start}-\ref{proof:comb:share:ro:end}), it
    is $cpk^* = vk$. This means that $\textsc{VerifyCS}\left(\sigma^*, m^*,
    vk\right) = 1$. We conclude that in the event $\left(F \wedge \mathcal{B}
    \text{ wins} \wedge \neg E\right)$, \adversary{} wins the \textsf{EUF-CMA}
    game. A final observation is that the probability that the events
    $\left(\mathcal{B} \text{ wins} \wedge \neg E\right)$ and $F$ are almost
    independent, thus
    \begin{gather*}
      \Pr\left[F \wedge \mathcal{B} \text{ wins} \wedge \neg E\right] =
      \ifelseieee{\\ =}{}
      \Pr\left[F\right] \Pr\left[\mathcal{B} \text{ wins} \wedge \neg E\right]
      \pm \mathrm{negl}\left(k\right)
      \overset{\left(\ref{proof:comb:share:nohash}\right)}{=} \\
      \frac{a - \mathrm{negl}\left(k\right)}{T\left(\mathcal{B}\right)} \pm
      \mathrm{negl}\left(k\right) = \mathrm{non-negl}\left(k\right)
    \end{gather*}
    }
  \end{proof}

  \begin{lemma}
    \label{lemma:comb:master}
    The construction \redden{above} is \textsf{master-EUF-CMA}-secure in the
    Random Oracle model under the assumption that the underlying signature
    scheme is strongly \textsf{EUF-CMA}-secure and the range of the Random
    Oracle coincides with that of the underlying signature scheme signing keys.
  \end{lemma}

  \begin{proof}
    \redden{
    Let $k \in \mathbb{N}, \mathcal{B}$ PPT algorithm such that
    \begin{equation*}
      \Pr\left[\mathsf{master \mhyphen EUF \mhyphen
      CMA}^{\mathcal{B}}\left(1^k\right) = 1\right] = a =
      \mathrm{non-negl}\left(k\right) \enspace.
    \end{equation*}
    We construct a PPT distinguisher \adversary{}
    (Fig.~\ref{proof:comb:master:distinguisher}) such that
    \begin{equation*}
      \Pr\left[\mathsf{EUF \mhyphen CMA}^{\adversary}\left(1^k\right) =
      1\right] = \mathrm{non-negl}\left(k\right)
    \end{equation*}
    that breaks the assumption, thus proving Lemma~\ref{lemma:comb:master}.
    }

    \begin{figure}[!htbp]
      \begin{algobox}{$\adversary\left(vk\right)$}
        \begin{algorithmic}[1]
          \State $j \overset{\$}{\gets} U\left[1, T\left(\mathcal{B}\right) +
          T\left(\adversary\right)\right]$
          \Comment{$T\left(M\right)$ is the maximum running time of $M$}
          \Indent
            \State Random Oracle: for every first-seen query $q$ from
            $\mathcal{B}$ set $\mathcal{H}\left(q\right)$ to a random value
            \State \Return $\mathcal{H}\left(q\right)$ to $\mathcal{B}$
          \EndIndent
          \State $\left(mpk, msk\right) \gets
          \textsc{MasterKeyGen}\left(1^k\right)$
          \Indent
            \State Random Oracle: Let $q$ be the $j$th first-seen query from
            $\mathcal{B}$ or \adversary:
            \If{$q = \left(mpk \concat x\right)$}
              \If{$\mathcal{H}\left(x \concat mpk\right)$ unset}
                \State set $\mathcal{H}\left(x \concat mpk\right)$ to a random
                value
              \EndIf
              \State set $\mathcal{H}\left(mpk \concat x\right)$ to $\left(vk
              - x \cdot \mathcal{H}\left(x \concat mpk\right)\right) \cdot
              mpk^{-1}$
            \ElsIf{$q = \left(x \concat mpk\right)$}
              \If{$\mathcal{H}\left(mpk \concat x\right)$ unset}
                \State set $\mathcal{H}\left(mpk \concat x\right)$ to a random
                value
              \EndIf
              \State set $\mathcal{H}\left(x \concat mpk\right)$ to $\left(vk
              - mpk \cdot \mathcal{H}\left(mpk \concat x\right)\right) \cdot
              x^{-1}$
            \Else
              \State set $\mathcal{H}\left(q\right)$ to a random value
            \EndIf
            \State \Return $\mathcal{H}\left(q\right)$ to $\mathcal{B}$ or
            \adversary
          \EndIndent
          \State $i \gets 0$
          \State $\left(\mathtt{aux}_i, \mathrm{response}\right) \gets
          \mathcal{B}\left(\textsc{init}, mpk\right)$
          \While{response can be parsed as $\left(pk, sk, m\right)$}
            \State $i \gets i + 1$
            \State store $pk, sk, m$ as $pk_i, sk_i, m_i$
            \State $\left(cpk_i, csk_i\right) \gets
            \textsc{CombineKey}\left(mpk, msk, pk_i, sk_i\right)$
            \State $\sigma_i \gets \textsc{SignCS}\left(m_i, csk_i\right)$
            \State $\left(\mathtt{aux}_i, \mathrm{response}\right) \gets
            \mathcal{B}\left(\textsc{signature}, \mathtt{aux}_{i-1},
            \sigma_i\right)$
          \EndWhile
          \State parse response as $\left(cpk^*, pk^*, m^*, \sigma^*\right)$
          \If{$vk = cpk^* \wedge \mathcal{B}$ wins the \textsf{master-EUF-CMA}
          game} \Comment{\adversary{} won the \textsf{EUF-CMA} game}
          \label{proof:comb:master:distinguisher:won}
            \State \Return $\left(m^*, \sigma^*\right)$
          \Else
            \State \Return \textsc{fail}
          \EndIf
        \end{algorithmic}
      \end{algobox}
      \caption{}
      \label{proof:comb:master:distinguisher}
    \end{figure}

    \redden{The modified random oracle used in
    Fig.~\ref{proof:comb:master:distinguisher} is indistinguishable from the
    standard random oracle for the same reasons as in the proof of
    Lemma~\ref{lemma:comb:share}.}

    \redden{Let $E$ denote the event in which \textsc{CombinePubKey} is not
    invoked to produce $cpk^*$. In that case the values $\mathcal{H}\left(pk^*
    \concat mpk\right)$ and $\mathcal{H}\left(mpk \concat pk^*\right)$ are
    decided after $\mathcal{B}$ terminates
    (Fig.~\ref{proof:comb:master:distinguisher},
    line~\ref{proof:comb:master:distinguisher:won}) and thus
    \begin{equation}
      \begin{gathered}
        \Pr\left[cpk^* = \textsc{CombinePubKey}\left(mpk, pk^*\right) |
        E\right] =
        \ifelseieee{\\ =}{}
        \mathrm{negl}\left(k\right) \Rightarrow \\
        \Pr\left[cpk^* = \textsc{CombinePubKey}\left(mpk, pk^*\right) \wedge
        E\right] =
        \ifelseieee{\\ =}{}
        \mathrm{negl}\left(k\right) \enspace.
      \end{gathered}
      \label{proof:comb:master:nocomb}
    \end{equation}
    We can reason like in the proof of Lemma~\ref{lemma:comb:share} to deduce
    that
    \begin{equation}
      \label{proof:comb:master:nohash}
      \Pr\left[\mathcal{B} \text{ wins} \wedge \neg E\right] > a -
      \mathrm{negl}\left(k\right) \enspace.
    \end{equation}
    }

    \redden{We now focus at the event $\neg E$. Let $F$ the event in which the
    call of to \textsc{CombinePubKey} that produces $cpk^*$ results in the $j$th
    invocation of the Random Oracle. Since $j$ is chosen uniformly at random and
    using Proposition~1 of~\cite{full-version}, $\Pr\left[F | \neg E\right] =
    \frac{1}{T\left(\mathcal{B}\right) + T\left(\adversary\right)}$. Observe
    that $\Pr\left[F | E\right] = 0 \Rightarrow \Pr\left[F\right] = \Pr\left[F |
    \neg E\right] = \frac{1}{T\left(\mathcal{B}\right) +
    T\left(\adversary\right)}$.}

    \redden{Once more we can reason in the same fashion as in the proof of
    Lemma~\ref{lemma:comb:share} to deduce that
    \begin{gather*}
      \Pr\left[F \wedge \mathcal{B} \text{ wins} \wedge \neg E\right] =
      \ifelseieee{\\ =}{}
      \Pr\left[F\right] \Pr\left[\mathcal{B} \text{ wins} \wedge \neg E\right]
      \pm \mathrm{negl}\left(k\right)
      \overset{\left(\ref{proof:comb:master:nohash}\right)}{=} \\
      \frac{a - \mathrm{negl}\left(k\right)}{T\left(\mathcal{B}\right) +
      T\left(\adversary\right)} \pm \mathrm{negl}\left(k\right) =
      \mathrm{non-negl}\left(k\right)
    \end{gather*}
    }
  \end{proof}

  The two results can then be combined to obtain the desired security property:

  \begin{theorem}
    \label{thm:combsign}
    The construction \redden{above} is \textsf{combine-EUF}-secure in the Random
    Oracle model under the assumption that the underlying signature scheme is
    strongly \textsf{EUF-CMA}-secure.
  \end{theorem}

  \redden{
  \begin{proof}
    The construction is \textsf{combine-EUF}-secure as a consequence of
    Lemma~\ref{lemma:comb:share}, Lemma~\ref{lemma:comb:master} and the
    definition of \textsf{combine-EUF}-security.
  \end{proof}
  }
