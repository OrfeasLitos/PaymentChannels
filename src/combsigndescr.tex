\section{The Combined Signature primitive}
\label{sec:ov-combined-ds}
  As previously mentioned, we define a new primitive for combining keys and
  generating signatures\redden{, which} is leveraged in the revocation and
  punishment mechanism of channel updates. Furthermore, we prove that the
  construction designed by the creators of LN realizes this primitive. We 
  provide here \redden{the concrete syntax and correctness definitions, along with}
  the intuition behind it. The exact security definitions, concrete construction
  and proof of security can be found in Appendix~\ref{appendix:combinedsign}.

  Previous work on the subject of multi-party
signatures~\cite{DBLP:conf/ndss/NicolosiKDM03,DBLP:journals/iacr/BellareS01,boyd1986digital,DBLP:conf/ndss/Ganesan95a,DBLP:conf/crypto/MacKenzieR01,ganesan1994secure}
  focuses on use-cases where some parties desire to generate a signature without
  revealing their private information; the latter is created using an
  interactive protocol. The resulting signatures can be verified by a single
  verification key, which is also included in the output of the key generation
  protocol. As we will see however, the primitive defined here has different
  aims and limitations and, to our knowledge, has not been formalized yet.

  A combined signature is a two-party primitive, say \redden{between} \alice{}
  and \bob\redden{,} with \bob{} being the signer and \alice{} the holder of a
  share of the secret key\redden{. This share} is essential for issuing
  signatures\redden{, which in turn are} verifiable with the ``combined''
  verification key. The verification key is \redden{generated} using public
  information drawn from \alice{} and \bob{} and is feasible without any party
  knowing the corresponding signing key. \redden{\bob{} will be able to
  construct the signing key only if \alice{} shares her secret information with
  him.}

  \redden{More specifically, the seven algorithms used by a Combined Signatures
  scheme are:
  \begin{itemize}
    \item $\left(mpk, msk\right) \gets \textsc{MasterKeyGen}\left(1^k\right)$
    \item $\left(pk, sk\right) \gets \textsc{KeyShareGen}\left(1^k\right)$
    \item $cpk_l \gets \textsc{CombinePubKey}\left(mpk, pk\right)$
    \item $\left(cpk_l, csk_l\right) \gets \textsc{CombineKey}\left(mpk,
    msk, pk, sk\right)$
    \item $\{0, 1\} \gets \textsc{TestKey}\left(pk, sk\right)$
    \item $\sigma \gets \textsc{SignCS}\left(m, csk\right)$
    \item $\left\{0, 1\right\} \gets \textsc{VerifyCS}\left(\sigma, m,
    cpk\right)$
  \end{itemize}}
  \redden{We demand that these three properties hold for a scheme to have
  correctness:}
  \begin{itemize}
    \item \redden{$\forall k \in \mathbb{N},$ \\
    $\Pr[\left(pk, sk\right) \gets \textsc{KeyShareGen}\left(1^k\right),$ \\
    $\textsc{TestKey}(pk, sk) = 1] = 1$}

    \redden{I.e. \textsc{KeyShareGen}() must always generate a valid keypair.}

    \item \redden{$\forall k \in \mathbb{N},$ \\
    $\Pr[(mpk, msk) \gets \textsc{MasterKeyGen}\left(1^k\right),$ \\
    $\left(pk, sk\right) \gets \textsc{KeyShareGen}\left(1^k\right),$ \\
    $\left(cpk_1, csk_1\right) \gets \textsc{CombineKey}\left(mpk, msk, pk,
    sk\right),$ \\
    $cpk_2 \gets \textsc{CombinePubKey}\left(mpk, pk\right),$ \\
    $cpk_1 = cpk_2] = 1$}

    \redden{I.e. for suitable input, \textsc{CombinePubKey}() and
    \textsc{CombineKey}() produce the same public key.}

    \item \redden{$\forall k \in \mathbb{N}, m \in \mathcal{M},$ \\
    $\Pr[(mpk, msk) \gets \textsc{MasterKeyGen}\left(1^k\right),$ \\
    $\left(pk, sk\right) \gets \textsc{KeyShareGen}\left(1^k\right),$ \\
    $\left(cpk, csk\right) \gets \textsc{CombineKey}\left(mpk, msk, pk,
    sk\right),$ \\
    $\textsc{VerifyCS}(\textsc{SignCS}(m, csk), m, cpk) = 1] = 1$}

    \redden{I.e. for suitable input, honestly generated signatures always verify
    correctly.}
  \end{itemize}

  Beyond correctness, combined signatures have two security properties expressed
  as follows. \redden{\textsf{Share-EUF}} security expresses security from the
  point of view of \alice{}, and establishes that \bob{} \redden{cannot issue} a
  valid combined signature if he does not possess \alice's corresponding secret
  share. In Appendix~\ref{appendix:combinedsign} we formalize it and prove:

  \begin{lemma}
    \label{lemma:comb:share}
    The construction defined in Appendix~\ref{appendix:combinedsign} is
    \textsf{share-EUF}-secure in the Random Oracle model under the assumption
    that the underlying signature scheme is strongly \textsf{EUF-CMA}-secure and
    the range of the Random Oracle coincides with that of the underlying
    signature scheme signing keys.
  \end{lemma}

  On the other hand, \redden{\textsf{master-EUF-CMA}} security is modeled very
  similarly to standard \redden{\textsf{EUF-CMA}} security, with the difference
  that \bob{} (the signer) combines malicious shares into his public key and
  issues signatures with respect to such combined keys\redden{. The security
  property ensures that these signatures} provide no advantage to the adversary
  in terms of producing a forged message for a combined key of its choice. In
  this respect we prove\redden{:}

  \begin{lemma}
    \label{lemma:comb:master}
    The construction of Appendix~\ref{appendix:combinedsign} is
    \textsf{master-EUF-CMA}-secure in the Random Oracle model under the
    assumption that the underlying signature scheme is strongly
    \textsf{EUF-CMA}-secure and the range of the Random Oracle coincides with
    that of the underlying signature scheme signing keys.
  \end{lemma}

  The two results can then be combined to obtain the desired security property:

  \begin{theorem}
    \label{thm:combsign}
    The construction of Appendix~\ref{appendix:combinedsign} is
    \textsf{combine-EUF}-secure in the Random Oracle model under the assumption
    that the underlying signature scheme is strongly \textsf{EUF-CMA}-secure.
  \end{theorem}
