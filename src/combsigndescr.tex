\section{Overview of our Combined Signature primitive} \TODO{drop this section,
combine with LN overview}
  As previously mentioned, we define in this work a new primitive for combining
  keys and generating signatures that is leveraged in the revocation and
  punishment mechanism of channel updates. Furthermore, we prove that the
  construction designed by the creators of LN realizes this primitive. We here
  provide the intuition behind it and refer the reader to
  Appendix~\ref{appendix:combinedsign} for the exact syntax, correctness and
  security definitions, concrete construction and proof of security.

  The central aim of the primitive is to enable derivation of a verification key
  using public information of \alice{} and \bob, without any party knowing the
  corresponding signing key. Only if \alice{} shares her secret information with
  \bob{} will he be able to construct the signing key.

  Let us now describe how this primitive is used. \alice{} derives a ``share
  keypair'' and \bob{} derives a ``master keypair''. They share the public part
  of their keypairs with each other and both can derive the ``combined''
  verification key. A signature valid by this verification key is enough to
  spend \alice's funds from her local commitment transaction. When the channel
  is updated and \alice{} wants to revoke this transaction, she sends her secret
  share to \bob. He then checks that \alice's public share is indeed derived
  from this secret and he can now compute the combined signature key. If
  \alice{} broadcasts the revoked commitment transaction, \bob{} can punish her
  by spending her output.
