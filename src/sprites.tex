\section{Sprites}
  Sprites~\cite{sprites} constitute an improvement upon Lightning~\cite{lightning}
  regarding the worst-case time needed to settle in case of a dispute. Consider a channel
  of $l$ hops, where $\Delta$ is the time given to each participant to publish their state
  after a counterparty has unilaterally broadcast theirs. The worst-case time to settle in
  the case of Lightning is $\Theta\left(l\Delta\right)$, whereas in Sprites it is
  $\Theta\left(l + \Delta\right)$.

  To achieve this, Sprites propose a smart contract called Preimage Manager (\texttt{PM}).
  Let $\mathcal{H}\left(\cdot\right)$ be a suitable hash function. Parties can interact
  with \texttt{PM} in the following way:
  \begin{itemize}
    \item Call $\mathtt{publish(}x\mathtt{)}$ at time $T$: \texttt{PM} stores
    $\mathtt{timestamp}\left[\mathcal{H}\left(x\right)\right] = T$.
    \item Call $\mathtt{published(}h, T\mathtt{)}$: \texttt{PM} returns \texttt{True} if
    \begin{itemize}
      \item $h \in \mathtt{timestamp}$ and
      \item $\mathtt{timestamp}\left[h\right] \leq T$,
    \end{itemize}
    \texttt{False} otherwise.
  \end{itemize}

  In case all parties are honest, \texttt{PM} is not invoked, the entire interaction
  happens off-chain and needs $l + 1$ rounds to complete. In case a party misbehaves by
  delaying sending the preimage until the last possible moment (i.e. time $\Delta$ after
  she received the preimage from the previous link), she will have to publish the preimage
  to the blockchain instead of just sharing it with the next link in the chain of
  payments in order to ensure she gets her funds. Thus, the rest of the (honest) players
  can settle the channel by asking \texttt{PM} whether the hash they already know has been
  \texttt{published()}. This action can be completed concurrently, thus the maximum delay
  that can be incurred is $l + \Delta$.
