\section{Overview of \fpaynet{}}
\label{sec:ov-paynet}
  One of our contributions is the specification of \fpaynet{}
  (Appendix~\ref{appendix:payfunc}) a functionality that describes the
  functional and security guarantees given by an ideal payment network. The
  central aim of \fpaynet{} is opening payment channels, keeping
  track of their state, updating them according to requested payments and
  closing them, as requested by honest players, all in a secure manner. In
  particular, the three main messages it can receive from \alice{} are
  (\textsc{openChannel}), (\textsc{pay}) and (\textsc{closeChannel}).

  When \fpaynet{} receives (\textsc{openChannel}, \alice, \bob, $x$,
  \textit{tid}) from \alice, it informs \simulator{} of \environment's intention
  to create a channel between \alice{} and \bob{} where \alice{} owns $x$ coins.
  When it receives (\textsc{pay}, \bob, $x, \overrightarrow{\mathtt{path}}$,
  \texttt{receipt}) from \alice, it informs \simulator{} that \environment{}
  asked to perform a multi-hop payment of $x$ coins from \alice{} to \bob{}
  along the $\overrightarrow{\mathtt{path}}$. As expected, when \fpaynet{}
  receives (\textsc{closeChannel}, \texttt{receipt}, \textit{tid}) from \alice,
  it leaks to \simulator{} the fact that \environment{} wants to close the
  relevant channel.

  In order to provide security guarantees, there are various moments when
  \fpaynet{} verifies whether certain expected events have actually taken place.
  A number of messages prompt \fpaynet{} to read from \ledger{} and perform
  these checks. In the actual implementations of LN these checks are done
  periodically by a polling daemon. Such checks are done by \fpaynet{} in the
  following cases:
  \begin{itemize}
    \item On receiving (\textsc{poll}) by \alice, \fpaynet{} asks \ledger{} for
    \alice's latest $\Sigma_{\alice}$ and verifies that no bad events have
    happened. In particular, \fpaynet{} halts if any of \alice's channels has
    been closed maliciously with a transaction at height $h$ and, even though
    \alice{} has \textsc{poll}ed within $[h, h + \mathtt{delay}(\alice) - 1]$,
    she did not manage to punish the counterparty. Refer to
    lines~\ref{alg:fpaynet:poll:ifdsforgery}
    and~\ref{alg:fpaynet:poll:haspolled} of Fig.~\ref{alg:fpaynet:poll} for more
    details. If \fpaynet{} does not halt, it leaks \simulator{} the polling
    details (including the identity of the poller and the state of the ledger in
    their view).
    \item \fpaynet{} expects \simulator{} to send a (\textsc{resolvePays},
    \texttt{charged}) message that gives details on the outcome of one or more
    multi-hop payments. \fpaynet{} checks that for each payment the charged
    party was (a) the one that initiated the payment, (b) a malicious party or
    (c) a honest party that is negligent. The latter case happens when the
    honest party either did not \textsc{poll} in time to catch a malicious
    closure similarly to the previous bullet (Fig.~\ref{alg:fpaynet:resolvepay},
    line~\ref{alg:fpaynet:resolvepay:haltcond:rel}) or did not \textsc{poll}
    twice while the block height in the view of the player was within
    $[\mathtt{OutgoingCltvExpiry}, \mathtt{IncomingCltvExpiry} - \tochain]$ with
    a distance of at least \tochain{} between the two \textsc{poll}s in order to
    learn the preimage from a correct closure, or did not enforce the retrieval
    of her funds by using the preimage to fulfill on chain when the chain in
    view had height $\mathtt{IncomingCltvExpiry} - \tochain$
    (Fig.~\ref{alg:fpaynet:resolvepay},
    line~\ref{alg:fpaynet:resolvepay:haltcond:abs}). \fpaynet{} ensures that the
    two expiries have a distance of at least $\mathtt{relayDelay}(\alice) +
    \tochain$. In case the charged party was honest and non-negligent,
    \fpaynet{} halts. It also halts if a particular payment resulted in a
    channel update for which \simulator{} did not inform \fpaynet{}
    (Fig.~\ref{alg:fpaynet:resolvepay:loops},
    line~\ref{alg:fpaynet:resolvepay:loops:halt:do}).
    \item \fpaynet{} executes the function
    \texttt{checkClosed}($\Sigma_{\alice}$) every time it receives
    $\Sigma_{\alice}$ from \ledger{} (Fig.~\ref{alg:fpaynet:close:func},
    lines~\ref{alg:fpaynet:close:func:start}-\ref{alg:fpaynet:close:func:end}).
    In this case, it checks that every channel that \environment{} has asked to
    be closed or \simulator{} designated as closed indeed has a closing
    transaction that corresponds to its latest state in $\Sigma_{\alice}$.
    Enough time is given for that transaction to settle in $\Sigma_{\alice}$,
    but if that time passes and the channel is still open or it is closed to a
    wrong version and no opportunity for punishment was given, \fpaynet{} halts.
  \end{itemize}

  A number of  messages that support the protocol progress are also handled:
  \begin{itemize}
    \item Every player has to send (\textsc{register}, delay, relayDelay) before
    participating in the network. This informs \fpaynet{} how often the player
    has to \textsc{poll}. ``delay'' corresponds to the maximum time between
    \textsc{poll}s so that malicious closures will be caught. ``relayDelay'' is
    useful when the player is an intermediary of a multi-hop payment. It roughly
    represents the size of the time window the player has to learn a preimage
    from the next and reveal it to the previous node. Subsequently \fpaynet{}
    asks \simulator{} to create and send a public key that will hold the
    player's funds. This public key is subsequently sent back to the player.
    \item To complete her registration, \alice{} has to send the
    (\textsc{toppedUp}) message. It lets \fpaynet{} know that the desired amount
    of initial funds have been transferred to \alice's public key. \fpaynet{}
    reads \alice's state on \ledger{} to retrieve this number and subsequently
    allows \alice{} to participate in the payment network after it updates her  \texttt{onChainBalance}.
    \item When \fpaynet{} receives (\textsc{checkForNew}, \alice, \bob,
    \textit{tid}) from \alice, it asks \ledger{} for \alice's latest state
    $\Sigma_{\alice}$ and looks for a funding transaction $F$ in it. If one is
    found, \simulator{} is asked to complete the opening procedure.
    \item (\textsc{pushFulfill}, \textit{pchid}),
    (\textsc{pushAdd}, \textit{pchid}) and (\textsc{commit}, \textit{pchid})
    all nudge \simulator{} to carry on with the protocol that updates the state
    of a specific channel. \fpaynet{} simply forwards these messages to
    \simulator.
    \item (\textsc{fulfillOnChain}) prompts \simulator{} to close channels in which
    the counterparty is not willing to pay, even though they have promised to do
    so upon disclosure of a particular preimage. This message is simply forwarded to
    \simulator{}, but \fpaynet{} takes a note that such a message was
    received and the current blockheight in the view of the calling party.
  \end{itemize}

  Last but not least, \environment{} sends (\textsc{getNews}) to obtain the
  latest changes regarding newly opened or closed channels, along with updates
  to the state of existing ones. Here we make an interesting observation: The
  most complex part of LN is arguably the negotiations that happen when a
  multi-hop payment takes place, due to the many channel updates needed; indeed,
  two complete channel updates for each hop are needed for a successful payment
  to go through. The fact that a proposal for an update can happen
  asynchronously with the commitment to this update, along with the fact that a
  single commitment may commit to many indiviual update proposals only adds to
  the complexity. It is therefore only natural to want \fpaynet{} to be unaware
  of these details. In order to disentangle the abstraction of \fpaynet{} from
  such minutiae, we allow the adversary full control of the
  updates that are reported back
  to \environment{} via  \fpaynet{}. 
  Nevertheless, \fpaynet{} enforces that 
  any   reporting deviations induced by the adversary will 
   be caught  when a channel
  closes.  This is quite intuitive: Consider
  a user of the payment network that does not understand its inner workings but
  can read \ledger{} and count her funds there. \fpaynet{} provides no guarantees 
  regarding any specific interim reporting but the user is assured that   in case she chooses to close  the relevant channel,  her state in
  \ledger{} will be consistent with all the payments that went through.
