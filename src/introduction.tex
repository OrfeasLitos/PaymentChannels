\section{Introduction}
  Payment channels are constructions that permit the secure exchange of assets between
  remote agents without the need for each transaction to be recorded in a global database.
  They are constructed in a way that gives the opportunity to the cheated agents to report
  the latest valid state to a global database (i.e. blockchain) and reclaim their assets.

  For example, imagine that $Alice$ works in $Bob$'s pin factory. They have agreed that
  $Alice$ be paid right after she makes each pin a small amount $x$ \cite{bdevguide}. This
  can add up to hundreds, even thousands small of payments each day. Since most
  cryptocurrencies impose fees per transaction, it would be a waste to broadcast a new
  transaction for each small payment. For this reason, they turn to payment channels.

  At the beginning of each month, $Bob$ creates a transaction that pays e.g. 100 coins (a
  bit more than $Alice$'s expected pay for the month) to himself. He builds it in a way
  that needs both his and $Alice$'s signature to be spent (i.e. 2-of-2 multisig). This is
  the ``bond'' transaction. $Alice$ confirms that the ``bond'' looks fine and gives $Bob$
  a special transaction that spends the ``bond''. This transaction is the ``refund''
  transaction. $Bob$ broadcasts the ``bond'' (but not yet the ``refund'') to the
  blockchain. The channel is now open.

  Every time $Alice$ makes a pin, $Bob$ pays $x$ to $Alice$ as follows: He creates a new
  ``refund'' that pays to $Alice$ the amount she already owned according to the previous
  ``refund'' plus $x$; accordingly, his payment is reduced by $x$. The total coins in
  the refund are always the same. He signs the new ``refund'' and sends it to $Alice$. She
  in turn signs the new ``refund'' and sends it back to $Bob$. The channel is now updated.

  Finally, the end of the month comes and $Alice$ wants to cash out on the blockchain, so
  that she can use her coins elsewhere. In order to do so, she simply broadcasts the
  latest ``refund''. The ``bond'' is spent according to the latest update, so she takes
  her rightful payment and $Bob$ takes the rest of the 100 initial coins. The channel is
  now closed.

  Note that exactly two transactions have been broadcast on the blockchain no matter how
  many payments were made, so the fees are kept low. Furthermore, both parties can
  unilateraly close the channel at any given point and claim the coins of their latest
  ``refund'', thus no trust is required between the two parties.
