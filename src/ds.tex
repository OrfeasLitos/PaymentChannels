\section{Digital Signatures}
  Digital signatures~\cite{katzlindell} enable a party to authenticate messages
  to other parties. A signature on a message is created by the signing party
  using the secret ``signing key''; other parties can later verify that the
  signature was indeed made on the message using the public ``verification
  key''. Transactions in Bitcoin~\cite{bitcoin} are signed using digital
  signatures and are considered valid only if signatures verify correctly, thus
  ensuring that only parties entitled to particular funds can spend them.
  Bitcoin uses ECDSA signatures over the secp256k1
  curve\footnote{\url{https://en.bitcoin.it/wiki/Secp256k1}}.

  To ensure compatibility, LN uses ECDSA over the same curve as its basic
  signature scheme. In this work, we abstract the particular construction away
  and use instead the established primitive that a secure construction must
  realize.

  The three algorithms used by a Digital Signatures scheme are:
  \begin{itemize}
    \item $(pk, sk) \gets \textsc{KeyGen}(1^k)$
    \item $\sigma \gets \textsc{SignDS}(m, sk)$
    \item $\{0, 1\} \gets \textsc{VerifyDS}(\sigma, m, pk)$
  \end{itemize}

  We demand that the following holds for a scheme to have correctness:

  $\forall k \in \mathbb{N}, m \in \mathcal{M},$ \\
  $\Pr[(pk, sk) \gets \textsc{KeyGen}(1^k),$ \\
  $\textsc{VerifyDS}(\textsc{SignDS}(m, sk), m, pk) = 1] = 1$

  \begin{figure}[!htbp]
    \begin{gamebox}{$\mathsf{EUF \mhyphen CMA}^{\adversary}\left(1^k\right)$}
      \begin{algorithmic}[1]
        \State $(pk, sk) \gets \textsc{KeyGen}(1^k)$
        \State $i \gets 0$
        \State $(\mathtt{aux}_i, \mathrm{response}) \gets
        \adversary(\textsc{init}, pk)$
        \While{response can be parsed as $m$}
          \State $i \gets i + 1$
          \State store $m$ as $m_i$
          \State $\sigma_i \gets \textsc{SignDS}(m, sk)$
          \State $(\mathtt{aux}_i, \mathrm{response}) \gets
          \adversary(\textsc{signature}, \mathtt{aux}_{i-1}, \sigma_i)$
        \EndWhile
        \State parse response as $(m^*, \sigma^*)$
        \If{$m^* \notin \{m_1, \dots, m_i\} \wedge \textsc{VerifyDS}(\sigma^*,
        m^*, pk) = 1$}
          \State \Return 1
        \Else
          \State \Return 0
        \EndIf
      \end{algorithmic}
    \end{gamebox}
    \caption{}
    \label{game:ds}
  \end{figure}
  \begin{definition}
    \label{def:ds:secure}
    A Digital Signatures scheme is \emph{strongly \textsf{EUF-CMA}-secure} if
    \begin{equation*}
      \forall k \in \mathbb{N}, \forall \adversary \in \mathtt{PPT},
      \Pr\left[\mathsf{EUF \mhyphen CMA}^{\adversary}\left(1^k\right) =
      1\right] = \mathit{negl}\left(k\right) \enspace.
    \end{equation*}
  \end{definition}

  Let $\mathrm{E \mhyphen ds}(k) = \underset{\adversary \in
  \mathtt{PPT}}{\sup}\{\Pr[\mathsf{EUF \mhyphen
  CMA}^{\adversary}\left(1^k\right) = 1]\}$. Then
  Definition~\ref{def:ds:secure} is equivalent to the following:

  \begin{definition}
    \label{def:ds:secure:sup}
    A Digital Signatures scheme is \emph{\textsf{EUF-CMA}-secure} if
    \begin{equation*}
      \forall k \in \mathbb{N}, \mathrm{E \mhyphen ds}(k) =
      \mathit{negl}\left(k\right) \enspace.
    \end{equation*}
  \end{definition}
