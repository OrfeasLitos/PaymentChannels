\section{The Ledger, Clock and Network Functionality}
  \label{appendix:ledger}
  We next provide the complete description of the ledger functionality as well as the clock and network functionalities  that are
drawn from the   UC formalisation of~\cite{BMTZ17,genesis}.

  The key characteristics of the functionality are as follows. The variable
  \state maintains the current immutable state of the ledger. An honest,
  synchronised party considers finalised a prefix of \state (specified by a
  pointer position $\pt_{i}$ for party $\stkh_i$ below). The functionality has a
  parameter \windowSize such that no finalised prefix of any player will be
  shorter than $|\state| - \windowSize$. On any input originating from an honest
  party the functionality will run the \LFextend function that ensures that a
  suitable sequence of transactions will be ``blockified'' and added to \state.
  Honest parties may also find themselves in a desynchronised state: this is
  when honest parties lose access to some of their resources. The resources that
  are necessary for proper ledger maintenance and that the functionality keeps
  track of are the global random oracle \Foracle, the clock \Fclock and network
  \FnetworkEd. If an honest party maintains registration with all the resources
  then after \vdelay clock ticks it necessarily becomes synchronised.

  The progress of the \state variable is guaranteed via the \LFextend function
  that is executed when honest parties submit inputs to the functionality. While
  we do not specify \LFextend in our paper (we refer to the citations above for
  the full specification) it is sufficient to note that \LFextend guarantees the
  following properties:
  \begin{enumerate}
    \item in a period of time equal to \maxTime, a number of blocks at least
    \windowSize are
    added to \state.
    \item in a period of time equal to $\minTime$, no more blocks may be added to $\state$ if $\windowSize$ blocks have been already added. 
    \item each window of \windowSize blocks has at most \advBlocksinWindowSize\/
    adversarial blocks included in it.
    \item any transaction that (i) is submitted by an honest party earlier than
    $\frac{\vdelay}{2}$ rounds before the time that the block that is
    \windowSize positions before the head of the \state was included, 
    and (ii) is valid with respect to an honest block that extends $\state$,
    then it must be included in such block. 
  \end{enumerate}

  Given a synchronised honest party, we say that a transaction \tx is
  finalised when it becomes a part of \state in its view.

  \begin{proposition}
    \label{prop:tochain}
    Consider any synchronised honest party that wishes to place a transaction
    \tx in some specific block height $[h+1, h+t-1]$ where $t$ is a parameter
    and $h$ an arbitrary positive integer. Then, as long as $t\geq 1+(2+r)\windowSize$, where $r = \lceil (\maxTime +  \frac{\vdelay}{2})/ \minTime \rceil$, \tx is
    guaranteed to be included in the intended block range as long as the party
    submits $\tx$ to the ledger functionality 
    by the time the block indexed by $h+ t - 1 - (2+r) \windowSize$
    is added to $\state$ in its view.
  \end{proposition}

  \begin{proof}
Consider $\tau^{\stkh}_{h+x}$ to be 
the round that a party $\stkh$  becomes aware of the 
$(h+x)$-th block in the $\state$. It follows that 
$\tau_{h+x} \leq \tau^{\stkh}_{h+x}$ where $\tau_{h+x}$ is the round the $(h+x)$ enters $\state$. Note that by time $\tau_{h+x} + \maxTime$ another $\windowSize$
blocks are added to $\state$ and thus $\tau^{\stkh}_{h+x} 
\leq \tau_{h+x}  + \maxTime$. 

Suppose $\stkh$ transmits the transaction $\tx$
to the ledger at time $\tau^{\stkh}_{h+x}$. Observe that as
long as $\tau_{h+x} + \maxTime$ is $\vdelay/2$ 
before the time that block with index $h+t-1-2\windowSize$ 
enters $\state$, then $\tx$ is guaranteed to enter the $\state$ in a 
block with index up to $h+t-1$ since $\advBlocksinWindowSize<\windowSize$. 
It follows we need
$\tau_{h+x}  + \maxTime < \tau_{h+t-1-2\windowSize} - \frac{\vdelay}{2} $. 
Let $r = \lceil (\maxTime +  \frac{\vdelay}{2})/ \minTime \rceil$.
Recall that in a period of $\minTime$ rounds at most $\windowSize$
blocks enter $\state$. As a result $r \windowSize$ blocks require
at least $r \minTime \geq \maxTime +  \frac{\vdelay}{2}$ rounds. 
It follows that if $t \geq 1+(2+r)\windowSize$
and $x = t - 1 - (2+r) \windowSize$
the inequality follows. 
  \end{proof}

  \begin{systembox}{$\Fledger$}\vspace{1ex}
    \small
    {\bf General:}
    The functionality is parameterized by four algorithms, \LFvalidate,
    \LFextend, \LFblockify, and \vsync, along with three parameters:
    $\windowSize, \vdelay\in\mathbb{N}$, and $\StakeHolderSet :=
    \{(\stkh_{1},\stake_{1}),\ldots,(\stkh_{n},\stake_{n})\}$.
    %
    The functionality manages variables $\state$ (the immutable state of the
    ledger), $\cbuffer$ (a list of transaction identifiers to be added to the
    ledger), \buffer (the set of pending transactions), $\LFtau$ (the rules
    under which the state is extended), and \tlast (the time sequence where all
    immutable blocks where added). The variables are initialized as follows:
    $\state:=\tlast:= \cbuffer:= \varepsilon$, $\buffer:=\emptyset$, $\LFtau=0$.
  %
    For each party $\party\in\PS$ the functionality maintains a pointer $\pt_i$
    (initially set to 1) and a current-state view $\state_p:=\varepsilon$
    (initially set to empty). The functionality also keeps track of the  timed
    honest-input sequence in a vector $\TIH$ (initially $\TIH:=\varepsilon$).
    \medskip

    {\bf Party Management:}
    The functionality maintains the set of registered parties $\PS$, the
    (sub-)set of honest parties $\HO \subseteq \PS$, and the (sub-set) of
    de-synchronized honest parties $\DSyncPS\subset\HO$ (as discussed below).
    The sets $\PS, \HS, \DSyncPS$ are all initially set to $\emptyset$. When a
    (currently unregistered) honest party is registered at the ledger, {\em if
    it is registered with the clock and the global RO already,} then it is added
    to the party sets $\HS$ and $\PS$ and the current time of registration is
    also recorded; if the current time is $\LFtau > 0$, it is also added to
    $\DSyncPS$. Similarly, when a party is deregistered, it is removed from both
    $\PS$ (and therefore also from $\DSyncPS$ or $\HS$). The ledger maintains
    the invariant that it is registered (as a functionality) to the clock
    whenever $\HS \neq \emptyset$.

    \medskip

    {\bf Handling initial stakeholders:}
    If during round $\tau = 0$, the ledger did not received a registration from
    each initial stakeholder, i.e., $\party \in \StakeHolderSet$, the
    functionality halts.

    \medskip
    \hrule
    \medskip

    %\emph{To be executed upon any activation:}
    {\bf Upon receiving any input $I$} from any party or from the adversary,
    send $(\ClockRead, \sidClock)$ to \Fclock and upon receiving response
    $(\ClockRead, \sidClock, \current)$ set $\LFtau:=\current$ and do the
    following if $\current > 0$ (otherwise, ignore input):

    \begin{enumerate}\setlength\itemsep{1.5ex}
      \item Updating synchronized/desynchronized party set:
      \begin{enumerate}
        \item Let $\hat{\PS}\subseteq\DSyncPS$  denote the set of desynchronized
        honest parties that have been registered (continuously) to the ledger,
        the clock, and the GRO since time $\tau' < \LFtau-\vdelay$. Set
        $\DSyncPS:=\DSyncPS\setminus\hat{\PS}$.
        \item For any synchronized party $\party\in\HO\setminus\DSyncPS$, if
        $\party$ is not registered to the clock, then consider it
        desynchronized, i.e., set $\DSyncPS\cup\{\party\}$.
      \end{enumerate}
      \item If $I$ was received from an honest party $\party\in\PS$:\\[1ex]
      \begin{enumerate}\setlength\itemsep{1ex}
        \item Set $\TIH:=\TIH||(I,\party,\LFtau)$;
        \item Compute
        $\vec{N}=(\vec{N}_1,\ldots,\vec{N}_\ell):=\LFextend(\TIH,\state,\cbuffer,\buffer,\tlast)$
        and if $\vec{N}\neq\varepsilon$ set
        $\state:=\state||\LFblockify(\vec{N}_1)||\ldots||\LFblockify(\vec{N}_\ell)$
        and $ \tlast:=\tlast||\LFtau^\ell$, where
        $\LFtau^\ell=\LFtau||\ldots,||\LFtau$.
        \item For each $\BTX\in\buffer$: if $\LFvalidate(\BTX,\state,\buffer)=0$
        then delete  $\BTX$ from $\buffer$. Also, reset $\cbuffer:=
        \varepsilon$.
        \item If there exists $\stkh_{j}\in\HS\setminus\DSyncPS$ such that
        $|\state| -\pt_{j} > \windowSize$ or $\pt_{j} < |\state_{j}|$, then set
        $\pt_{k}:=|\state|$ for all $\stkh_{k}\in\HS\setminus\DSyncPS$.
      \end{enumerate}

      \item If the calling party $\party$ is \emph{stalled or time-unaware}
      (according to the defined party classification), then no further actions
      are taken. Otherwise, depending on the above input $I$ and its sender's
      ID, $\Fledger$ executes the corresponding code from the following list:
      \begin{itemize}
        \let\labelitemi\labelitemii
        \item \emph{Submitting a transaction:}\\
        If $I=(\LFsubmit, \sid,\tx)$ and is received from a party $\party \in
        \PS$ or from \Adv (on behalf of a corrupted party $\party$) do the
        following%\\[0.5ex]
        \begin{enumerate}\setlength\itemsep{1ex}
          \item Choose a unique transaction ID $\txid$ and set
          $\BTX:=(\tx,\txid,\LFtau,\party)$
          \item If $\LFvalidate(\BTX,\state,\buffer)=1$, then
          $\buffer:=\buffer\cup\{\BTX\}$.
          \item Send $(\LFsubmit, \BTX)$ to \Adv.%\\[2ex]
        \end{enumerate}

        \item \emph{Reading the state:}\\ If $I=(\LFread, \sid)$ is received
        from a party $\party \in \PS$ then set $\state_p:=
        \state|_{\min\{\pt_p,|\state|\}}$ and return $(\LFread, \sid, \state_p)$
        to the requester. If the requester is \Adv then send
        $(\state,\buffer,\TIH)$ to~\Adv.%\\[2ex]

        \item \emph{Maintaining the ledger state:}\\
        If $I=(\LFmine,\sid, \minerID)$ is received by an honest party
        $\party\in\PS$ and  (after updating \TIH  as above)
        $\vsync(\TIH)=\hat{\tau}>\LFtau$ then send $(\ClockUp, \sidClock)$ to
        \Fclock. Else send $I$ to \Adv.%\\[2ex]


        \item \emph{The adversary proposing the next block:} \\
        If $I=(\LFmkcore, \honestyFlag, (\txid_1,\ldots,\txid_\ell))$ is sent
        from the adversary, update \cbuffer as follows:%\\[1ex]
        \begin{enumerate}\setlength\itemsep{1ex}
          \item Set $\mathrm{listOfTxid} \gets \epsilon$
          \item For $i=1,\ldots, \ell$ do: if there exists
          $\BTX:=(x,\txid,\minerID,\LFtau,\stkh_j)\in\buffer$ with ID
          $\txid=\txid_i$ then set
          $\mathrm{listOfTxid}:=\mathrm{listOfTxid}||\txid_i$.
          \item Finally, set $\cbuffer:= \cbuffer
          ||(\honestyFlag,\mathrm{listOfTxid})$ and output $(\LFmkcore, ok)$ to
          \Adv.%\\[2ex]
        \end{enumerate}
        \item \emph{The adversary setting state-slackness:}\\
        If $I=(\LFslack,(\stkh_{i_1},\hat{\pt}_{i_1}),\ldots,
        (\stkh_{i_\ell},\hat{\pt}_{i_\ell})),$ with
        $\{\party_{i_1},\ldots,\party_{i_\ell}\}\subseteq\HS\setminus\DSyncPS$
        is received from the adversary \Adv do the following:%\\[1ex]
        \begin{enumerate}\setlength\itemsep{1ex}
          \item If for all $j\in[\ell]:$  $|\state|
          -\hat{\pt}_{i_j}\leq\windowSize$ and $\hat{\pt}_{i_j}\geq
          |\state_{i_j}|$, set $\pt_{i_1}:=\hat{\pt}_{i_1}$ for every
          $j\in[\ell]$ and return $(\LFslack,ok)$ to \Adv.
          \item Otherwise set $\pt_{i_j}:=|\state|$ for all $j\in[\ell]$.
        \end{enumerate}
        \item \emph{The adversary setting the state for desychronized
        parties:}\\
        If $I=(\LFdsstate,(\stkh_{i_1},\state_{i_1}'),\ldots,
        (\stkh_{i_\ell},\state_{i_\ell}')),$ with
        $\{\stkh_{i_1},\ldots,\stkh_{i_\ell}\}\subseteq\DSyncPS$ is received
        from the adversary \Adv, set $\state_{i_j}:=\state_{i_j}'$ for each
        $j\in[\ell]$ and return $(\LFdsstate,ok)$ to \Adv.
      \end{itemize}
    \end{enumerate}
  \end{systembox}


  \ignore{ %%% WE SKIP THE BELOW FOR SIMPLICITY
  \subsection{Formal Specification of \LFextend for the PoS Ledger}\label{app:model_extpol}

  The detailed ExtendPolicy for Ouroboros is given below. % for completeness.
  %The extend policy is
  %parameterized by the ledger parameters that control the speed and the quality of
  %the ledger.

  \begin{algobox}{\LFextend for $\Fledger$}
    \begin{algorithmic}
      \Function{\text{$\LFextend$}}{$\TIH,\state,\cbuffer,\buffer,\tlast$}
    \State \Comment{First, create a default honest client block as alternative:}  
      \State $\defaultblock \gets$ \Call{DefaultExtension}{$\TIH,\LFstate,\cbuffer,\buffer,\tlast$} \Comment{Extension if adversary violates policy.}
      \State Let $\tau_L$ be current ledger time (computed from $\TIH$)  
      \State \Comment{The function must not have side-effects: Only modify copies of relevant values.}
      \State Create local copies of the values $\buffer$, $\state$, and $\tlast$.
    
      \State \Comment{Now, parse the proposed block by the adversary}
      \State Parse $\cbuffer$ as a vector $((\honestyFlag_1,\cbuffer_1), \cdots, (\honestyFlag_n, \cbuffer_n))$
      \State $\vec{N} \gets \varepsilon$
      \quad\Comment{Initialize Result}
      \State \Comment{Determine the time of the state block which is $\windowSize$ blocks behind the head of the state}
      \If{$|\state| \geq \windowSize$}
      \State Set $\timeLast \gets \tlast[|\state|-\windowSize + 1]$
      \Else
      \State Set  $\timeLast \gets 1$
      \EndIf
      \State $\mathsf{oldValidTxMissing} \gets \mathsf{false}$ \quad \Comment{Flag to keep track whether old enough, valid transactions are inserted.}
      \For{each list $\cbuffer_i$ of transaction IDs}  
      \State \Comment{Compute the next state block}
      %\State $\vec{N}_i \gets \varepsilon$ \TODO{Assignment never used?}
      \State \Comment{Verify validity of $\cbuffer_i$ and compute content}
      \State Use the $\txid$ contained in $\cbuffer_i$ to determine the list of transactions
      \State Let $\vec{\tx}=(\tx_1,\dots,\tx_{|\cbuffer_i|})$ denote the transactions of $\cbuffer_i$
      \If{$\tx_1$ is not a coin-base transaction}
      \State \Return $\defaultblock$
      \Else
      \State $\vec{N}_i \gets \tx_1$
      \For{$j=2$ to $|\cbuffer_i|$}
      \State Set $\st_i \gets \bitcoinBlockify(\vec{N}_i)$
      \If{$\bitcoinValid(\tx_j,\state||\st_i)=0$}
      \State  \Return $\defaultblock$
      \EndIf
      \State $\vec{N}_i \gets \vec{N}_i||\tx_j$
      \EndFor
      \State Set $\st_i \gets \bitcoinBlockify(\vec{N}_i)$
      \EndIf
      
      \State \Comment{Test that all old valid transaction are included}
      \If{the proposal is declared to be an honest block, i.e., $\honestyFlag_i=1$}
      \For{each $\BTX=(\tx,\txid,\tau',\party) \in \buffer$ of an honest party $\party$ with time $\tau' < \timeLast - \frac{\vdelay}{2}$}
      \If{$\bitcoinValid(\tx,\state||\st_i)=1$ but $\tx \not \in \vec{N}_i$}
      \State $\mathsf{oldValidTxMissing} \gets \mathsf{true}$
      \EndIf
      \EndFor
      \EndIf
      \State $\vec{N} \gets \vec{N}||\vec{N}_i$
      \State $\state \gets \state||\st_i$
      \State $\tlast \gets \tlast||\tau_L$
      \State \Comment{Must not proceed with too many adversarial blocks}
      \State $i \gets \max\lbrace \{\windowSize\} \cup \{k \ | \ \st_k \in \state \wedge \, \text{proposal of } \st_k \text{ had } \honestyFlag=1\}\rbrace$ \Comment{Determine most}
      \State \Comment{recent honestly-generated block in the interval behind the head.}
      \If{$|\state|-i \geq \advBlocksinWindowSize$}
      \State \Return $\defaultblock$
      \EndIf
      
      \State \Comment{Update $\timeLast$: the time of the state block which is $\windowSize$ blocks behind the head of the}
      \State \Comment{current, potentially already extended state}
      \If{$|\state| \geq \windowSize$}
      \State Set $\timeLast \gets \tlast[|\state|-\windowSize + 1]$
      \Else
      \State Set  $\timeLast \gets 1$
      \EndIf
      \EndFor
      
      %Now, the checks proceed.
      \State \Comment{Final checks (if policy is violated, it is enforced by the ledger):}
      \State \Comment{Must not proceed too slow or with missing transaction.}
      \If{$\timeLast > 0$ \textbf{and}  $\tau_L - \timeLast > \maxTime$} \Comment{A sequence of blocks cannot take too much time.}
      \State \Return $\defaultblock$
      \ElsIf{$\timeLast = 0$ \textbf{and}  $\tau_L - \timeLast > 2\cdot\maxTime$} \Comment{Bootstrapping cannot take too much time.}
      \State \Return $\defaultblock$
      \ElsIf{$\mathsf{oldValidTxMissing}$} \Comment{If not all old enough, valid transactions have been included.}
      \State \Return $\defaultblock$
      \EndIf
      \State \Return $\vec{N}$
      \EndFunction
    \end{algorithmic}
  \end{algobox}



  \begin{figure}
  \begin{algobox}{{\bf for Default State Extension}}
    \begin{algorithmic}
      \Function{{DefaultExtension}}{$\TIH,\LFstate,\cbuffer,\buffer,\tlast$}
      \State {\it \textcolor{gray}{We assume call-by-value and hence the function has no side effects.}}
      \State {\it \textcolor{gray}{The function returns a policy-compliant extension of the ledger state.}}
      \State
      \State Set $\defaultblock\gets \coinbase{\minerID}$ of an honest
      miner
      \State Sort $\buffer$ according to time stamps.
      \State Let $\vec{\tx}=(\tx_1,\dots,\tx_{\ell})$ be the transactions in $\buffer$
      \State Set $\st \gets \bitcoinBlockify(\defaultblock)$
      \Repeat
      \State Let $\vec{\tx}=(\tx_1,\dots,\tx_{\ell})$ be the current list of (remaining) transactions
      \For{$i=1$ to $\ell$}
      \If{$\bitcoinValid(\tx_i,\state||\st)=1$}
      \State $\defaultblock \gets \defaultblock||\tx_i$
      \State Remove $\tx_i$ from $\vec{\tx}$
      \State Set $\st \gets \bitcoinBlockify(\defaultblock)$
      \EndIf
      \EndFor
      \Until{$\defaultblock$ does not increase anymore}  
      %PG: I guess this is a forgotten comment from an older version?
      %\State \Comment{Possibly more than one block have to be added}
      \State \Comment{Let $\timeLast$ be the time of the block which is $\windowSize-1$ blocks behind the head of the state.}
      \If{$|\state| + 1 \geq \windowSize$}
      \State Set $\timeLast \gets \tlast[|\state|-\windowSize + 2]$
      \Else
      \State Set  $\timeLast \gets 1$ \Comment{First epoch starts at time $1$ (time $0$ is initialization time).}
      \EndIf
      \State $c \gets 1$
      \While{$\tau_L - \timeLast > \maxTime$}
      \State Set $\vec{N}_c \gets \coinbase{\minerID}$ of an honest miner
      \State $\defaultblock \gets \defaultblock||\vec{N}_c$
      \State $c \gets c+1$
      \State \Comment{Update $\timeLast$ to the time of the state block which is $\windowSize-c$ blocks behind the head.}
      \If{$|\state| + c \geq \windowSize$}
      \State Set $\timeLast \gets \tlast[|\state|-\windowSize + c + 1]$
      \Else
      \State Set  $\timeLast \gets 1$
      \EndIf
      \EndWhile  
      \State \Return $\defaultblock$
      \EndFunction
    \end{algorithmic}
  \end{algobox}
  \caption{Function to compute a policiy-compliant default ledger-state extension.}
  \label{fig:defaultExtension}
  \end{figure}
  }%%%%%%%%%%%


  \begin{systembox}{Functionality $\Fclock$}
    The functionality manages the set $\PS$ of registered identities, i.e.,
    parties $\party = (\pid,\sid)$. It also manages the set $F$ of
    functionalities (together with their session identifier). Initially,
    $\PS:=\emptyset$ and $F := \emptyset$.

    \smallskip
    For each session $\sid$ the clock maintains a variable $\current_\sid$. For
    each identity $\party := (\pid,\sid) \in \PS$ it manages variable
    $d_\party$. For each pair $(\Func,\sid) \in F$ it manages variable
    $d_{(\Func,\sid)}$ (all integer variables are initially $0$).

    \medskip

    \emph{Synchronization:}
    \begin{itemize}
      \item Upon receiving $(\ClockUp, \sidClock)$ from some party $\party \in
      \PS$ set $d_{\party}:=1$; execute \emph{Round-Update} and forward
      $(\ClockUp, \sidClock, \party)$  to $\Adv$.

      \item Upon receiving $(\ClockUp, \sidClock)$ from some functionality
      $\Func$ in a session $\sid$ such that $(\Func,\sid) \in F$ set
      $d_{(\Func,\sid)}:=1$, execute \emph{Round-Update} and return $(\ClockUp,
      \sidClock,\Func)$ to this instance of $\Func$.

      \item Upon receiving $(\ClockRead, \sidClock)$ from any participant
      (including the environment on behalf of a party, the adversary, or any
      ideal---shared or local---functionality) return $(\ClockRead,
      \sid,\current_\sid)$ to the requestor (where $\sid$ is the sid of the
      calling instance).
    \end{itemize}

    \emph{Procedure Round-Update:}
    For each session $\sid$ do:
    If $d_{(\Func,\sid)}:=1$ for all $\Func \in F$ and $d_{\party}=1$ for all
    honest parties $\party=(\cdot,\sid) \in \PS$, then set
    $\current_\sid:=\current_\sid+1$  and reset $d_{(\Func,\sid)}:=0$ and
    $d_{\party}:=0$ for all parties $\party=(\cdot,\sid) \in \PS$.
  \end{systembox}


\begin{systembox}{$\FnetworkEd^{\Delay}$}
	The functionality is parameterized with a set possible senders and receivers $\PS$. Any newly registered (resp. deregistered) party is added to  (resp. deleted from) $\PS$. 
	
	{\small
		\begin{itemize}
			\item \textbf{Honest sender multicast.} Upon receiving $(\NetworkMulticast,\sid,m)$ from some $\party\in\PS$, where $\PS=\{\stkh_1,\ldots,\stkh_n\}$ denotes the current party set, choose $n$ new unique message-IDs $\mid_1,\ldots,\mid_{n}$, initialize $2n$ new variables $D_{\mid_1}:=D_{\mid_1}^{MAX}\ldots:=D_{\mid_n}:=D_{\mid_n}^{MAX}:=1$, set $\vec{M}:=\vec{M}||(m,\mid_1,D_{\mid_1},\stkh_1)||\ldots||(m,\mid_n,D_{\mid_n},\stkh_n)$, and send $(\NetworkMulticast,\sid,m,\party,(\stkh_1,\mid_1),\ldots,(\stkh_n,\mid_n))$ to the adversary.
			\smallskip
			\item \textbf{Adversarial sender (partial) multicast.}
			Upon receiving $(\NetworkMulticast,\sid,(m_{i_1},\stkh_{i_1}),\ldots,(m_{i_\ell},\stkh_{i_\ell})$ from the adversary with $\{\stkh_{i_1},\ldots,\stkh_{i_\ell}\}\subseteq\PS$, choose $\ell$ new unique message-IDs $\mid_{i_1},\ldots,\mid_{i_\ell}$, initialize $\ell$ new variables $D_{\mid_{i_1}}:=D_{\mid_{i_1}}^{MAX}:=\ldots:=D_{\mid_{i_\ell}}:=D_{\mid_{i_\ell}}^{MAX}:=1$, set $\vec{M}:=\vec{M}||(m_{i_1},\mid_{i_1},D_{\mid_{i_1}},\stkh_{i_1})||\ldots||(m_{i_\ell},\mid_{i_\ell},D_{\mid_{i_\ell}},\stkh_{i_\ell})$, and send $(\NetworkMulticast,\sid,(m_{i_1},\stkh_{i_1},\mid_{i_1}),\ldots, (m_{i_\ell},\stkh_{i_\ell},\mid_{i_\ell})$ to the adversary.   
			
			\smallskip
			\item \textbf{Honest party fetching.}
			Upon receiving $(\NetworkFetch,\sid)$ from $\party \in\PS$ (or from $\Adv$ on behalf of $\party$ if $\party$ is corrupted):
			\begin{enumerate}
				\item For all tuples  $(m,\mid,D_\mid,\party)\in\vec{M}$, set $D_\mid:=D_\mid-1$.
				\item Let $\vec{M}_0^{\party}$ denote the subvector $\vec{M}$ including all tuples of the form $(m,\mid,D_\mid,\party)$ with $D_\mid=0$ (in the same order as they appear in $\vec{M}$).  
				Delete all entries in $\vec{M}^{\party}_0$ from $\vec{M}$,  and send $\vec{M}^{\party}_0$ to $\party$.
				
			\end{enumerate}
			
			\smallskip
			\item \textbf{Adding adversarial delays.}
			Upon receiving $(\NetworkDelays,\sid, (T_{\mid_{i_1}},\mid_{i_1}), \ldots, (T_{\mid_{i_\ell}},\mid_{i_\ell}))$ from the adversary do the following for each pair $(T_{\mid_{i_j}},\mid_{i_j})$: \\
			If $D_{\mid_{i_j}}^{MAX}+T_{\mid_{i_j}}\leq\Delta$ and $\mid$ is a message-ID registered in the current $\vec{M}$, set $D_{\mid_{i_j}}:= D_{\mid_{i_j}} + T_{\mid_{i_j}}$ and set $D_{\mid_{i_j}}^{MAX}:= D_{\mid_{i_j}}^{MAX} + T_{\mid_{i_j}}$; otherwise, ignore this pair.   
			
			\smallskip
			\item \textbf{Adversarially reordering messages.}
			Upon receiving $(\NetworkSwap,\sid, \mid, \mid')$ from the adversary, if $\mid$ and $\mid'$ are message-IDs registered in the current $\vec{M}$, then swap the triples $(m,\mid,D_\mid,\cdot)$ and $(m,\mid', D_{\mid'},\cdot)$ in $\vec{M}$. Return $(\NetworkSwap,\sid)$ to the adversary. 
		\end{itemize}
	}
\end{systembox}

