\section{Lightning Network high level overview}
  \subsection{Two-party channels}
    The aim of LN is to enable fast and cheap transactions that do not have to
    be added to the blockchain, without compromising security. Specifically no
    additional trust between counterparties is assumed. This is achieved in the
    following way: Two parties, \alice{} and \bob, that have recurring monetary
    exchanges create one on-chain transaction that locks up some funds, known as
    the ``funding transaction''. This transaction is funded by one of the two
    parties and has a 2-of-2 multisig output, which needs the signatures of both
    counterparties by their ``funding'' secret keys in order to be spent. Before
    actually submitting this transaction though, both parties individually
    ensure that they hold a transaction that spends the 2-of-2 funding output in
    a way that gives the funds to the funder, along with the signature of this
    transaction with the counterparty's funding key. These two transactions (one
    for each counterparty) are called ``commitment transactions''. Each party
    can broadcast her ``local'' commitment transaction and has signed the
    ``remote'' commitment transaction, which is the one held by the
    counterparty.

    Every time they want to make a payment to each other, they exchange a
    sequence of messages (that include specially crafted signatures) that have
    two effects.  Firstly, a new pair of commitment transactions, along with
    their signatures by the funding keys, is created. Each of these transactions
    ensures that, if broadcast, each party will be able to spend the appropriate
    share from the coins contained in the funding output.  Secondly, the two old
    commitment transactions are revoked.  This ensures that no party can close a
    channel using an old commitment transaction which may be more beneficial to
    her than the latest one.

    Invalidating past commitment transactions requires some care. The reason is
    that it is impossible to actually make past commitment transactions invalid
    without spending the funding output; however, spending it would need an
    on-chain transaction for each channel update, thus essentially defeating the
    purpose of LN. The following idea is leveraged instead: If \alice{}
    broadcasts an old commitment and \bob{} sees it, he can punish \alice{} by
    taking all the money in the channel. Therefore \alice{} is technically able
    to broadcast an old commitment transaction, but has no financial benefit in
    doing so. At the same time this imposes the requirement that parties are
    vigilant about observing the blockchain --- see below when we talk about
    time-locks and how they facilitate a  time window within which  parties
    should react.

    The punishing mechanism operates as follows. Suppose \alice{} considers
    posting her old local commitment transaction which  has an output that
    carries her old share of the funds. This output can be spent in two ways:
    either with a signature by \alice's ``delayed payment'' secret key which is
    a usual ECDSA key, or with a signature by \bob's ``revocation'' secret key,
    which is also an ECDSA key, but with an additional characteristic that we
    will explain soon. Thus, if \alice{} broadcasts an old commitment
    transaction, \bob{} will be able to obtain her funds by spending her output
    using his ``revocation'' key. This privilege of course opens the possibility
    for \bob{} to abuse it (in particular, when a channel is closed --- see
    below --- \bob{} may steal \alice's funds by using such revocation key) and
    hence this side effect should also be carefully mitigated. The mitigation
    has the following form. At the time of creation of a new commitment, both
    parties will know \bob's ``revocation'' public key, but no party knows its
    corresponding secret -- the key can only be computed by combining one secret
    value that \alice{} knows and one secret value that \bob{} knows. \alice{}
    therefore can prevent \bob{} from using his revocation key until she  sends
    this secret value to him. As a result, \alice{} will send the secret value
    to \bob{} only after the new commitment transaction is built and signed.
    Thus \bob{} cannot abuse his ability to use the revocation key on a
    commitment transaction before this transaction is revoked. We note that the
    underlying cryptographic mechanism  that enables such ``revocation keys''
    is not straightforward and, as part of our contributions, we formalise it as
    a new two-party cryptographic primitive. We call the primitive ``combined
    signature'' and we prove that the construction hidden in the LN
    implementation realizes it in the random oracle model under the assumption
    that the underlying digital signature scheme is secure.

    The last element needed to make updates secure is the so called ``relative
    timelock''. If \alice{} broadcasts a commitment transaction, she is not
    allowed to immediately spend her funds with her ``delayed payment'' key.
    Instead, she has to wait for the transaction to reach a pre-agreed block
    depth (the relative timelock, hardcoded in the output script of the
    commitment transaction) in order to give some time to \bob{} to see the
    transaction and, if it does not correspond to the latest version of the
    channel, punish her with his ``revocation'' key. This avoids a scenario in
    which \alice{} broadcasts an old commitment transaction and immediately
    spends her output, which would prevent \bob{} from ever proving that this
    commitment was old.

    Lastly, if \alice{} wants to unilaterally close a channel, all she has to do
    is broadcast her latest local commitment transaction (the only one that is
    not revoked) and wait for the timelock to expire in order to spend her
    funds. The LN specification allows for cooperative channel closure which
    avoids the need to wait for the timelock, but in the current work this last
    type of closure is not considered.

    As mentioned time locks provide specific time windows within which both
    parties have to be vigilant in order to punish a misbehaving  counterparty
    who  broadcasts an old commitment transaction. This means that parties have
    to be regularly online to safeguard against theft. LN makes it possible to
    trustlessly outsource this, but this mechanism is not analyzed in the
    current work.

  \subsection{Multi-hop payments}

    Having funds locked down for exclusive use with a particular counterparty
    would be a serious limitation. LN goes beyond that by allowing multi-hop
    payments. In a situation where \alice{} has a channel with \bob{} and he has
    another channel with \charlie{}, it is possible for \alice{} to pay
    \charlie{} off-chain by leveraging \bob's help. Remarkably, this can be
    achieved without any one party trusting any of the other two. One can think
    of \alice{} giving some ``marked'' money to \bob{}, who in turn either
    delivers it to \charlie{} or returns it to \alice{} -- it is impossible for
    \bob{} to keep the money. It is also impossible for \alice{} and \charlie{}
    to make \bob{} pay for this transaction out of his pocket.

    We will now give an informal overview of how this counterintuitive dynamic
    is made possible. \alice{} initiates the payment by asking \charlie{} to
    create a new hash for a payment of $x$ coins. \charlie{} chooses a random
    secret, hashes it and sends the hash to \alice. \alice{} promises \bob{} to
    pay him $x$ in their channel if he shows her the preimage of this particular
    hash within a specific time frame. \bob{} makes the same promise to
    \charlie{}: if \charlie{} tells \bob{} the preimage of the same hash within
    a specific time frame (shorter than the one \bob{} has agreed with
    \alice{}), \bob{} will pay him $x$ in their common channel. \charlie{} then
    sends him the preimage (which is the secret he generated initially) and
    \bob{} agrees to update the channel to a new version where $x$ is moved from
    him to \charlie. Similarly, \bob{} sends the preimage to \alice{} and once
    again \alice{} updates their channel to give \bob{} $x$ coins. Therefore $x$
    coins were transmitted from \alice{} to \charlie{} and \bob{} did not gain
    or lose anything, he just increased his balance in the channel with \alice{}
    and reduced his balance by an equal amount in the channel with \charlie.

    This type of promise where a preimage is exchanged for money is called Hash
    TimeLocked Contract (HTLC). It is enforceable on-chain in case the payer
    does not cooperatively update upon disclosure of the preimage, thus no trust
    is needed. In the previous example with \alice{}, \bob{} and \charlie, two
    HTLCs were signed and fulfilled for the payment to go through and the whole
    interaction was completely off-chain. Two updates happened in each channel:
    one to sign the HTLC and one to fulfill it. The time frames were chosen so
    that every intermediary has had the time to learn the preimage and give it
    to the previous party on the path.

    Direct payments are also carried out using HTLCs.

    LN gives the possibility for intermediaries to charge a fee for their
    service, but such fees are not incorporated in the current analysis.
    Furthermore, LN leverages the Sphinx onion packet scheme~\cite{sphinx} to
    increase the privacy of payments, but we do not formaly analyze the privacy
    of LN in this work -- we just use it in our protocol description to
    syntactically match the message format used by LN.
