\section{Lightning Network overview}
\label{sec:ov-ln}

  \noindent {\bf Two-party channels.}
    The aim of LN is to enable fast, cheap, off-chain transactions, without
    compromising security. Specifically no trust between counterparties is
    needed. This is achieved as follows: Two parties that plan to have recurring
    monetary exchanges lock up some funds with one special on-chain transaction.
    We say that they opened a new channel. They can then transact with the
    locked funds multiple times solely by interacting privately, without
    informing the blockchain. If they want to use their funds in the usual,
    on-chain way again, they have to close the channel and unlock the funds with
    one more on-chain transaction. Each party can unilaterally close the channel
    and retrieve the coins they are entitled to -- according to the latest
    channel state -- and thus neither party has to trust the other.

    In more detail, to open a channel \alice{} and \bob{} first exchange a
    number of keys and desired timelocks $\mathrm{relDel}_A, \mathrm{relDel}_B$
    (explained below) and then build locally some transactions. The ``funding
    transaction'' $F$ contains a 2-of-2 multisig output with one ``funding''
    public key $pk_{F, A}, pk_{F, B}$ for each counterparty. This multisig
    output needs signatures for both designated public keys in order to be
    spent. $F$ is funded with $c_F$ coins that belong only to one of the two
    parties, say \alice.

    \begin{figure}[H]
    \centering
    \begin{pspicture}
      \drawtx{funding}{
        {{},{}}%
      }{
        {{$pk_{F{,} A} \wedge pk_{F{,} B}$},{$c_F$}}%
      }
    \end{pspicture}
    \label{fig:ln:funding}
    \caption{Funding TX (on-chain): Rules over, coins below output.}
    \end{figure}

    Each party also builds a slightly different version of the ``commitment
    transaction'' $C_A, C_B$. \alice{} uses her ``delayed payment'' key
    $pk_{\mathrm{dcom}, A}$ and \bob's ``revocation'' key $pk_{\mathrm{rev}, B}$
    (received before), whereas \bob{} uses \alice's ``payment'' key
    $pk_{\mathrm{com}, A}$ (received before). Both $C_A$ and $C_B$ spend the
    funding output of $F$ and allow \alice{} to retrieve her funds if she acts
    honestly, as we will explain shortly. \alice{} sends \bob{} the signature of
    $C_B$ made with her $sk_{F, A}$ and vice-versa.

    \begin{figure}[H]
    \centering
    \begin{pspicture}
      \drawtx{$\mathrm{comm}_A$}{
        {{$\sigma_{F{,} A} \wedge \sigma_{F{,} B}$},{funding}}%
      }{
        {{$pk_{\mathrm{rev}{,} B} \vee$,
        $(\mathrm{relDel}_B \wedge pk_{\mathrm{dcom}{,} A})$},{$c_F$}}%
      }
    \end{pspicture}
    \label{fig:ln:commitment:alice}
    \caption{\alice's initial commitment TX (off-chain): Required data over
    input, spent output below input.}
    \end{figure}

    \begin{figure}[H]
    \centering
    \begin{pspicture}
      \drawtx{$\mathrm{comm}_B$}{
        {{$\sigma_{F{,} A} \wedge \sigma_{F{,} B}$},{funding}}%
      }{
        {{$pk_{\mathrm{com}{,} A}$},{$c_F$}}%
      }
    \end{pspicture}
    \label{fig:ln:commitment:bob}
    \caption{\bob's initial commitment TX (off-chain): All coins belong to
    \alice, so she can immediately spend them if \bob{} closes.}
    \end{figure}

    She now broadcasts $F$; once both parties see that it is confirmed, they
    generate and exchange new ``commitment'' keys (used for updating the channel
    later) and the channel is open.

    Every time they want to make a payment to each other, they exchange a series
    of messages that have two effects. First, a new pair of commitment
    transactions, along with their signatures by the funding keys, is created,
    one for each counterparty. Each of these transactions ensures that, if
    broadcast, each party will be able to spend the appropriate share from the
    coins contained in the funding output. Second, the two old commitment
    transactions are revoked. This ensures that no party can close a channel
    using an old commitment transaction which may be more beneficial to her than
    the latest one.

    Invalidating past commitments requires some care. The reason is that it is
    impossible to actually make past commitments invalid without spending the
    funding output on-chain; doing this for every update would however defeat
    the purpose of LN. The following idea is leveraged instead: If \alice{}
    broadcasts an old commitment and \bob{} sees it, he can punish \alice{} by
    taking all the money in the channel. Therefore \alice{} is technically able
    to broadcast an old commitment, but has no financial benefit in doing so.
    The same reasoning holds if \bob{} broadcasts an old commitment. On the
    donwside this imposes the requirement that parties must observe the
    blockchain periodically --- see below the explanation of timelocks and how
    they facilitate a time window within which parties should react.

    The punishing mechanism operates as follows. Suppose \alice{} considers
    posting her old local commitment which has an output that carries her old
    share of the funds. This output can be spent in two ways: either with a
    signature by \alice's ``delayed payment'' secret key $sk_{\mathrm{dcom}, A}$
    which is a usual ECDSA key, or with a signature by \bob's ``revocation''
    secret key $sk_{\mathrm{rev}, B}$, which is also an ECDSA key, but with an
    additional characteristic that we will explain soon. Thus, if \alice{}
    broadcasts an old commitment, \bob{} will be able to obtain her funds by
    spending her output using his ``revocation'' key. This privilege of course
    opens the possibility for \bob{} to abuse it (in particular, when a channel
    is closed --- see below --- \bob{} may steal \alice's funds by using such a
    revocation key) and hence this side effect should also be carefully
    mitigated. The mitigation has the following form. At the time of creation of
    a new commitment, both parties will know \bob's ``revocation'' public key
    $pk_{\mathrm{rev}, B}$, but no party knows its corresponding secret -- the
    key can only be computed by combining one secret value that \alice{} knows
    $sk_{\mathrm{com}, n, A}$ and one secret value that \bob{} knows
    $sk_{\mathrm{com}, n, B}$. \alice{} therefore can prevent \bob{} from using
    his revocation key until she sends him $sk_{\mathrm{com}, n, A}$. Therefore,
    \alice{} will send \bob{} $sk_{\mathrm{com}, n, A}$ only after the new
    commitment transaction is built and signed. Thus \bob{} cannot abuse his
    revocation key on a commitment before this transaction is revoked. We note
    that the underlying cryptographic mechanism  that enables such ``revocation
    keys'' is not straightforward and, as part of our contributions, we
    formalise it as a new two-party cryptographic primitive. We call it
    ``combined signature'' and we prove that the construction hidden in the LN
    implementation realizes it in the random oracle model under the assumption
    that the underlying digital signature scheme is secure in
    Appendix~\ref{appendix:combinedsign}.

    The last element needed to make channel updates secure is the already
    mentioned ``relative timelock''. If \alice{} broadcasts a commitment
    transaction, she is not allowed to immediately spend her funds with her
    ``delayed payment'' key. Instead, she has to wait for the transaction to
    reach a pre-agreed block depth (the relative timelock, negotiated during
    the opening of the channel and hardcoded in the output script of the
    commitment transaction) in order to give some time to \bob{} to see the
    transaction and, if it does not correspond to the latest version of the
    channel, punish her with his ``revocation'' key. This avoids a scenario in
    which \alice{} broadcasts an old commitment transaction and immediately
    spends her output, which would prevent \bob{} from ever proving that this
    commitment was old.

    Lastly, if \alice{} wants to unilaterally close a channel, all she has to do
    is broadcast her latest local commitment (the only one not revoked) and any
    outstanding HTLC transactions (explained below) and wait for the timelock to
    expire in order to spend her funds. The LN specification further allows for
    cooperative channel closure, achieved by negotiating and broadcasting the
    ``closing transaction'' which is not encumbered with a timelock, providing
    immediate availability of funds.

    As mentioned timelocks provide specific time windows within which both
    parties have to check the blockchain in order to punish a misbehaving
    counterparty who  broadcasts an old commitment transaction. This means that
    parties have to be regularly online to safeguard against theft. Furthermore,
    LN makes it possible to trustlessly outsource this to so-called watchtowers,
    but this mechanism is not analyzed in the current work.

  \noindent {\bf Multi-hop payments.}
    Having funds locked down for exclusive use with a particular counterparty
    would be a serious limitation. LN goes beyond that by allowing multi-hop
    payments. In a situation where \alice{} has a channel with \bob{} and he has
    another channel with \charlie{}, it is possible for \alice{} to pay
    \charlie{} off-chain by leveraging \bob's help. Remarkably, this can be
    achieved without any one party trusting any of the other two. One can think
    of \alice{} giving some ``marked'' money to \bob{}, who in turn either
    delivers it to \charlie{} or returns it to \alice{} -- it is impossible for
    \bob{} to keep the money. It is also impossible for \alice{} and \charlie{}
    to make \bob{} pay for this transaction out of his pocket.

    We will now give a brief overview of how this counterintuitive dynamic
    is made possible. \alice{} initiates the payment by asking \charlie{} to
    create a new hash for a payment of $x$ coins. \charlie{} chooses a random
    secret, hashes it and sends the hash to \alice. \alice{} promises \bob{} to
    pay him $x$ in their channel if he shows her the preimage of this particular
    hash within a specific time frame. \bob{} makes the same promise to
    \charlie{}: if \charlie{} tells \bob{} the preimage of the same hash within
    a specific time frame (shorter than the one \bob{} has agreed with
    \alice{}), \bob{} will pay him $x$ in their common channel. \charlie{} then
    sends him the preimage (which is the secret he generated initially) and
    \bob{} agrees to update the channel to a new version where $x$ is moved from
    him to \charlie. Similarly, \bob{} sends the preimage to \alice{} and once
    again \alice{} updates their channel to give \bob{} $x$ coins. Therefore $x$
    coins were transmitted from \alice{} to \charlie{} and \bob{} did not gain
    or lose anything, he just increased his balance in the channel with \alice{}
    and reduced his balance by an equal amount in the channel with \charlie.

    This type of promise where a preimage is exchanged for money is called
    Hashed Time Locked Contract (HTLC). It is enforceable on-chain in case the
    payer does not cooperatively update upon disclosure of the preimage, thus no
    trust is needed. It is realised as an additional output of the commitment
    transactions, which contains the specified hash and transfers its funds
    either to the party that should provide the preimage or to the other party
    after a timeout. A corresponding ``HTLC transaction'' that can spend this
    output is built by each party. In the previous example with \alice{}, \bob{}
    and \charlie, two HTLC transactions were signed and fulfilled in total for
    the payment to go through. Two updates happened in each channel: one to sign
    the HTLC and one to fulfill it. The time frames were chosen so that every
    intermediary has had the time to learn the preimage and give it to the
    previous party on the path.

    In LN zero-hop payments are also carried out using HTLCs.

    LN gives the possibility for intermediaries to charge a fee for their
    service, but such fees are not incorporated in the current analysis.
    Furthermore, LN leverages the Sphinx onion packet scheme~\cite{sphinx} to
    increase the privacy of payments, but we do not formaly analyze the privacy
    of LN in this work -- we just use it in our protocol description to
    syntactically match the message format used by LN.
