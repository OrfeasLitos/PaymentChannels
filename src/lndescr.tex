\section{Lightning Network high level overview}
  The aim of LN is to enable fast and cheap transactions that do not have to be
  added to the blockchain, without compromising security. This is achieved in
  the following way: Two parties that have recurring monetary exchanges create
  one on-chain transaction that locks up some funds, known as the ``funding
  transaction''. This transaction is funded by one of the two parties and has a
  2-of-2 multisig output, which needs the signatures of both counterparties in
  order to be spent. Before actually submitting this transaction though, both
  parties individually ensure that they hold a transaction that spends the
  2-of-2 output in a way that gives the funds to the funder, along with its
  signature from the counterparty. These two transactions (one for each
  counterparty) are called ``commitment transactions''. Each party can broadcast
  her ``local'' commitment transaction and has signed the ``remote'' commitment
  transaction, which is the one held by the counterparty.

  Every time they want to make a payment to each other, they exchange a sequence
  of messages (that include specially crafted signatures) that have two effects.
  Firstly, a new pair of commitment transactions, along with their signatures,
  is exchanged. Each of these transactions ensures that, if broadcast, each
  party will be able to spend their righteous share from the coins contained in
  the funding output. Secondly, the two old commitment transactions are revoked.
  This ensures that no party can close a channel using an old commitment
  transaction that is more beneficial to her than the latest one. Observe that
  it is impossible to actually make these commitment transactions invalid
  without spending the funding output; but spending it would need an on-chain
  transaction for each update, thus defeating the purpose of LN.

  The following technique is used instead: \alice's old local commitment has an
  output that carries her old share of the funds. This output can be spent in
  two ways: either with a signature by \alice's ``delayed payment'' secret key
  which is a usual ECDSA key, or with a signature by \bob's ``revocation''
  secret key, which is a new type of key. If \alice{} broadcasts an old
  commitment transaction, \bob{} will be able to take her funds by spending her
  output using his ``revocation'' key. At the time of creation of a new
  commitment, both parties know \bob's ``revocation'' public key, but no party
  knows its secret part -- it can only be computed by combining one secret that
  \alice{} knows and one secret that \bob{} knows. \alice{} therefore sends this
  secret to \bob{} after the new commitment transaction is built and signed, in
  order to revoke the old commitment. Thus \bob{} cannot abuse this output
  before an old commitment transaction is revoked. A new cryptographic
  construction has been designed by the creators of LN in order to enable such
  ``revocation'' keys, which we define as a new primitive named ``combined
  signature'' and prove that their construction realizes it.

  The last element needed to make updates secure is the so called ``relative
  timelock''. If \alice{} broadcasts a commitment transaction, she is not
  allowed to immediately spend her funds with her ``delayed payment'' key.
  Instead, she has to wait for the transaction to reach a pre-agreed block depth
  (the relative timelock, hardcoded in the output script of the commitment
  transaction) in order to give some time to \bob{} to see the transaction and,
  if it does not correspond to the latest version of the channel, punish her
  with his ``revocation'' key. This avoids a scenario in which \alice{}
  broadcasts an old commitment transaction and immediately spends her output,
  which would prevent \bob{} from ever proving that this commitment was old.

  If \alice{} wants to unilaterally close a channel, all she has to do is
  broadcast her latest local commitment transaction and wait for the timelock to
  expire in order to spend her funds. The LN specification allows for
  cooperative channel closure which avoids the need to wait for the timelock,
  but in the current work cooperative closure is not considered.
