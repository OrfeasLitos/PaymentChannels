\begin{abstract}
%  Blockchains have enabled the decentralized exchange of digital currencies with
%  minimal trust assumptions~\cite{bitcoin,gkl}. Nevertheless, the massive
%  replication of data amongst participating nodes has a detrimental effect on
%  scalability. One approach to addressing the issue is layer-2
%  payment networks, which enable users to perform the vast majority of their
%  transactions without adding them to the
%  blockchain~\cite{lightning,decker,teechan,perun}. The most widely adopted
%  layer-2 solution, which is deployed on top of Bitcoin, is the Lightning
%  Payment Network (LN)~\cite{lightning}. In this work we prove that LN is secure
%  in the Universal Composability framework~\cite{uc}. We formally describe the
%  LN
%  protocol\footnote{\url{https://github.com/lightningnetwork/lightning-rfc/}},
%  abstract its intended purpose in a functionality and prove that the specified
%  protocol UC-realizes it.
The high latency and low throughput of blockchain protocols constitute one of the fundamental barriers for their wider adoption. Overlay protocols, notably the {\em lightning network}, have been touted as the most viable direction for rectifying this in practice. In this work we present for the first time a full formalisation and security analysis of the lightning network in the (global) universal composition setting that takes into account a global ledger functionality for which previous work [Crypto'17] has demonstrated its realisability by the Bitcoin blockchain protocol. As a result, our treatment delineates exactly how the security guarantees  of the protocol depend on the properties of the underlying ledger. Moreover, we provide a complete and modular description of the core of the  lightning protocol that highlights precisely its dependency to underlying basic cryptographic primitives such as digital signatures, pseudorandom functions, identity-based signatures and a less common two-party primitive, which we term a combined digital signature, that were originally  hidden within the lightning protocol's implementation. 
\end{abstract}
