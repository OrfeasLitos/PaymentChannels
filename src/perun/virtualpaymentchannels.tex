\subsection{Virtual payment channels}
  Virtual payment channels are channels created on top of suitable preexisting multistate
  channels that facilitate trustless funds exchange between parties that do not share an
  on-chain channel. Going into more detail, a virtual payment channel $\gamma$ is a tuple:
  \begin{gather*}
    \left(\gamma\mbox{.\texttt{id}}, \gamma\mbox{.\texttt{Alice}},
    \gamma\mbox{.\texttt{Ingrid}}, \gamma\mbox{.\texttt{Bob}},
    \gamma\mbox{.\texttt{cash}}, \gamma\mbox{.\texttt{subchan}}, \right. \\
    \left.\gamma\mbox{.\texttt{validity}}, \gamma\mbox{.\texttt{ver-num}},
    \gamma\mbox{.\texttt{sign}}\right)
  \end{gather*}
  Let $Alice$ have a multistate channel with $Ingrid$ ($Alice
  \overset{\gamma_a}{\Leftrightarrow} Ingrid$); also let $Bob$ have a multistate channel
  with $Ingrid$ ($Bob \overset{\gamma_b}{\Leftrightarrow} Ingrid$).

  \subsubsection{Channel creation} \ \\

    To build $\gamma$, two nanocontracts $\nu_a$ and $\nu_b$ are created, each on the
    corresponding multistate channel. $\nu_a$ has
    $\gamma\mbox{.\texttt{cash}}\left(Alice\right)$ blocked by $Alice$ and
    $\gamma\mbox{.\texttt{cash}}\left(Bob\right)$ blocked by $Ingrid$. Similarly, $\nu_b$
    has $\gamma\mbox{.\texttt{cash}}\left(Alice\right)$ blocked by $Ingrid$ and
    $\gamma\mbox{.\texttt{cash}}\left(Bob\right)$ blocked by $Bob$. At a high level, a
    virtual payment channel creation protocol is as follows:
    \begin{enumerate}
    \setcounter{enumi}{-1}
      \item $Alice$ and $Bob$ discover that $Ingrid$ is an intermediary. They also agree
      on the initial balance of $\gamma$.
      \item $Alice$ sends a signed $\nu_a$ to $Ingrid$.
      \item $Ingrid$ sends a signed $\nu_b$ to $Bob$.
      \item $Bob$ replies to $Ingrid$ with $\nu_b$, signed by the former.
      \item $Ingrid$ replies to $Alice$ with $\nu_a$, signed by the former.
    \end{enumerate}

    We now say that there exists the virtual payment channel $\gamma$ between $Alice$ and
    $Bob$ ($Alice \overset{\gamma}{\leftrightarrow} Bob$).

  \subsubsection{Channel update} \ \\

    Updating the cash balance of the channel can be accomplished in the same way as for
    the basic payment channels, with both $Alice$ and $Bob$ signing the new state with an
    incremented version number. Note that, in contrast to Lightning, the end-users do not
    need to interact with $Ingrid$ at all to update the channel. This decreases the number
    of rounds needed for an update to 2 in the optimistic case that both $Alice$ and $Bob$
    are honest. Furthermore, it somewhat increases the privacy of the end-users.

  \subsubsection{Closing the channel} \ \\

    In case all three parties are honest, they agree that they want to close $\gamma$ and
    $\gamma\mbox{.\texttt{validity}}$ time has not yet passed, then both $Alice$ and $Bob$
    attempt to terminate their respective nanocontract with $Ingrid$, $\nu_a$ and $\nu_b$.
    The end-users send their latest version of $\gamma$ to the intermediary, who expects a
    tuple with valid signatures by the initially registered end-users and a total balance
    equal to that of the initial state of the channel. If both end-users send different
    valid tuples, then $Ingrid$ chooses the one with the higher version number as valid.
    Thus both nanocontracts can be terminated with each of the end-users unblocking their
    respective balance, as defined by the valid $\gamma$, on their multistate channel with
    $Ingrid$. The sum of coins $Ingrid$ will unblock in both multistate channels will be
    equal to the sum of the original coins she blocked during the virtual payment channel
    creation, only redistributed between the two channels as defined by the latest
    $\gamma$ state.

    $Alice$ can unilaterally register the $\nu_a$ nanocontract state on the ledger and
    provide her latest $\gamma$ version, thus she does not need $Ingrid$'s cooperation to
    unblock her funds. If $Ingrid$ learns a newer valid $\gamma$ version from $Bob$ (which
    means that $Alice$ tried to cheat), then $Ingrid$ can publish it within a
    predetermined timeframe and claim her rightful funds.

    Furthermore, if the channel hasn't closed after $\gamma\mbox{.\texttt{validity}}$ time
    has passed, any of the three parties can unilaterally finalize the nanocontract(s) she
    has access to and unblock the respective funds.

    Thus we have seen that no trust between the parties is necessary. Similarly to
    Lightning though, cooperating parties can unblock their funds faster and with less
    interaction with the ledger (and thus lower fees).
