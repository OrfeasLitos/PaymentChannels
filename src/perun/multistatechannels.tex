\subsection{Multistate channels}
  A basic multistate channel is a tuple
  \begin{equation*}
    \gamma = \left(\gamma\mbox{.\texttt{id}}, \gamma\mbox{.\texttt{Alice}},
    \gamma\mbox{.\texttt{Bob}}, \gamma\mbox{.\texttt{cash}},
    \gamma\mbox{.\texttt{nspace}}\right) \enspace,
  \end{equation*}
  where $\gamma\mbox{.\texttt{id}}$, $\gamma\mbox{.\texttt{Alice}}$,
  $\gamma\mbox{.\texttt{Bob}}$ and $\gamma\mbox{.\texttt{cash}}$ are as in a payment
  channel and $\gamma\mbox{.\texttt{nspace}}$ is a set of nanocontracts, or
  \textit{nanocontracts space}.

  The multistate channel $\gamma$ has a corresponding contract
  $\mbox{\texttt{MSContract}}_{\gamma\mbox{.\texttt{id}}}$ on the ledger. The end-users
  have to interact with this contract upon channel creation, channel closure and in case
  of dispute over the state of a nanocontract. Note that the end-users can create new
  nanocontracts, as well as cooperatively update them, without touching the ledger.

  A nanocontract $\nu \in \gamma\mbox{.\texttt{nspace}}$ is a tuple
  \begin{equation*}
    \nu = \left(\nu\mbox{.\texttt{nid}}, \nu\mbox{.\texttt{blocked}},
    \nu\mbox{.\texttt{storage}}, \nu\mbox{.\texttt{ver-num}},
    \nu\mbox{.\texttt{sign}}\right) \enspace,
  \end{equation*}
  where $\nu\mbox{.\texttt{nid}}$ is a globally unique identifier of the nanocontract,
  $\nu\mbox{\texttt{.blocked}}$ is a function from the end-users of the multistate channel
  to a real non-negative value that denotes the amount of cash the end-user has in the
  nanocontract and $\nu\mbox{\texttt{.storage}}$ contains the storage of the nanocontract.
  Like simple payment channels, the nanocontract with the highest
  $\nu\mbox{\texttt{.ver-num}}$ and a valid $\nu\mbox{\texttt{.sign}}$ will be accepted by
  the blockchain in case of registration of the state of the nanocontract on the ledger.

  The update mechanism for a nanocontract is similar to the update mechanism of a simple
  payment channel and thus will not be explained in detail. The only substantial
  differences are the following:
  \begin{enumerate}
    \item After $Alice$ proposes a nanocontract update, $Bob$ has time $\Upsilon$ to reply
    whether he agrees with this update or not. If he agrees the update goes through, else
    the state of the nanocontract is not updated (apart from increasing the version
    number). This is not considered a dispute, so (on-chain) nanocontract state
    registration does not need to take place.
    \item In case of a sucessful update, the cash balance of both end-users in the
    underlying multistate channel ($\gamma\mbox{.\texttt{cash}}\left(Alice\right)$ and
    $\gamma\mbox{.\texttt{cash}}\left(Alice\right)$) are updated to reflect the fact that
    the nanocontract update has consumed or returned some funds to the end-users.
  \end{enumerate}

  Each nanocontract has its own $\nu\mbox{.\texttt{ver-num}}$ and
  $\nu\mbox{.\texttt{sign}}$ field, so that several nanocontracts of the same multistate
  channel can be updated in parallel. Let $\nu'$ be the state of the nanocontract $\nu$
  after an update. The only requirement is that
  \begin{align*}
    \nu'\mbox{\texttt{.blocked}}\left(Alice\right) &+
    \nu'\mbox{\texttt{.blocked}}\left(Bob\right) \leq \\
    \nu\mbox{\texttt{.blocked}}\left(Alice\right) &+
    \nu\mbox{\texttt{.blocked}}\left(Bob\right) \enspace.
  \end{align*}
  This ensures that no two nanocontracts will together require more funds than are
  available in the multistate channel and thus that all nanocontracts can be updated in
  parallel. If it is the case that $\nu\mbox{\texttt{.blocked}}\left(Alice\right) +
  \nu\mbox{\texttt{.blocked}}\left(Bob\right) = 0$, we say that the nanocontract $\nu$ is
  \texttt{terminated}.

  To create a new nanocontract $\nu$, users simply apply the update mechanism. They have
  to ensure that $\nu\mbox{.\texttt{nid}}$ is a new, globally unique identifier and that
  the $\nu\mbox{.\texttt{ver-num}} = 0$.

  Registration of the state of a nanocontract on \texttt{MSContract} happens in case of
  dispute with regard to the state of the nanocontract or when the parties want to close
  the multistate channel. The registration mechanism is very similar to that of closing a
  simple payment channel, so will not be described in detail.

  Given that a nanocontract $\nu$ is registered on the \texttt{MSContract} of the
  underlying multistate channel of $\nu$, any end-user can unilaterally execute a
  nanocontract function \texttt{fun} by calling the \texttt{MSContract} function
  $\mbox{\texttt{execute}}\left(\nu\mbox{.\texttt{nid}}, \mbox{\texttt{fun}}, z\right)$.
  \texttt{MSContract} then updates the state of the nano\-contract on the ledger and
  returns the output to the end-users.

  Finally, when the end-users wish to close the multistate channel, they have to update
  all nanocontracts such that they are \texttt{terminated}, register their state (or
  alternatively register their state and then execute functions on the ledger until they
  are \texttt{terminated}) and then initiate a procedure similar to the closing of basic
  payment channels, which gives the opportunity to both end-users to publish their latest
  version of all nanocontracts of the multistate channel. The nanocontract states with the
  highest version number are accepted by the ledger as valid. Each end-user receives coins
  equal to the initial coins they contributed to the multistate channel, amended by the
  changes introduced by the nanocontracts. These coins are now available to use with other
  users of the ledger.

  It may be the case that some nanocontracts cannot be updated to a \texttt{terminated}
  state due to dispute between end-users or design problems of the nanocontract. The
  end-users can have special provision in \texttt{MSContract} for such cases to be able to
  kill such misbehaving nanocontracts and distribute the funds in a predefined manner.
  Such a mechanism is not explicitly specified.
