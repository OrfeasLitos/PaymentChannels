\section{Perfect Ledger Unrealisability Proof}
\label{appendix:perfectledger}
  \begin{proof}[Proof of Theorem~\ref{theorem:perfectledger}]
    We first define the offending environment and adversary and subsequently
    show how they can distinguish the ideal from the real world.
    \begin{figure}[H]
      \begin{titlebox}{Environment \normalfont \perfectenv}{commonbox}{normal}
        Spawn two players, \alice{} and \bob. Flip a coin. If it returns 0,
        execute \texttt{writeWithPlayer}, otherwise execute
        \texttt{writeWithAdversary}.
        \begin{algorithmic}[1]
          \Procedure{writeWithPlayer}{}
            \State First activation:
            \Indent
              \State choose random number $m \overset{\$}{\gets} \{0, 1\}^k$
              \State assign at random names \alice, \bob{} to two players
              \State send (\textsc{submit}, $m$) to \alice
            \EndIndent
            \State Second activation:
            \Indent
              \State send (\textsc{read}) to \bob
              \If{\bob{} does not give subroutine output}
                \State \Return 0 \Comment{real world}
                \label{fig:perfectledger:env:coin0:real}
              \ElsIf{\bob's subroutine output $\mathcal{L}$ contains $m$}
                \State \Return 1 \Comment{players communicated}
                \label{fig:perfectledger:env:coin0:comm}
              \ElsIf{$\mathcal{L}$ does not contain $m$}
                \State \Return 0 \Comment{players did not communicate}
                \label{fig:perfectledger:env:coin0:nocomm}
              \EndIf
            \EndIndent
          \EndProcedure
          \Statex

          \Procedure{writeWithAdversary}{}
            \State First activation:
            \Indent
              \State choose random number $m \overset{\$}{\gets} \{0, 1\}^k$
              \State assign at random names \alice, \bob{} to two players
              \State send (\textsc{leak}, $m$, \alice) to \adversary{}
              \Comment{in real world \adversary{} will \textsc{multicast} to
              \alice}
            \EndIndent
            \State Second activation:
            \Indent
              \State send (\textsc{read}) to \bob
            \EndIndent
            \State Third activation:
            \Indent
              \If{\bob{} does not give subroutine output}
                \State \Return 0 \Comment{real world}
                \label{fig:perfectledger:env:coin1:real1}
              \ElsIf{\bob's subroutine output $\mathcal{L}_{\bob}$ contains $m$}
                \State \Return 1 \Comment{ideal world}
                \label{fig:perfectledger:env:coin1:ideal1}
              \EndIf
              \State send (\textsc{read}) to \alice
              \If{\alice{} does not give subroutine output}
                \State \Return 0 \Comment{real world}
                \label{fig:perfectledger:env:coin1:real3}
              \ElsIf{\alice's subroutine output $\mathcal{L}_{\alice}$ contains
              $m$}
                \State \Return 0 \Comment{real world}
                \label{fig:perfectledger:env:coin1:real2}
              \ElsIf{$\mathcal{L}_{\alice}$ does not contain $m$}
                \State \Return 1 \Comment{ideal world or real \alice{}
                misbehaving}
                \label{fig:perfectledger:env:coin1:ideal2}
              \EndIf
            \EndIndent
          \EndProcedure
        \end{algorithmic}
      \end{titlebox}
      \caption{}
      \label{fig:perfectledger:env}
    \end{figure}

    \begin{figure}[H]
      \begin{titlebox}{Adversary \normalfont \perfectadv}{commonbox}{normal}
        Upon receiving (\textsc{leak}, $m$, \alice) from \perfectenv, simulate
        \perfectprot{} reacting to (\textsc{submit}, $m$). If it attempts to
        send a message (\textsc{multicast}, $m'$) to $\FnetworkEd^1$, send
        (\textsc{multicast}, ($m'$, \alice)) to $\FnetworkEd^1$.
      \end{titlebox}
      \caption{}
      \label{fig:perfectledger:adv}
    \end{figure}

    Since we quantify over all possible \simulator{} and \perfectprot, we have
    to refer to the probabilities of them taking specific actions of interest:
    \begin{gather*}
      \ppsubmits = \Pr[\text{Upon receiving (\textsc{submit}, $m$) from
      \environment,} \\
      \ifelseieee{
        \text{\perfectprot{} sends (\textsc{multicast}, $f(m)$) to} \\
        \text{$\FnetworkEd^1$ for some function $f$}] \\
      }{
        \text{\perfectprot{} sends (\textsc{multicast}, $f(m)$) to
        $\FnetworkEd^1$ for some function $f$}] \\
      }
      \ppfetches = \Pr[\text{Upon receiving (\textsc{read}) from \environment,}
      \\
      \text{\perfectprot{} sends (\textsc{fetch}) to $\FnetworkEd^1$ for data }
      m' \\
      \text{and sends back to \environment{} a (\textsc{read}, $\mathcal{L}$)
      such that} \\
      \text{if there is a unique element $m$ in $\mathcal{L}$, it is } f(m) =
      m']
    \end{gather*}

    We first analyze the event in which the initial coin flip of \environment{}
    results in 0, $\mathrm{Coin}_0$. In the ideal world, the submitted message
    $m$ always ends up in the ledger right away and therefore when
    \environment{} has \bob{} \textsc{read}, it always sees $m$ in the answer,
    therefore (Fig.~\ref{fig:perfectledger:env},
    line~\ref{fig:perfectledger:env:coin0:in})
    \begin{equation*}
      \Pr[\textsc{Exec}^{\perfectledger, \gFclock}_{\simulator, \perfectenv} = 1
      | \mathrm{Coin}_0] = 1 \enspace.
    \end{equation*}

    In the real world, in order for the submitted message $m$ to be in \bob's
    response to \textsc{read}, he must have fetched from $\FnetworkEd^1$ and
    considered this data as a new ledger entry, and \alice{} must have sent some
    function of $m$ to $\FnetworkEd^1$ when she received (\textsc{submit}, $m$),
    except if he could guess $m$, which can happen with negligible probability,
    therefore
    \ifelseieee{
      \begin{gather*}
        \Pr[\textsc{Exec}^{\FnetworkEd^1, \gFclock}_{\perfectprot, \perfectadv,
        \perfectenv} = 1 | \mathrm{Coin}_0] < \\
        < \ppsubmits \ppfetches + \negl(k) \enspace.
      \end{gather*}
    }{
      \begin{equation*}
        \Pr[\textsc{Exec}^{\FnetworkEd^1, \gFclock}_{\perfectprot, \perfectadv,
        \perfectenv} = 1 | \mathrm{Coin}_0] < \ppsubmits \ppfetches + \negl(k)
        \enspace.
      \end{equation*}
    }

    We now move on to the event in which the initial coin flip results in 1,
    $\mathrm{Coin}_1$. In the ideal world, if \simulator{} \textsc{submit}s the
    received $m$ to the ledger then \environment's \textsc{read} request to
    \bob{} will be answered with an output that contains $m$ an \environment{}
    will output 1 (Fig.~\ref{fig:perfectledger:env},
    line~\ref{fig:perfectledger:env:coin1:ideal1}). If on the other hand
    \simulator{} does not \textsc{submit} it, then neither \bob's nor \alice's
    answer will contain $m$, so \environment's output will also be 1
    (Fig.~\ref{fig:perfectledger:env},
    line~\ref{fig:perfectledger:env:coin1:ideal2}).
    \begin{equation*}
      \Pr[\textsc{Exec}^{\perfectledger, \gFclock}_{\simulator, \perfectenv} = 1
      | \mathrm{Coin}_1] = 1 \enspace.
    \end{equation*}

    Lastly, in the real world, \bob's buffer in $\FnetworkEd^1$ does not
    contain any information, so he may return a ledger containing $m$ only with
    negligible probability. In case he returns a ledger without $m$, \alice{}
    will respond to \environment's \textsc{read} query with a ledger containing
    $m$ exactly in the case that the event that defines \ppfetches{} is true,
    therefore
    \ifelseieee{
      \begin{gather*}
        \Pr[\textsc{Exec}^{\FnetworkEd^1, \gFclock}_{\perfectprot, \perfectadv,
        \perfectenv} = 1 | \mathrm{Coin}_1] < \\
        < (1 - \ppfetches) + \negl(k) \enspace.
      \end{gather*}
    }{
      \begin{equation*}
        \Pr[\textsc{Exec}^{\FnetworkEd^1, \gFclock}_{\perfectprot, \perfectadv,
        \perfectenv} = 1 | \mathrm{Coin}_1] < (1 - \ppfetches) + \negl(k)
        \enspace.
      \end{equation*}
    }
    Note that \perfectprot{} cannot leverage knowledge of its own pid in order
    to have \alice{} behave differently from \bob{} in a manner that tricks
    \perfectenv{} into believing that it interacts with the ideal world (i.e.
    make \alice{} also not return a ledger that contains $m$) because the roles
    of \alice{} and \bob{} are assigned and the coin is flipped secretly at
    random by \perfectenv.

    In aggregate,
    \begin{gather*}
      \Pr[\textsc{Exec}^{\FnetworkEd^1, \gFclock}_{\perfectprot, \perfectadv,
      \perfectenv} = 1] = \\
      \frac{1}{2}(\Pr[\textsc{Exec}^{\FnetworkEd^1, \gFclock}_{\perfectprot,
      \perfectadv, \perfectenv} = 1 | \mathrm{Coin}_0] +
      \ifelseieee{\\ +}{}
      \Pr[\textsc{Exec}^{\FnetworkEd^1, \gFclock}_{\perfectprot, \perfectadv,
      \perfectenv} = 1 | \mathrm{Coin}_1]) < \\
      \frac{1}{2}(\ppsubmits \ppfetches + 1 - \ppfetches) + \negl(k) = \\
      \frac{1}{2} + \ppfetches \frac{\ppsubmits - 1}{2} + \negl(k) \enspace,
    \end{gather*}
    and
    \begin{gather*}
      \Pr[\textsc{Exec}^{\perfectledger, \gFclock}_{\simulator, \perfectenv} =
      1] = \\
      \frac{1}{2}(\Pr[\textsc{Exec}^{\perfectledger, \gFclock}_{\simulator,
      \perfectenv} = 1 | \mathrm{Coin}_0] +
      \ifelseieee{\\ +}{}
      \Pr[\textsc{Exec}^{\perfectledger, \gFclock}_{\simulator, \perfectenv} = 1
      | \mathrm{Coin}_1]) = 1 \enspace.
    \end{gather*}

    For these two probabilities to be equal (which is necessary and sufficient
    for indistinguishability to hold), it would have to be
    $\ppfetches(\ppsubmits - 1) = 1$. One can verify that there is no assignment
    to the two probabilities that satisfies this equation and maintains both
    values within $[0, 1]$. Therefore, the real and the ideal world are
    distinguishable.
  \end{proof}
