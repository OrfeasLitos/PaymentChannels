\section{Security proof overview}
\label{sec:ov-security-proof}
  \begin{theorem}[Lightning Payment Network Security]
The protocol $\Pi_{\mathrm{LN}}$ realises $\fpaynet$
given a global functionality $\ledger$
assuming the security of the underlying digital signature, identity-based signature, 
combined digital signature and PRF. Specifically, 
    \label{theorem:simulation}
    \begin{gather*}
      \forall k \in \mathbb{N}, \text{ PPT } \environment, 
      |\Pr[\textsc{Exec}^{\ledger}_{\Pi_{\mathrm{LN}}, \adversary_{\mathrm{d}},
      \environment} = 1] - \Pr[\textsc{Exec}^{\fpaynet, \ledger}_{\simulator,
      \environment} = 1]| \leq \\
      2nm  \mathrm{E \mhyphen ds}(k)  + 6np  \mathrm{E \mhyphen ids}(k) +
      2nmp  \mathrm{E \mhyphen share}(k) + 2nm \mathrm{E \mhyphen master}(k)+  2 \mathrm{E \mhyphen
      prf}(k)  \enspace,
    \end{gather*}
    where $n$ is the max number of registered users, 
    $m$ is the max number of channels that a user is involved in,
    $p$ is the max number of times that a channel is updated
    and the ``E-'' terms correspond to the insecurity bounds of the
    primitives. 
  \end{theorem}

  \begin{proofsketch}
    The proof is done in 5 steps. In Lemma~\ref{lemma:dummy} we define a
    simulator $\simulator_{\mathrm{LN}}$ that internally simulates a full
    execution of $\Pi_{\mathrm{LN}}$ for each player, and a ``dummy''
    functionality that acts as a simple relay between \environment{} and
    $\simulator_{\mathrm{LN}}$. We argue that this version of the ideal world
    trivially produces the exact same messages for \environment{} as the real
    world.

    In each subsequent step, we incrementally move responsibilities from the
    simulator to the functionality. Each step defines a different functionality
    that handles some additional messages from \environment{} exactly like
    \fpaynet, until the last step (Lemma~\ref{lemma:close}) where we use
    \fpaynet{} itself. Correspondingly, the simulator of each step is adapted so
    that the new ideal execution is computationally indistinguishable from the
    previous one.

    Lemma~\ref{lemma:reg} lets $\mathcal{F}$ handle registration messages,
    along with the corruption messages from \simulator. In Lemma~\ref{lemma:open}
    the functionality additionally handles messages related to channel opening. It
    behaves like \fpaynet, but does not execute \texttt{checkClosed}().
    Lemma~\ref{lemma:pay} has the functionality handle all messages sent during
    channel updates. Lastly, Lemma~\ref{lemma:close} has the entire \fpaynet{} as
    its functionality, by incorporating the message for closing a channel,
    executing \texttt{checkClosed}() normally and handing the message that returns
    to \environment{} the receipts for newly opened, updated and closed
    channels. The last two steps introduce a probability of failure in case
    the various types of signatures used in  $\Pi_{\mathrm{LN}}$ are forged. We analyze these cases separately and argue that, if
    such forgeries do not happen, the emulation is perfect. Therefore we can
    calculate the concrete security bounds shown in the theorem.
  \end{proofsketch}
  For the formal proof, we refer the reader to Appendix~\ref{appendix:secproof}.
