\section{Security proof overview}
  The main target of the security proof is to prove
  Theorem~\ref{theorem:simulation}, which in other words states that there
  exists a simulator \simulator{} such that a real-world execution of the
  protocol $\Pi_{\mathrm{LN}}$ with a dummy adversary
  $\adversary_{\mathrm{d}}$~\cite{uc} and an ideal-world execution of the
  functionality \fpaynet{} with \simulator{} make any PPT environment
  \environment{} with binary output return on expectation the same value with
  high probability.

  To achieve this, we break the proof into 5 steps. In Lemma~\ref{lemma:dummy}
  we define a simulator $\simulator_{\mathrm{LN}}$ that internally simulates a
  full-blown execution of $\Pi_{\mathrm{LN}}$ for each player, and a ``dummy''
  functionality that acts as a simple relay between \environment{} and
  $\simulator_{\mathrm{LN}}$. We argue that this version of the ideal world
  trivially produces the exact same messages for \environment{} as the real
  world.

  In each subsequent step, we incrementally move responsibilities from the
  simulator to the functionality. Each step defines a different functionality
  that handles some additional messages from \environment{} exactly like
  \fpaynet, until the last step (Lemma~\ref{lemma:close}) where we use
  \fpaynet{} itself. Correspondingly, the simulator of each step is adapted so
  that the new ideal execution is computationally indistinguishable from the
  previous one.

  Lemma~\ref{lemma:reg} lets the functionality handle the registration messages,
  along with the corruption message from \simulator. In Lemma~\ref{lemma:open}
  the functionality additionally handles messages related to channel opening. It
  behaves like \fpaynet, but does not execute \texttt{checkClosed}().
  Lemma~\ref{lemma:pay} has the functionality handle all messages sent during
  channel updates. Lastly, Lemma~\ref{lemma:close} has the entire \fpaynet{} as
  its functionality, by incorporating the message for closing a channel,
  executing \texttt{checkClosed}() normally and handing the message that returns
  to \environment{} the receipts for newly opened, updated and closed channels.
  For the formal proof, we refer the reader to Appendix~\ref{appendix:secproof}.
