\section{PRF}
  A ``pseudorandom function''~\cite{katzlindell} $F$ is informally a function
  with two inputs: a secret seed and a bitstring. Given that the seed is
  randomly selected, no PPT algorithm can distinguish $F$ from a randomly
  selected function.

  In the current work a PRF is used in $\Pi_{\mathrm{LN}}$ to generate the
  randomness used for \textsc{KeyShareGen}(), which returns the per commitment
  keypairs $(s_{\alice, \mathrm{com}, n}, p_{\alice, \mathrm{com}, n})$
  (Fig.~\ref{alg:protocol:open:env}, line~\ref{alg:protocol:open:env:prf},
  Fig.~\ref{alg:protocol:open:openChannel},
  line~\ref{alg:protocol:open:peer:prf}, Fig.~\ref{alg:protocol:checkForNew},
  line~\ref{alg:protocol:checkForNew:prand},
  Fig.~\ref{alg:protocol:fundingLocked},
  line~\ref{alg:protocol:fundingLocked:prand} and
  Fig.~\ref{alg:protocol:pay:commitmentSigned},
  line~\ref{alg:protocol:pay:commitmentSigned:prf}).

  \begin{definition}
    Let $k \in \mathbb{N}$. Let $\mathsf{Func}_k$ the set of all functions
    mapping $k$-bitstrings to $k$-bitstrings. We say that function $F: \{0,
    1\}^k \times \{0, 1\}^k \rightarrow \{0, 1\}^k$ is pseudorandom if $\forall
    \text{ PPT } D$,
    \begin{equation*}
      |\underset{\substack{s \in \{0, 1\}^k \\ D\text{'s
      coins}}}{\Pr}[D^{F(s, \cdot)}(1^k) = 1] - \underset{\substack{f \in
      \mathsf{Func}_k \\ D\text{'s coins}}}{\Pr}[D^{f(\cdot)}(1^k) = 1]|
      \leq \negl(k) \enspace,
    \end{equation*}
    where $D$ is given oracle access to $F(s, \cdot)$ and $f(\cdot)$ in each of
    the probability expressions above respectively.
  \end{definition}
