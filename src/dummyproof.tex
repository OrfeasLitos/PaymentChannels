\begin{figure}[H]
  \begin{systembox}{$\fpaynet{}_{\mathrm{, dummy}}$}
    \begin{algorithmic}[1]
      \State Upon receiving any message $M$ from \alice:
      \Indent
        \If{$M$ is a valid \fpaynet{} message from a player}
          \State send ($M, \alice$) to \simulator
        \EndIf
      \EndIndent
      \Statex

      \State Upon receiving any message ($M, \alice$) from \simulator:
      \Indent
        \If{$M$ is a valid \fpaynet{} message from \simulator}
          \State send $M$ to \alice
        \EndIf
      \EndIndent
    \end{algorithmic}
  \end{systembox}
  \caption{}
  \label{alg:proof:fpaynet:dummy}
\end{figure}

\begin{figure}[H]
  \begin{simulatorbox}{$\simulator{}_{\mathrm{LN}}$}
    Expects the same messages as the protocol, but messages that the protocol
    expects to receive from \environment, the simulator expects to receive
    from $\fpaynet{}_{\mathrm{, dummy}}$ with the name of the player appended.
    The simulator internally executes one copy of the protocol per player.
    Upon receiving any message, the simulator runs the relevant code of the
    protocol copy tied to the appended player name. Mimicking the real-world
    case, if a protocol copy sends a message to another player, that message
    is passed to \adversary{} as if sent by the player and if \adversary{}
    allows the message to reach the receiver, then the simulator reacts by
    acting upon the message with the protocol copy corresponding to the
    recipient player. A message sent by a protocol copy to \environment{} will
    be routed by \simulator{} to $\fpaynet{}_{\mathrm{, dummy}}$ instead. To
    distinguish which player it comes from, \simulator{} also appends the
    player name to the message.
  \end{simulatorbox}
  \caption{}
  \label{alg:sim:ln}
\end{figure}

\begin{lemma}
  \label{lemma:dummy}
  $\textsc{Exec}^{\ledger}_{\Pi_{\mathrm{LN}}, \adversary_{\mathrm{d}},
  \environment} = \textsc{Exec}^{\fpaynet{}_{\mathrm{, dummy}},
  \ledger}_{\simulator_{\mathrm{LN}}, \environment}$
\end{lemma}

\begin{proof}
  Consider a message that \environment{} sends. In the real world, the
  protocol ITIs produce an output. In the ideal world, the message is given to
  $\simulator{}_{\mathrm{LN}}$ through $\fpaynet{}_{\mathrm{, dummy}}$. The
  former simulates the protocol ITIs of the real world (along with their coin
  flips) and so produces an output from the exact same distribution, which is
  given to \environment{} through $\fpaynet{}_{\mathrm{, dummy}}$. Thus the
  two outputs are indistinguishable.
\end{proof}
