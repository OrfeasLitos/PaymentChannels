\simulator{} expects the same messages as the protocol, but messages that the
protocol expects to receive from \environment, the simulator expects to receive
from \fpaynet{} with the name of the player appended. The simulator internally
executes one copy of the protocol per player. Upon receiving any message, the
simulator runs the relevant code of the protocol copy tied to the appended
player name. Mimicking the real-world case, if a protocol copy sends a message
to another player, that message is passed to \adversary{} as if sent by the
player and if \adversary{} allows the message to reach the receiver, then the
simulator reacts by acting upon the message with the protocol copy corresponding
to the recipient player. A message sent by a protocol copy to \environment{}
will be routed by \simulator{} to \fpaynet{} instead. To distinguish which
player it comes from, \simulator{} also appends the player name to the message.

\noindent \hrulefill \\
\fpaynet{}: \\
Upon receiving any message $M$ by \alice: send ($M, \alice$) to \simulator \\
Upon receiving any message ($M, \alice$) by \simulator: send $M$ to \alice
