\simulator{} expects the same messages as the protocol, but messages that the
protocol expects to receive from \environment, the simulator expects to receive
from \fpaynet{} with the name of the player appended. The simulator internally
executes one copy of the protocol per player. Upon receiving any message, the
simulator runs the relevant code of the protocol copy tied to the appended
player name. Mimicking the real-world case, if a protocol copy sends a message
to another player, that message is passed to \adversary{} as if sent by the
player and if \adversary{} allows the message to reach the receiver, then the
simulator reacts by acting upon the message with the protocol copy corresponding
to the recipient player. A message sent by a protocol copy to \environment{}
will be routed by \simulator{} to \fpaynet{} instead. To distinguish which
player it comes from, \simulator{} also appends the player name to the message.

\noindent \hrulefill \\
$\fpaynet{}_{\mathrm{, dummy}}$: \\
Upon receiving any message $M$ by \alice: send ($M, \alice$) to \simulator \\
Upon receiving any message ($M, \alice$) by \simulator: send $M$ to \alice

\begin{lemma}
  \label{lemma:dummyfunc}
  $\textsc{Exec}^{\ledger}_{\Pi_{\mathrm{LN}}, \adversary_{\mathrm{d}},
  \environment} = \textsc{Exec}^{\fpaynet{}_{\mathrm{, dummy}},
  \ledger}_{\simulator_{\mathrm{LN}}, \environment}$
\end{lemma}
\begin{proof}
  Consider a message that \environment{} sends. In the real world, the protocol
  ITIs produce an output. In the ideal world, the message is given to
  $\simulator{}_{\mathrm{LN}}$ through $\fpaynet{}_{\mathrm{, dummy}}$. The
  former simulates the protocol ITIs of the real world (along with their coin
  flips) and so produces an output from the exact same distribution, which is
  given to \environment{} through $\fpaynet{}_{\mathrm{, dummy}}$. Thus the two
  outputs are indistinguishable.
\end{proof}
