\section{Security proof overview}
\label{sec:ov-security-proof}
  \begin{theorem}[Lightning Payment Network Security]
    The protocol $\Pi_{\mathrm{LN}}$ realises $\fpaynet$ given a global
    functionality $\ledger$ and assuming the security of the underlying digital
    signature, identity-based signature, combined digital signature and PRF.
    Specifically,
    \label{theorem:simulation}
    \ifelseieee{
      \begin{gather*}
        \forall k \in \mathbb{N}, \text{ PPT } \environment, \\
        |\Pr[\textsc{Exec}^{\ledger}_{\Pi_{\mathrm{LN}}, \adversary_{\mathrm{d}},
        \environment} = 1] - \Pr[\textsc{Exec}^{\fpaynet, \ledger}_{\simulator,
        \environment} = 1]| \leq \\
        2nm \mathrm{E \mhyphen ds}(k) + 6np \mathrm{E \mhyphen ids}(k) + 2nmp
        \mathrm{E \mhyphen share}(k) + \\
        + 2nm \mathrm{E \mhyphen master}(k)+ 2 \mathrm{E \mhyphen prf}(k)
        \enspace,
      \end{gather*}
    }{
      \begin{gather*}
        \forall k \in \mathbb{N}, \text{ PPT } \environment,
        |\Pr[\textsc{Exec}^{\ledger}_{\Pi_{\mathrm{LN}}, \adversary_{\mathrm{d}},
        \environment} = 1] - \Pr[\textsc{Exec}^{\fpaynet, \ledger}_{\simulator,
        \environment} = 1]| \leq \\
        2nm  \mathrm{E \mhyphen ds}(k)  + 6np  \mathrm{E \mhyphen ids}(k) + 2nmp
        \mathrm{E \mhyphen share}(k) + 2nm \mathrm{E \mhyphen master}(k)+  2
        \mathrm{E \mhyphen prf}(k)  \enspace,
      \end{gather*}
    }
    where $n$ is the maximum number of registered users, $m$ is the maximum
    number of channels that a user is involved in, $p$ is the maximum number of
    times that a channel is updated and the ``E-'' terms correspond to the
    insecurity bounds of the primitives.
  \end{theorem}

  \begin{proofsketch}
    The proof is done in 5 steps \redden{of successive game replacement}. In
    \redden{the first lemma, (Lemma~\ref{lemma:dummy} in the appendix)} we
    define a simulator $\simulator_{\mathrm{LN}}$
    that internally simulates a full execution of $\Pi_{\mathrm{LN}}$ for each
    player, and a ``dummy'' functionality that acts as a simple relay between
    \environment{} and $\simulator_{\mathrm{LN}}$. We argue that this version of
    the ideal world trivially produces the exact same messages for
    \environment{} as the real world.

    In each subsequent step, we incrementally move responsibilities from the
    simulator to the functionality\redden{, while ensuring the change is
    transparent to both \environment{} and \adversary}. Each step defines a
    different functionality that handles some additional messages from
    \environment{} exactly like \fpaynet, until the last step
    (Lemma~\ref{lemma:close} \redden{in the appendix}) where we use \fpaynet{} itself. Correspondingly,
    the simulator of each step is adapted so that the new ideal execution is
    computationally indistinguishable from the previous one. \redden{For each
    step we exhaustively trace the differences from the previous step in order
    to prove that, given the same messages from \environment{} and \adversary{},
    the resulting responses remain unchanged.}

    \redden{The second step, (Lemma~\ref{lemma:reg} in the appendix)} lets $\mathcal{F}$ handle registration messages,
    along with the corruption messages from \simulator. In \redden{the third
    step,  (Lemma~\ref{lemma:open} in the appendix)}
    the functionality additionally handles messages related to channel opening. It
    behaves like \fpaynet, but does not execute \texttt{checkClosed}().
    \redden{The fourth step, (Lemma~\ref{lemma:pay} in the appendix)} has the functionality handle all messages sent during
    channel updates. Lastly, \redden{(Lemma~\ref{lemma:close} in the appendix)}
    the entire \fpaynet{} \redden{is used} as functionality, by incorporating the message for closing a channel,
    executing \texttt{checkClosed}() normally and handing the message that returns
    to \environment{} the receipts for newly opened, updated and closed
    channels. The last two steps introduce a probability of failure in case
    the various types of signatures used in  $\Pi_{\mathrm{LN}}$ are forged. We analyze these cases separately and argue that, if
    such forgeries do not happen, the emulation is perfect. Therefore we can
    calculate the concrete security bounds shown in the theorem.
  \end{proofsketch}
  \redden{As a concrete example of the proof approach, the second step (Lemma~\ref{lemma:reg} in the appendix) entails the following parts: First
  $\fpaynet{}_{\mathrm{, Reg}}$ is defined (Fig.~\ref{alg:proof:fpaynet:reg}),
  which is a functionality that behaves exactly like \fpaynet{} when receiving
  the messages \textsc{register}, \textsc{registerDone}, \textsc{toppedUp} and
  \textsc{corrupted}, but simply forwards all other messages along with the
  sender to \simulator. Then $\simulator{}_{\mathrm{LN - Reg}}$ is defined
  (Fig.~\ref{alg:sim:reg} in the appendix), which simulates all protocol instances, but in
  response to \textsc{register} messages from $\fpaynet{}_{\mathrm{, Reg}}$, it
  provides the public key of the key it just generated (as
  $\fpaynet{}_{\mathrm{, Reg}}$ expects). It also keeps track of corruptions and
  informs $\fpaynet{}_{\mathrm{, Reg}}$ thereof. Lastly, we argue that the
  functionality and simulator that were used in  the first step (Lemma~\ref{lemma:dummy} in the appendix) can be
  replaced by their newly defined counterparts without introducing any
  discernible difference to the transcript that any \environment{} sees. This is
  achieved by exhaustive enumeration of all possible messages and comparison of
  the behaviour of the ideal and the real world for each, to conclude that the
  change is transparent to \environment.}
  The formal proof can be found in Appendix~\ref{appendix:secproof}.
