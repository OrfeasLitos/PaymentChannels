\section{Future Work and Conclusion}
  In order to remain tractable, the current analysis omits some parts of
  the lightning specification. In particular, the specification defines how
  intermediaries of multi-hop payments can charge a fee for their service.
  Furthermore, the per-update secret generation is not done with a PRF according
  to the specification: an optimisation that reduces the storage overhead for
  the counterparty is used instead. The security of this optimisation however
  has not been yet formally inspected. Additionally, the specification
  provisions for a number of failure messages that help in keeping
  counterparties informed of issues with requested payments and in alleviating
  the problem of unneeded precautionary channel closures. Transactions
  that are added on-chain offer a fee to the blockchain miners (unrelated to the
  fee of the off-chain multi-hop payments). When closing a channel
  cooperatively, this fee is contributed by both counterparties, therefore the
  closing sequence of the specification includes an iterative negotiation of
  said fee where the two parties repeatedly propose a value based on their
  settings until they converge to a compromise or fail to agree. Lastly, most
  Bitcoin nodes do not relay transactions that include outputs with tiny amounts
  of coins, a.k.a ``dust'' outputs, to avoid bloating the blockchain. The
  lightning specification provides extensive instructions as to how to prune
  such outputs.

  All aforementioned parts of the protocol were not analysed so that the
  security of the core parts of the lightning protocol could be discussed
  without distractions. In order for the analysis to cover the entirety of the
  current version of the lightning specification however, the aforementioned
  features should be incorporated and their security should be proven. This
  expansion of the analysis is left as future work.

  In a different direction, big parts of our main security proof consist
  of an exhaustive enumeration of the possible messages that \environment{} and
  \adversary{} can send to the protocol, the simulator or the functionality and
  tracking how such messages would change the flow of the execution of the ideal
  and the real world. It is then argued that in all cases the messages that
  would be sent to \environment{} and \adversary{} are indistinguishable. These
  parts of the proof are good candidates for rewriting in the environment of an
  automated proof assistant~\cite{mccarthy} to instill additional certainty that
  all possible execution paths are indeed checked and do not contain subtle
  sources of distinguishability. %Developing an environment that aids in the  production and verification of indistinguishability proofs in the framework of  Universal Composability is left as future work.
Combining our results with the recent  mechanization of UC via Easycrypt~\cite{DBLP:conf/csfw/CanettiSV19} would be a natural and interesting direction for future work. 
  

  The lightning specification is not static, but it is continuously
  undergoing a number of improvements. The most noteworthy upcoming change is
  the introduction of Pointlocked TimeLocked Contracts (PTLCs). This mechanism
  replaces HTLCs and promises to combat the ``wormhole''
  attack~\cite{DBLP:conf/ndss/MalavoltaMSKM19}, while increasing privacy. Our
  work can be modified to cover the case of PTLCs with relative ease. It also
  provides a suitable framework for future work that aims to shed light on the
  exact privacy benefit that PTLCs offer as opposed to HTLCs.

  The present analysis constitutes the first comprehensive treatment in
  the Universal Composability framework of a deployed layer-2 protocol on top of
  a functional ledger. It can be extended and adapted to analyse other similar
  protocols that achieve different security goals or use another ledger as base
  layer.

