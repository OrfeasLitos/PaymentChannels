\section{Perun}
  Perun~\cite{perun} is a payment network designed for Turing-complete smart contract
  scripting languages. It has been implemented for Ethereum. Its main contribution is
  \textit{multistate channels} that allow the dynamic deployment of virtual contracts,
  known as \textit{nanocontracts}. Contracts of this type do not have to enter the
  blockchain if all parties are cooperative and only do so in case of a dispute.

  The paper describes specifically the use of such multistate channels for creating
  virtual payment channels between parties that do not have a basic payment channel
  between them, but both have basic multistate channels with an intermediary. Then the
  intermediary could substitute for the blockchain and thus a virtual payment channel on
  top of the two basic multistate channels can be created. The parties need the
  intermediary only for setting up the channel and to close it fast. If the intermediary
  refuses to close the channel, they can always fall back to the blockchain in order to
  close it.

  \subsection{Payment channels}
    A basic payment channel is a tuple
    \begin{equation*}
      \gamma = \left(\gamma\mbox{\texttt{.id}}, \gamma\mbox{\texttt{.Alice}},
      \gamma\mbox{\texttt{.Bob}}, \gamma\mbox{\texttt{.cash}},
      \gamma\mbox{\texttt{.ver-num}}, \gamma\mbox{\texttt{.sign}}\right)
    \end{equation*}
    Versions of this tuple are held by $Alice$ and $Bob$. $\gamma\mbox{\texttt{.id}}$ is a
    unique identifier for the channel, $\gamma\mbox{\texttt{.Alice}}$ and
    $\gamma\mbox{\texttt{.Bob}}$ are the end-users of $\gamma$ and
    $\gamma\mbox{\texttt{.cash}}$ is a function from the end-users to a real non-negative
    value that denotes the amount of cash the user has in the channel.
    $\gamma\mbox{\texttt{.ver-num}}$ is a number that is incremented with each channel
    update (so that the latest state of the channel is known in case of dispute) and
    $\gamma\mbox{\texttt{.sign}}$ is the singature of the other party on
    $\left(\gamma\mbox{\texttt{.id}}, \gamma\mbox{\texttt{.cash}},
    \gamma\mbox{\texttt{.ver-num}}\right)$.

    A payment channel has a corresponding
    $\mbox{\texttt{PaymentContract}}_{\gamma\mbox{\texttt{.id}}}$ on the ledger. End-users
    interact with the contract only to set up and close the channel, whereas updating the
    channel happens off-chain. The contract does not contain the fields
    $\gamma\mbox{\texttt{.ver-num}}$ and $\gamma\mbox{\texttt{.sign}}$; the two fields are
    kept only by the end-users.

    \subsubsection{Channel creation} \ \\

    The procedure of creating a channel is as follows:
    \begin{enumerate}
      \item $Alice$ creates a $\mbox{\texttt{PaymentContract}}\left(\gamma\right)$, pays
      it $\gamma\mbox{\texttt{.cash}}\left(\gamma\mbox{\texttt{.Alice}}\right)$ coins and
      broadcasts it on the ledger. The fields $\gamma\mbox{\texttt{.ver-num}}$ and
      $\gamma\mbox{\texttt{.sign}}$ are not included.
      \item The contract sends the message (\texttt{initialising}, $\gamma$) to both
      end-users ($\gamma\mbox{\texttt{.Alice}}$ and $\gamma\mbox{\texttt{.Bob}}$).
      \item $Bob$ calls the \texttt{confirm}() function of the contract and pays it the
      already specified amount of
      $\gamma\mbox{\texttt{.cash}}\left(\gamma\mbox{\texttt{.Bob}}\right)$ coins.
      \item The contract sends the message (\texttt{initialised}, $\gamma$) to both
      end-users.
      \item If $Alice$ does not receive (\texttt{initialised}, $\gamma$) after a
      predefined period $\Delta$ has passed from receiving (\texttt{initialising},
      $\gamma$), she calls the contract function \texttt{refund}() and gets her deposit
      back.
    \end{enumerate}

    Note that $Alice$ can get her money back if $Bob$ does not cooperate and $Bob$ only
    pays the contract after he verifies that $Alice$ has set up everything correctly. The
    contract code is public and thus end-users do not engage with it if it does not
    correspond to the expected code; no trust towards the contract is needed.

    \subsubsection{Channel update} \ \\

    Assume that the end-users want to update an existing channel balance from
    $\gamma\mbox{\texttt{.cash}}$ to $\mbox{\texttt{cash}}'$, where the total
    channel balance has remained unchanged:
    \begin{equation*}
      \gamma\mbox{\texttt{.cash}}\left(\gamma\mbox{\texttt{.Alice}}\right) +
      \gamma\mbox{\texttt{.cash}}\left(\gamma\mbox{\texttt{.Bob}}\right) =
      \mbox{\texttt{cash}}'\left(\gamma\mbox{\texttt{.Alice}}\right) +
      \mbox{\texttt{cash}}'\left(\gamma\mbox{\texttt{.Bob}}\right)
    \end{equation*}
    The procedure of updating to the new balance is as follows:
    \begin{enumerate}
      \item $Alice$ builds a new channel tuple $\gamma^{Alice}$ where
      \begin{itemize}
        \item the fields \texttt{id} and \texttt{users} are as in $\gamma$,
        \item $\gamma^{Alice}\mbox{\texttt{.cash}} = \mbox{\texttt{cash}}'$,
        \item $\gamma^{Alice}\mbox{\texttt{.ver-num}} = \gamma\mbox{\texttt{.ver-num}} +
        1$ and
        \item $\gamma^{Alice}\mbox{\texttt{.sign}}$ is $Alice$'s signature on \newline
        $\left(\gamma^{Alice}\mbox{\texttt{.id}}, \gamma^{Alice}\mbox{\texttt{.cash}},
        \gamma^{Alice}\mbox{\texttt{.ver-num}}\right)$.
      \end{itemize}
      \item $Alice$ sends $\gamma^{Alice}$ to $Bob$ and waits for his response.
      \item $Bob$ checks that all fields are as expected and replaces the old channel
      tuple, $\gamma$, with the newly tuple, $\gamma^{Alice}$. From his point of view, the
      payment has gone through.
      \item $Bob$ sends to $Alice$ the updated channel, $\gamma^{Bob}$, of which all
      fields are the same as $\gamma^{Alice}$ except for
      $\gamma^{Bob}\mbox{\texttt{.sign}}$, which is $Bob$'s signature on
      $\left(\gamma^{Bob}\mbox{\texttt{.id}}, \gamma^{Bob}\mbox{\texttt{.cash}},
      \gamma^{Bob}\mbox{\texttt{.ver-num}}\right)$.
      \item If $Alice$ receives the expected $\gamma^{Bob}$, she replaces the old channel
      tuple with $\gamma^{Bob}$. From her point of view, the payment has gone through.
    \end{enumerate}

    The above description holds symmetrically if $Bob$ initiates the channel update. If
    any player diverges from these steps, the other player can assume that the first has
    been corrupted and should close the channel immediately.

    Note that after the first update, the channel tuples held by the two players are not
    the same, their only difference being in the signature field. Strictly speaking, this
    means that the description of updating a channel above abuses the notation when it
    refers to $\gamma$ as the common previous channel state.

    Also note that the following scenario may arise: $Alice$ sends the updated version of
    the channel along with her signature, but $Bob$ does not reply. In this case, $Alice$
    wants to close the channel since $Bob$ is assumed to be corrupt, but the latest state
    of which she has $Bob$'s signature is one version earlier than $Bob$'s latest state.
    The only way $Alice$ can retrieve her funds is by broadcasting this older state. $Bob$
    can then broadcast his latest state, which supersedes $Alice$'s state. From the point
    of view of the blockchain, $Alice$ has tried to close the channel with an older state.

    Since there is a situation where the blockchain cannot say which player was corrupt,
    $Alice$ cannot be punished for broadcasting an older state of the channel by losing
    all her funds in the channel. She should be entitled to her share, as defined by the
    latest channel state that has been broadcast. Thus the punishment scheme of Lightning
    cannot be applied here.

    \subsubsection{Closing the channel} \ \\

    Finally, we present the procedure of closing a channel.

    \begin{enumerate}
      \item $Alice$ calls the function $\mbox{\texttt{close}}\left(\gamma^{Alice}\right)$
      of $\mbox{\texttt{PaymentContract}}_{\gamma\mbox{.\texttt{id}}}$.
      \item $\mbox{\texttt{PaymentContract}}_{\gamma\mbox{.\texttt{id}}}$ checks that
      $\gamma^{Alice}$ is correctly formed and holds the same total balance
      as the initial channel recorded in the contract. If so, it accepts $\gamma^{Alice}$
      as the channel state. Additionally, $Bob$ can call
      $\mbox{\texttt{close}}\left(\right)$ at any time and either $Alice$ or $Bob$ can
      call $\mbox{\texttt{finalize}}\left(\right)$ after time $\Delta$ has passed. If
      $\gamma^{Alice}$ is does not pass the checks, the contract ignores the call.
      \item If $Bob$ disagrees with the channel state published by $Alice$, he calls
      $\mbox{\texttt{close}}\left(\gamma^{Bob}\right)$ of
      $\mbox{\texttt{PaymentContract}}_{\gamma\mbox{.\texttt{id}}}$.
      \item Upon receiving a $\mbox{\texttt{close}}\left(\gamma^{Bob}\right)$ call from
      $Bob$, $\mbox{\texttt{PaymentContract}}_{\gamma\mbox{.\texttt{id}}}$ checks that
      $\gamma^{Bob}$ is correctly formed, holds the same total balance as the initial
      channel recorded in the contract and additionally has a higher version number than
      $\gamma^{Alice}$. If so, it accepts $\gamma^{Bob}$ as the channel state. Either
      $Alice$ or $Bob$ can still call $\mbox{\texttt{finalize}}\left(\right)$ after time
      $\Delta$ from $Alice$'s original $\mbox{\texttt{close}}\left(\right)$ call has
      passed. If $\gamma^{Bob}$ does not pass the checks, the contract ignores the call.
      \item After time $\Delta$ has passed, either end-user can call
      $\mbox{\texttt{finalize}}\left(\right)$ of
      $\mbox{\texttt{PaymentContract}}_{\gamma\mbox{\texttt{.id}}}$.
      \item Upon receiving a $\mbox{\texttt{finalize}}\left(\right)$ call from either
      end-user, the contract $\mbox{\texttt{PaymentContract}}_{\gamma\mbox{.\texttt{id}}}$
      checks that time $\Delta$ has passed since the original
      $\mbox{\texttt{close}}\left(\right)$. If so, it sends \texttt{closed} and
      $\gamma\mbox{\texttt{.cash}}\left(P\right)$ to each end-user $P$. If not, it ignores
      the $\mbox{\texttt{finalize}}\left(\right)$ call.
    \end{enumerate}

    The above closing sequence gives $Bob$ a window of duration at least $\Delta$ to
    dispute the closing channel state reported by $Alice$.

    Note that, in contrast to Lightning, there is no provision for cooperative closing of
    a channel, thus a delay of $\Delta$ must always be incurred between initiating a
    channel closure and getting access to the funds. The parameter $\Delta$ is decided by
    the parties when the channel is created and presents the same tradeoffs as the
    parameter $n$ of Lightning.

  \subsection{Multistate channels}
    A basic multistate channel is a tuple
    \begin{equation*}
      \gamma = \left(\gamma\mbox{.\texttt{id}}, \gamma\mbox{.\texttt{Alice}},
      \gamma\mbox{.\texttt{Bob}}, \gamma\mbox{.\texttt{cash}},
      \gamma\mbox{.\texttt{nspace}}\right) \enspace,
    \end{equation*}

    where $\gamma\mbox{.\texttt{id}}$, $\gamma\mbox{.\texttt{Alice}}$,
    $\gamma\mbox{.\texttt{Bob}}$ and $\gamma\mbox{.\texttt{cash}}$ are as in a payment
    channel and $\gamma\mbox{.\texttt{nspace}}$ is a set of nanocontracts, or
    \textit{nanocontracts space}.

    A nanocontract $\nu$ is a tuple
    \begin{equation*}
      \nu = \left(\nu\mbox{.\texttt{nid}}, \nu\mbox{.\texttt{blocked}},
      \nu\mbox{.\texttt{storage}}\right) \enspace,
    \end{equation*}

    where $\nu\mbox{.\texttt{nid}}$ is a globally unique identifier of the nanocontract,
    $\nu\mbox{\texttt{.blocked}}$ is a function from the end-users of the multistate
    channel to a real non-negative value that denotes the amount of cash the end-user has
    in the nanocontract and $\nu\mbox{\texttt{.storage}}$ contains the storage of the
    nanocontract.

    The multistate channel $\gamma$ has a corresponding contract
    $\mbox{\texttt{MSContract}}_{\gamma\mbox{.\texttt{id}}}$ on the ledger. The end-users
    have to interact with this contract upon channel creation, new nanocontract creation,
    channel closure and of course in case of dispute.
