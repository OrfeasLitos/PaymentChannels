
\section{Introduction}
\label{sec:intro}

Improving the latency of blockchain protocols, in the sense of the time it takes
for a transaction to be ``finalised'', as well as their throughput, in the sense
of the number of transactions they can handle per unit of time, are perhaps the
two most crucial open questions in the area of permissionless distributed
ledgers and remain fundamental barriers for their wider adoption in applications
that require large scale and reasonably expedient transaction processing,
cf.~\cite{scaling}. The Bitcoin blockchain protocol, introduced by
Nakamoto~\cite{bitcoin}, provides settlement with probability of error that
drops exponentially in the number of blocks $k$ that accumulate over a
transaction of interest. This has been informally argued in the original white
paper, and further formally demonstrated in~\cite{gkl}, from where it can be
inferred that the total delay in actual time for a transaction to settle is
linear in $k$ in the worst case. These results were subsequently generalised to
the setting of partial synchrony~\cite{PSS16} and variable
difficulty~\cite{DBLP:conf/crypto/GarayKL17}. Interestingly, this latency
``deficiency'' is intrinsic to the blockchain approach (see below), i.e.,
latency's dependency on $k$ is not a side-effect of the security analysis but
rather a characteristic of the underlying protocol and the threat model it
operates in.

Given the above state of affairs, one has to either change the underlying
settlement protocol or devise some other mechanism that, in conjunction with the
blockchain protocol, achieves high throughput and low latency. A number of works
proceeded with the first direction, e.g., hybrid
consensus~\cite{DBLP:conf/wdag/PassS17},
Algorand~\cite{DBLP:journals/corr/Micali16}. A downside of this approach is that
the resulting protocols fundamentally change the threat model within which
Bitcoin is supposed to operate, e.g., by reducing the threshold of corrupted
players, strengthening the underlying cryptographic assumptions or complicating
the setup assumption required (e.g., from a public to a private setup).

The alternative approach is to build an \emph{overlay} protocol that utilises
the blockchain protocol as a ``fall back'' layer and does not relax the threat
model in any way while it facilitates fast ``off-chain'' settlement under
certain additional assumptions.  We note that in light of the impossibility
result regarding protocol ``responsiveness'' from~\cite{DBLP:conf/wdag/PassS17}
that states that no protocol can provide settlement in time proportional to
actual network delay (i.e., fast settlement) and provide a security threshold
over $1/3$, we know that maintaining Bitcoin's threat model will require some
additional assumption for the overlay protocol to offer fast settlement.

The first instance of this approach and by far the most widely known and
utilised to date, came with the \emph{lightning
network}~\cite{lightning}\footnote{The specification available online is a more
descriptive reference for the inner workings of the protocol, see
\url{https://github.com/lightningnetwork/lightning-rfc/blob/master/02-peer-protocol.md}.
See also the raiden network that implements Lightning over Ethereum,
\url{https://raiden.network}.} that functions over the Bitcoin
blockchain and leverages the concept of a bilateral payment channel. The
latency for a transaction becomes linear to actual network delay and another
factor that equals the number of bilateral payment channel hops in the path that
connects the two end-points of the transaction. Implicated parties are
guaranteed that, if they wish so, \emph{eventually} the ledger will record
a ``gross'' settlement transaction that reflects the balance
resulting from all in-channel payments. Deviations from this guarantee are
cryptographically feasible but deincentivised: a malicious party trying
to commit to an outdated state will lose funds to a peer that provides evidence
of a subsequent state. Moreover, note that no record of a specific payment
transaction need ever appear on-chain thus the number of lightning transactions
that can be exchanged can reach the maximum capacity the network allows between
the parties, without being impeded by any restrictions of the underlying
blockchain protocol.

The lightning network has been very influential in the space and spun a number
of follow up research and implementations (see below for references). We note
that the lightning network is not the only option for building an overlay over a
blockchain. See e.g.,~\cite{DBLP:conf/eurocrypt/PassS18} for an alternative
approach focusing on reducing latency, where it is shown that if the
assumption is raised to a security threshold of $3/4$ plus the honesty of an
additional special player, it is possible to obtain optimal latency.
Nevertheless, this approach does not offer the throughput benefits that are
intrinsic to the lightning network.
%TODO: clarify above

Despite the importance of the lightning network for blockchain scalability there
is still no work so far providing a thorough formal security analysis. This is a
dire state of affairs given the fact that the protocol is actually currently
operational\footnote{For current deployment statistics see e.g.,
\url{https://1ml.com/statistics}.} and its complexity makes it difficult to
extract precise statements regarding the level of security it offers.

\subsection{Our Results}
We present the first, to our knowledge, complete
security analysis of the lightning network, which we carry out in the
universal composition (UC) setting. We model the payment overlay that the
lightning network provides as a local ideal functionality and we demonstrate how it
can be implemented in a hybrid world which assumes a global ledger
functionality. Our treatment is general and does not assume any specific
implementation for the underlying ledger functionality. The ``paynet''
functionality that we introduce abstracts all the salient security features
achieved by the lightning network. We subsequently describe the whole lightning
protocol in this setting and we prove that it realises our paynet functionality
under standard cryptographic assumptions; the security guarantees of the
functionality reflect specific environmental conditions regarding the
availability of the honest parties to poll the status of the network. In more
details our results are as follows.

\begin{enumerate}
\item We present the $\fpaynet{}$ functionality which abstracts the syntax and
security properties provided by the lightning network. We describe our
$\fpaynet{}$ assuming a global ledger functionality $\ledger$ as defined
in~\cite{BMTZ17}, and further refined in~\cite{genesis}, which we know that is
realised by the Bitcoin blockchain. Our approach not only captures lightning,
but it is also general as it can be applied to any payment network by finely
tuning the following parts of the functionality: the exact channel
opening message sequence, the details of the on-chain checks performed by
\fpaynet, the negligence time bounds and the penalty in case of a malicious
closure being caught. Using $\fpaynet{}$, parties can open and close channels,
forward payments along channel paths in the network as well as poll its status.
Importantly, the functionality keeps track of all the off-chain and on-chain
balances and ensures that when a channel closes, the on-chain balances
are in line with the off-chain ones. In order to handle
realistic adversarial deviations, $\fpaynet{}$ allows the
adversary to choose one of the following outcomes for each multi-hop
payment: (i) let it go through as requested, (ii) charge it to an adversarial
party along the path, (iii) charge it to a \emph{negligent} honest party along
the path. This last outcome is a crucial security characteristic of the
lightning network: honest parties are required to poll the
functionality with a frequency that corresponds to their level of
involvement in the network and the properties of the underlying ledger. If a
party does not poll often enough, $\fpaynet{}$ identifies it as
negligent and it may lose funds.

\item We identify for the first time the exact polling requirements that are
imposed by the lightning network to the honest participating parties, i.e.\ how
often parties have to check the state of the ledger functionality $\ledger$
(over which the lightning network is overlaid) and how to act depending on what
they see. These requirements are a function of the parameters of $\ledger$ and
ensure that honest parties do not lose funds. $\ledger$ provides explicit
security guarantees with respect to consistency and liveness which in turn
impact the guarantees provided by $\fpaynet{}$. The polling requirements for
each party are two-fold: (i) the
first type refers to monitoring for closures of channels of which the
party is a member, and is specified by the parameter ``$\mathrm{delay}$''
(chosen by the party), (ii) the second type refers to monitoring for
specific events related to receiving and relaying payments. In detail,
let \emph{Alice} be an intermediary of a multi-hop payment. When the payment
starts, she specifies two blockheights $h, h'$. Also, let $a$ be the upper bound
to the number of blocks that may be finalised in the ledger from the time a
certain transaction is emitted to the time it becomes finalised (i.e. it is
included in a block in the ``stable'' part of the ledger). \emph{Alice}
should then poll twice while her local view of the chain advances from
blockheight $h$ to blockheight $h' - a$. Moreover, the two pollings should be
separated by a time window so that the chain can grow by at least $a$
blocks.

\item We provide a complete pseudocode description of the lightning network
protocol $\Pi_{\mathrm{LN}}$ and prove that it realises $\fpaynet{}$ in
the Random Oracle model. We identify a number of underlying cryptographic
primitives that have been used in the lightning network in
a non-black-box fashion and without reference. Interestingly, while
most of these primitives are quite standard (a PRF, a Digital Signature
scheme and an Identity Based Signature scheme), there is also one that is less standard  and requires a new definition. The \emph{combined digital signature} -- as we
will call it -- is a special case of an asymmetric two-party digital
signature primitive (e.g., see~\cite{DBLP:conf/ndss/NicolosiKDM03} and
references therein) with the following characteristic: One of
the two parties, called the shareholder, generates and stores a share of the signing
key. The public key of the combined signature can be
determined non-interactively from public information produced by both
parties. Issuing signatures requires the availability of the share,
which is verifiable given the public information provided by the shareholder. We
formalise the combined digital signature primitive and show that the
construction lying within the lightning specification realises
it under standard cryptographic assumptions. In summary, the realisation of
$\fpaynet{}$ is achieved assuming the security of the underlying
primitives, which in turn can be based on EC-DLOG and the Random Oracle
model.

\item We prove that a more idealized ledger functionality, i.e. a ledger with
instant finality, is unrealisable even assuming a synchronous multicast network.
This result supports our decision to use the more realistic ledger functionality
of~\cite{BMTZ17}, since it establishes that if our analysis was based on such a
perfect ledger, it would not be relevant for any real world deployment of a
payment network since such software would -- necessarily -- depend on a
non-perfect ledger. This choice also distinguishes our work compared to previous
attempts~\cite{DBLP:conf/ccs/DziembowskiFH18,Malavolta:2017:CPP:3133956.3134096,sprites,perun}
to formalize payment networks, as well as highlights the considerable
latency improvement that the  protocol offers in comparison to directly
using the ledger.
\end{enumerate}

\subsection{Related Work}
A first suggestion for building a
unidirectional payment channel appeared in~\cite{spilman}. Bidirectional payment
channels were developed in~\cite{DBLP:conf/sss/DeckerW15} and of course as part
of the lightning network~\cite{lightning}. Subsequent work on the topic dealt
with either improving payment networks by utilising more expressive blockchains
such as Ethereum~\cite{perun}, hardware assumptions, see
e.g.,~\cite{DBLP:conf/systor/LindNEKPS18}, or extending its functionality beyond
payments, to smart contracts,~\cite{sprites} or finally enhancing their privacy,
see
e.g.,~\cite{Malavolta:2017:CPP:3133956.3134096,DBLP:conf/ccs/0001M17,DBLP:conf/ndss/HeilmanABSG17}.
Additional work looked into developing supporting protocols for the payment
networks such as rebalancing~\cite{DBLP:conf/ccs/KhalilG17} or finding routes in
a decentralised fashion~\cite{flare,spider}. With respect to idealising the
payment network functionality in the UC setting, a number of previous
papers~\cite{DBLP:conf/ccs/DziembowskiFH18,Malavolta:2017:CPP:3133956.3134096,sprites,perun}
presented ideal functionalities abstracting the concept, but they did
not prove that the lightning network realises them. The main advantage of our
approach however here is that, for the first time, we present a payment network
functionality that interoperates with a global ledger functionality for which we
know, in light of the results of~\cite{BMTZ17}, that is realisable by the
Bitcoin blockchain and hence also reflects the actual parameters that can be
enforced by the implementation and the exact participation conditions needed for
security. In contrast, previous
work~\cite{DBLP:conf/ccs/DziembowskiFH18,Malavolta:2017:CPP:3133956.3134096,perun}
utilized ``too idealised'' ledger functionalities for their analysis which offer
instant finality; as we prove in Theorem~\ref{theorem:perfectledger}, a
representative variant of these functionalities
(Fig.~\ref{fig:perfectledger:func}) is unrealisable even under strong network
assumptions (cf. Section~\ref{sec:perfect-ledger}). It is worth noting
here that, were such a ledger realizable, layer-2 payment networks would not be
as useful in practice because one of their two main motivations is the high
latency of real blockchains. On the other hand, in~\cite{sprites} the ledger is
not explicitly specified as a functionality, but only informally described.
Several smart contracts are formally defined instead as UC ITMs, which are the
entities with which protocols ultimately interact. The execution model of these
contracts and their interaction with the blockchain is explained in an intuitive
way, but a complete formalization of the ledger is missing. Lastly, the
ledger used in~\cite{DBLP:conf/ccs/DziembowskiFH18} cannot be used directly by
protocol parties, only accessed via higher-level functionalities. This
limitation is imposed because otherwise any party could arbitrarily change the
balances of other parties, given the definition of the functionality. This
ledger is therefore a useful abstraction for higher-level protocols, but not
amenable to direct usage, let alone concrete realisation.

\subsection{Organisation}
In Section~\ref{sec:preliminaries} we present
preliminaries for the model we employ and the relevant cryptographic primitives.
In Section~\ref{sec:ov-ln} we present an overview of the lightning network,
accompanied by figures of the relevant transactions. Our payment network
functionality is given an overview description in Section~\ref{sec:ov-paynet}.
Our abstraction of the core lightning protocol is provided in
Section~\ref{sec:ov-protocol}. We give more details about the combined digital
signature primitive in Section~\ref{sec:ov-combined-ds}. In
Section~\ref{sec:ov-security-proof} we provide an overview of the security
proof of the main simulation theorem. Finally, in
Section~\ref{sec:perfect-ledger} we formalise our claim that a ledger
functionality with instant finality is unrealisable. \redden{We refer the reader
to the appendix of the full version~\cite{full-version} for the formal
definition of the paynet functionality \fpaynet{} and the protocol
$\Pi_{\mathrm{LN}}$, along with the proof that the latter UC-realises the
former. Our claim on the unrealisability of a perfect ledger is also proven
there.}
