
\section{Introduction}
\label{sec:intro}

Improving the latency of blockchain protocols, 
in the sense of the time it takes for a transaction to be ``finalised'',  
as well as their throughput, in the sense of the number of transactions they can handle per unit of time, 
are perhaps the two most crucial 
open questions in the area of permissionless distributed ledgers
and remain fundamental barriers for their wider adoption in applications that  require large scale  and reasonably expedient transaction processing, cf.~\cite{scaling}.
The Bitcoin blockchain protocol, introduced by Nakamoto \cite{bitcoin}, 
provides settlement with probability of error that drops exponentially
in the number of blocks $k$ that accumulate over a transaction of interest. 
This has been informally argued in the original white paper, 
and further formally demonstrated in \cite{gkl}, 
from where it can be inferred that the total delay in actual time 
for a transaction to settle
is linear in $k$ in the worst case. These results were subsequently
generalised to the setting of partial synchrony \cite{PSS16} and 
variable difficulty \cite{DBLP:conf/crypto/GarayKL17}. 
Interestingly, this latency ``deficiency'' is 
intrinsic to the blockchain approach (see below), i.e., 
latency's  dependency on $k$ is not a side-effect of the security analysis
but rather a characteristic of the underlying protocol and the threat model it
operates in. 

Given the above state of affairs, one has to either
change the underlying settlement protocol or devise some other mechanism that, 
in conjunction with the blockchain protocol, achieves high throughput and low
latency.
A number of works proceeded with the first direction, e.g., hybrid consensus
\cite{DBLP:conf/wdag/PassS17} or Algorand \cite{DBLP:journals/corr/Micali16}. A downside of this approach is that the resulting protocols fundamentally change the threat model within which Bitcoin is supposed to operate, e.g., 
by changing the threshold required for security or the underlying cryptographic assumptions and setup that is needed. 
The additional side-effect of such solutions is that they are fundamentally incompatible with the Bitcoin blockchain, which is arguably  the currently most successful deployed instance  of the  blockchain protocol. 

The alternative approach is to build an {\em overlay} protocol that utilises the blockchain protocol as a ``fall back'' layer, while facilitating ``off-chain'' settlement under certain additional assumptions. We note that in light of the impossibility result regarding protocol ``responsiveness''  from \cite{DBLP:conf/wdag/PassS17} that states that no protocol can provide settlement in time proportional to actual network delay and provide a security threshold over $1/3$, we know that some additional assumption would be required for the overlay protocol to work.  

The first instance of this approach and by far the most widely known and
utilised so far, 
came with the {\em lightning network } 
\cite{lightning}\footnote{The specification available online is a more descriptive reference for the inner workings of the protocol, see \url{https://github.com/lightningnetwork/lightning-rfc/blob/master/02-peer-protocol.md}. See also the raiden network that implements Lightning over Ethereum, \url{https://raiden.network}.} that provided an overlay mechanism over the Bitcoin blockchain that introduces and takes advantage of the concept of a bilateral payment channel. The latency for a transaction becomes linear to actual network delay and another factor that equals the number of bilateral payment channel hops in the path that  connects the two end-points of the transaction. If a payment  transaction is confirmed by the parties implicated in a payment then, should the parties wish it,  it is guaranteed that {\em eventually} the ledger will record ``gross'' settlement transactions between the parties in the path of the payment transaction that are consistent with it. Deviations from this guarantee are cryptographically feasible but deincentivised via on-chain penalties.   
Moreover, note that no record of a  specific payment transaction need ever appear on-chain 
thus the number of lightning transactions that can be  exchanged can reach the maximum capacity the network allows between the parties, without being impeded by any  restrictions of the underlying blockchain protocol. 

The lightning network has been very influential in the space and spun a number of follow up research and implementations (see below for references).
We note that the lightning network is not the only option for building an overlay over a blockchain, see e.g.,  \cite{DBLP:conf/eurocrypt/PassS18} for an alternative approach where it is shown that if the assumption is raised to a security threshold of $3/4$ plus the honesty of an additional special player it is possible to obtain optimal latency. Nevertheless, the lightning network is currently the only option that readily interoperates with the Bitcoin blockchain. 

Despite the importance of the lightning network for low latency payments over Bitcoin there is still no work so far providing a thorough formal security analysis. 
This is a dire state of affairs given the fact that the protocol is actually currently operational\footnote{For current deployment statistics see e.g.,  \url{https://1ml.com/statistics}.}
and its complexity makes it difficult to extract precise statements regarding the level of security it offers. 

\noindent {\bf Our Results.}
We present the first, to our knowledge, complete  security analysis of the lightning network in the universal composition (UC) setting. 
We model the payment overlay that the lightning network provides as an ideal functionality and we demonstrate how it can be implemented in a hybrid world which assumes a global ledger functionality. Our treatment is general and does not assume any specific implementation for the underlying ledger functionality. The ``paynet''  functionality that we introduce abstracts all the salient security features achieved by the lightning network. We subsequently describe the whole lightning protocol in this setting and we prove that it realises our paynet functionality under standard cryptographic assumptions; the security guarantees of the functionality reflect specific environmental conditions regarding the availability of the honest parties to poll the status of the network. In more details our results are as follows. 

\begin{enumerate}
\item We present the $\fpaynet{}$ functionality which abstracts the syntax and
security properties that are provided by the lightning network. 
Using  $\fpaynet{}$, parties can open and close channels, forward
payments along channel paths in the network as well as poll its status. 
Importantly, the functionality keeps track of all the off-chain and on-chain
balances
of the  parties registered  and ensures that when a channel closes
the on-chain balances are in line with the off-chain balances.
In order to handle adversarial deviations in multi-hop payments, 
the functionality
permits the adversary to determine the outcome of each payment
by choosing either one of the following options (i) let it go through
as requested, (ii) charge it to an adversarial party along the path, 
(iii) charge it to {\em negligent } honest party along the path. 
This last outcome is a crucial  characteristic of the security
that is provided by the lightning network to its participants: 
honest parties are expected to poll the functionality at a certain 
specific rhythm that corresponds to their level of involvement in the network
and the properties of the underlying ledger. If a party misses that requirement
it is identified as negligent by the functionality and may lose funds. 

\item We identify for the first time 
the exact polling requirements that are imposed by the lightning
network to the honest parties that are participating so that they do not lose 
funds as a function of the underlying parameters of the ledger functionality
over which the lightning network is overlaid. 
We describe our  $\fpaynet{}$ given the global ledger functionality defined in 
\cite{BMTZ17}, and further refined in \cite{genesis}, for which we know already that is realised by the Bitcoin blockchain. The functionality provides explicit security guarantees with respect to consistency and liveness which in turn impact 
the guarantees provided by $\fpaynet{}$. The polling requirements for each party 
are  two-fold: (i) the first type of polling refers to monitoring for closures of channels that the party is one of the end-points and is specified by the 
user chosen parameter $\mathrm{delay}$, 
(ii) the second type of polling refers to monitoring for specific 
events related to receiving and relaying payments; in particular for each payment that a party acts as an intermediary,  polling should happen twice within the window of time when the chain in the view of the party advances  from blockheight  $h$
to blockheight $h'-a$, where $h,h'$ are two blockheight parameters specified  
in the particular payment path, and $a$ is a (derived) ledger parameter 
which is the upper bound to 
the number of blocks  that may be finalised in the ledger 
from the time a certain transaction is emitted to the time it becomes finalised
(i.e. it is included in a block in the ``stable''  part of the ledger). 
Moreover, the two pollings should be separated by a time window that allows
the chain to grow by at least $a$ blocks. 

\item We provide a complete pseudocode description  of the lightning network protocol $\Pi_{\mathrm{LN}}$
and we prove that indeed it 
realises  $\fpaynet{}$ under a specific set of cryptographic 
assumptions. In order to express $\Pi_{\mathrm{LN}}$ in a way that is succinct we identify a number of underlying cryptographic primitives that have been used in the specification of the lightning network in a non-black-box fashion and without reference.  Interestingly, while some of these cryptographic primitives are standard (they include a PRF, a Digital Signature scheme and an Identity Based Signature scheme) there is one additional primitive that is slightly less standard and we call  {\em combined digital signature}. A combined digital signature is a special case of an asymmetric  two-party digital signature primitive (e.g., see \cite{DBLP:conf/ndss/NicolosiKDM03} and references therein) with the characteristic that one of the two parties, the shareholder, generates and stores a share of the signing key, while the public-key of the combined signature is determined non-interactively based on public-key information produced by both two parties.  Issuing signatures requires the availability of the share which is verifiable given the public information provided by the shareholder.   We formalise the combined digital signature primitive and show that the construction lying within the specification of lightning is realising it under standard cryptographic assumptions. In summary, the realisation of  $\fpaynet{}$ is achieved assuming the security of the underlying primitives (which in turn can be based on EC-DLOG and the Random Oracle model). 
\end{enumerate}

\noindent {\bf Related Work.} A first suggestion for building 
a unidirectional payment channel
appeared in \cite{spilman}. Bidirectional payment channels were developed in 
\cite{DBLP:conf/sss/DeckerW15} and, of course as part of the lightning network  \cite{lightning}. Subsequent work on the topic 
dealt  with either improving payment networks  by utilising more 
expressive blockchains such as Ethereum \cite{perun},  hardward assumptions, see e.g., \cite{DBLP:conf/systor/LindNEKPS18}, 
or extending its functionality beyond payments, to smart contracts, 
\cite{sprites} or finally  enhancing their privacy, see e.g., 
\cite{Malavolta:2017:CPP:3133956.3134096,DBLP:conf/ccs/0001M17,DBLP:conf/ndss/HeilmanABSG17}. Additional work looked into developing supporting protocols
for the payment networks such as rebalancing 
\cite{DBLP:conf/ccs/KhalilG17} or finding routes in a decentralised fashion
\cite{flare}. 
With respect to idealising the payment network functionality in the UC setting, 
a number of previous papers \cite{Malavolta:2017:CPP:3133956.3134096,sprites,perun} presented
ideal functionalities abstracting the concept. The fundamental advantage of our approach however
here is  that, for the first time, we present a payment network functionality
that interoperates with a global ledger functionality for which we
know, in light of the results of \cite{BMTZ17}, 
that is realisable by the Bitcoin blockchain
and hence also reflects the actual parameters that can be enforced by the implementation 
(in contrast, previous work utilised ``too idealised'' ledger functionalities for which it is unknown whether they can be realised). 

\noindent {\bf Organisation.} In section~\ref{sec:preliminaries} we present  preliminaries for the model we employ  and the relevant cryptographic primitives. 
In Section~\ref{sec:ov-lightning} we present an overview of the lightning network. 
Our payment network functionality is given an overview description in Section~\ref{sec:ov-paynet}. Our abstraction of the core lightning protocol
is provided in Section~\ref{sec:ov-lightning}. We give more details about the combined digital signature primitive in Section~\ref{sec:ov-combined-ds}. Finally, in Section~\ref{sec:ov-security-proof}  we provide an overview of the security proof. 
In the appendix, first we give more details on the preliminary primitives, specifically in Sections~\ref{sec:ds},\ref{sec:ibs},\ref{sec:prf} we discuss digital signatures, identity based signatures and pseudorandom functions. Subsequently in Section~\ref{appendix:combinedsign} we go into more details about combined digital signatures. The transaction structure that is assumed to be provided by the underlying distributed ledger is discussed in Section~\ref{appendix:txstructure}. The paynet functionality is presented in detail in Section~\ref{appendix:payfunc}. The complete description of protocol $\Pi_{\mathrm{LN}}$ is presented in Section~\ref{appendix:protocol}. The ledger functionality is presented in Section~\ref{appendix:ledger} and finally, the security proof is presented in Section~\ref{appendix:secproof}. 

\ignore{%%%  
1. lightning network, objectives, when it was proposed, problem it solves, and current status. scalability as a general objectives. layered approach off/chain, on-chain. room example.

...both on the throughput (measured in transactions/sec) and the time that each
transaction takes to settle...

2. high level overview of the protocol. complex. security?

3. Our results.
a. Functionality.
b. Proof security.
c. Combined signatures.
The benefits of our approach.
What do we learn.
 * Large scale protocol. Modular approach.
 * Exact security parameterisation with a proof ...
 * Highlight the underlying cryptographic primitives that are relevant,
 PRFs  and combined signatures.

Add LN spec as footnote somewhere: 

``we probably are the first to formally specify the exact time bounds for
intermediaries''
}%%%%%%
