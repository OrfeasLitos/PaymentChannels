\begin{figure}[!htbp]
  \begin{simulatorbox}{$\simulator$}
    Like $\simulator{}_{\mathrm{LN - Reg - Open - Pay}}$. Differences:
    \begin{algorithmic}[1]
      \State Upon receiving (\textsc{forceCloseChannel}, \texttt{receipt},
      \textit{pchid}, \alice) from \fpaynet:
      \Indent
        \State simulate Fig.~\ref{alg:protocol:close:unilateral} receiving
        (\textsc{forceCloseChannel}, \texttt{receipt}, \textit{pchid}) with
        \alice's ITI
        \If{during the simulation above,
        line~\ref{alg:protocol:close:unilateral:report:payment} of
        Fig.~\ref{alg:protocol:close:unilateral} is simulated in \alice's ITI}
          \State send (\textsc{newPayments}, \texttt{payment}, \alice) to
          $\fpaynet{}_{\mathrm{, Pay}}$, where \texttt{payment} consists of the
          tuple just added to the simulated \texttt{paymentsToReport}, appended
          with the \textit{payid} of
          line~\ref{alg:protocol:close:unilateral:check:gotpaid}
          (Fig.~\ref{alg:protocol:close:unilateral},
          line~\ref{alg:protocol:close:unilateral:report:payment})
          \label{alg:sim:close:unilateral:report}
          \State upon receiving (\textsc{continue}) from $\fpaynet{}_{\mathrm{,
          Pay}}$, carry on with the simulation
        \EndIf
      \EndIndent
      \Statex

      \State Upon receiving (\textsc{closeChannel}, \texttt{receipt},
      \textit{pchid}, \alice) from \fpaynet:
      \Indent
        \State simulate the interaction between \alice{} and \bob{} in
        Fig.~\ref{alg:protocol:close:coop}, starting with \alice{} receiving
        (\textsc{closeChannel}, \texttt{receipt}, \textit{pchid}) with \alice's
        ITI. If \bob{} is honest, simulate his part as well; if not, let
        \adversary{} handle his part.
        \label{alg:sim:close:coop}
      \EndIndent
      \Statex

      \State every time \texttt{closedChannels} of \alice{} is updated with data
      from a \texttt{channel} (Fig.~\ref{alg:protocol:close:unilateral},
      line~\ref{alg:protocol:close:unilateral:report},
      Fig~\ref{alg:protocol:close:coop},
      line~\ref{alg:protocol:close:coop:report} and
      Fig.~\ref{alg:protocol:poll:closedch},
      line~\ref{alg:protocol:poll:report}), send (\textsc{closedChannel},
      \texttt{channel}, \alice) to \fpaynet{} and expect (\textsc{continue})
      from \fpaynet{} to resume simulation
      \label{alg:sim:close:report}
    \end{algorithmic}
  \end{simulatorbox}
  \caption{}
  \label{alg:sim:close}
\end{figure}

\begin{lemma}
  \label{lemma:close}
  \begin{gather*}
    \forall k \in \mathbb{N}, \text{ PPT } \environment, \\
    |\Pr[\textsc{Exec}^{\fpaynet{}_{\mathrm{, Pay}},
    \ledger}_{\simulator_{\mathrm{LN - Reg - Open - Pay}}, \environment} = 1] -
    \ifelseieee{\\ -}{}
    \Pr[\textsc{Exec}^{\fpaynet, \ledger}_{\simulator, \environment} = 1]| \leq
    \\
    nm \cdot \mathrm{E \mhyphen ds}(k) + 3np \cdot \mathrm{E \mhyphen ibs}(k) +
    \\
    nmp \cdot \mathrm{E \mhyphen share}(k) + \mathrm{E \mhyphen prf}(k) + nm
    \cdot \mathrm{E \mhyphen master}(k) \enspace.
  \end{gather*}
\end{lemma}

\begin{proof}
  Like in the previous proof, we here also assume that $\neg P \wedge \neg Q
  \wedge \neg R \wedge \neg S$ holds.

  When \environment{} sends (\textsc{forceCloseChannel}, \texttt{receipt},
  \textit{pchid}) to \alice, in the ideal world, if it is not the first such
  message to \alice{} for this channel, the message is ignored
  (Fig.~\ref{alg:fpaynet:close},
  line~\ref{alg:fpaynet:close:unilateral:noserve}). Similarly in the real world,
  if there has been another such message, \alice{} ignores it
  (Fig.~\ref{alg:protocol:close:unilateral},
  lines~\ref{alg:protocol:close:unilateral:remove}
  and~\ref{alg:protocol:close:unilateral:ensure}).

  In the case that it is indeed the first \textsc{forceCloseChannel} message for
  this channel, in the ideal world \fpaynet{} takes note that this close is
  pending (Fig.~\ref{alg:fpaynet:close},
  lines~\ref{alg:fpaynet:close:unilateral:retrieve}-\ref{alg:fpaynet:close:unilateral:mark})
  and stops serving more requests for this channel
  (line~\ref{alg:fpaynet:close:unilateral:noserve}), before asking \simulator{}
  to carry out channel closing (Fig.~\ref{alg:fpaynet:close},
  line~\ref{alg:fpaynet:close:unilateral:send}). \simulator{} then simulates the
  response to the original message from \environment{} with \alice's ITI
  (Fig.~\ref{alg:sim:close}). Observe that, since \fpaynet{} has ensured that
  this is the first request for closing this particular channel, the simulated
  check of line~\ref{alg:protocol:close:unilateral:ensure} in
  Fig.~\ref{alg:protocol:close:unilateral} always passes and the rest of
  Fig.~\ref{alg:protocol:close:unilateral} is executed. In the real world, the
  check also passes (since we are in the case where this is the first closing
  message) and Fig.~\ref{alg:protocol:close:unilateral} is executed by the real
  \alice{} in its entirety. Therefore, when \environment{} sends
  \textsc{forceCloseChannel}, no opportunity for distinguishability arises.

  When \environment{} sends (\textsc{closeChannel}, \texttt{receipt},
  \textit{pchid}) to \alice, in both the ideal and the real world the request
  will be further processed only if the identical checks of
  Fig.~\ref{alg:fpaynet:close}, line~\ref{alg:fpaynet:close:coop:ensure} and
  Fig.~\ref{alg:protocol:close:coop}, line~\ref{alg:protocol:close:coop:ensure}
  respectively pass. Further, in the ideal world, if \alice{} has already
  received a closing message for this channel (either \textsc{closeChannel} or
  \textsc{forceCloseChannel}), the message is ignored
  (Fig.~\ref{alg:fpaynet:close}, lines~\ref{alg:fpaynet:close:coop:noserve}
  and~\ref{alg:fpaynet:close:unilateral:noserve}). Likewise in the real world,
  if \alice{} has received another such message, she ignores it
  (Fig.~\ref{alg:protocol:close:coop},
  line~\ref{alg:protocol:close:coop:noserve} and
  Fig.~\ref{alg:protocol:close:unilateral},
  line~\ref{alg:protocol:close:unilateral:remove}).

  In the case that this is the first closing message for this channel, in the
  ideal world \fpaynet{} takes note that this close is pending
  (Fig.~\ref{alg:fpaynet:close},
  lines~\ref{alg:fpaynet:close:coop:retrieve}-\ref{alg:fpaynet:close:coop:mark})
  and stops rerving more \textsc{closeChannel} requests for this channel
  (Fig.~\ref{alg:fpaynet:close}, line~\ref{alg:fpaynet:close:coop:noserve}),
  before asking \simulator{} to carry out the cooperative channel closing
  negotiation (Fig.~\ref{alg:fpaynet:close},
  line~\ref{alg:fpaynet:close:coop:send}). \simulator{} then simulates the
  negotiation, respecting the fact that the counterparty may be corrupted
  (Fig.~\ref{alg:sim:close}, line~\ref{alg:sim:close:coop} and
  Fig.~\ref{alg:protocol:close:coop}). Observe that all checks of
  Fig.~\ref{alg:protocol:close:coop}
  (lines~\ref{alg:protocol:close:coop:ensure},~\ref{alg:protocol:close:coop:init:continue},~\ref{alg:protocol:close:coop:shutdown:continue},~\ref{alg:protocol:close:coop:signed:ensure:exists},~\ref{alg:protocol:close:coop:signed:ensure:marked}
  and~\ref{alg:protocol:close:coop:signed:ensure:sig}) will either succeed or
  fail in both the real and the ideal world in the same manner, since the same
  code is run in both cases (simulated in the ideal, directly in the real) and
  the inputs and state are, as we have seen, indistinguishable up to that point.
  Furthermore, for the same reason, if \bob{} is corrupted in the ideal world at
  the moment when \alice{} sends him \textsc{shutdown}, he would also be
  corrupted in the real world at the corresponding moment. Therefore, when
  \environment{} sends \textsc{closeChannel}, no opportunity for
  distinguishability arises.

  When \environment{} sends (\textsc{getNews}) to \alice, in the ideal world
  \fpaynet{} sends (\textsc{news}, \texttt{newChannels}(\alice),
  \texttt{closedChannels}(\alice), \texttt{updatesToReport}\linebreak(\alice))
  to \environment{} and empties these fields (Fig.~\ref{alg:fpaynet:daemon},
  lines~\ref{alg:fpaynet:getnews}-\ref{alg:fpaynet:getnews:send}). In the real
  world, \alice{} sends (\textsc{news}, \texttt{newChannels},
  \texttt{closedChannels}, \texttt{updatesToReport}) to \environment{} and
  empties these fields as well (Fig.~\ref{alg:protocol:poll},
  lines~\ref{alg:protocol:getnews}-\ref{alg:protocol:getnews:send}).
  \texttt{newChannels}(\alice) in the ideal world is populated in two cases:
  First, when \fpaynet{} receives (\textsc{channelOpened}) after \alice{} has
  previously received (\textsc{checkForNew})
  (Fig.~\ref{alg:fpaynet:open:negotiate},
  line~\ref{alg:fpaynet:channelOpened:report}). This happens when the simulated
  \alice{} ITI handles a \textsc{fundingLocked} message from \bob{}
  (Fig.~\ref{alg:sim:open:fundingLocked},
  line~\ref{alg:sim:open:channelOpened}). In the real world \alice{} would have
  modified her \texttt{new}-\linebreak\texttt{Channels} while handling \bob's
  \textsc{fundingLocked} (Fig.~\ref{alg:protocol:fundingLocked},
  line~\ref{alg:protocol:fundingLocked:report}), thus as far as this case is
  concerned, \texttt{newChannels} has the same contents in the real world as
  does \texttt{newChannels}(\alice) in the ideal. The other case when
  \texttt{newChannels}(\alice) is populated is when \fpaynet{} receives
  (\textsc{fundingLocked}) after \bob{} has previously received
  (\textsc{checkForNew}) (Fig.~\ref{alg:fpaynet:open:negotiate},
  line~\ref{alg:fpaynet:fundingLocked:report}). This (\textsc{fundingLocked})
  can only be sent by \simulator{} if \alice{} is honest and right before the
  receiving of (\textsc{fundingLocked}) is simulated with her ITI
  (Fig.~\ref{alg:sim:open:fundingLocked},
  lines~\ref{alg:sim:fundingLocked:sim:alice}-\ref{alg:sim:fundingLocked:sim:bob}).
  In the real world, \alice's \texttt{newChannels} would be populated upon
  handling the same (\textsc{fundingLocked}). Therefore the \texttt{newChannels}
  part of the message is identical in the real and the ideal world at every
  moment when \environment{} can send (\textsc{getNews}).

  Moving on to \texttt{closedChannels}(\alice), we observe that \fpaynet{} adds
  \texttt{channel} information when it receives (\textsc{closedChannel},
  \texttt{channel}, \alice) from \simulator{} (Fig.~\ref{alg:fpaynet:close},
  line~\ref{alg:fpaynet:closedChannel:report}), which in turn happens exactly
  when the simulated \alice{} ITI adds the \texttt{channel} to her
  \texttt{closedChannels} (Fig.~\ref{alg:sim:close},
  line~\ref{alg:sim:close:report}). Therefore the real and ideal
  \texttt{closedChannels} are always synchronized.

  Regarding \texttt{updatesToReport}, in the real world it is populated
  exclusively in line~\ref{alg:protocol:pay:raa:report:updates} of
  Fig.~\ref{alg:protocol:pay:revokeAndAck}. In the ideal world on the other
  hand, it is updated in line~\ref{alg:fpaynet:update:add} of
  Fig.~\ref{alg:fpaynet:pay}, which is triggered only by a (\textsc{newUpdate})
  message by \simulator. This message is sent only when
  line~\ref{alg:protocol:pay:raa:report:updates} of
  Fig.~\ref{alg:protocol:pay:revokeAndAck} is simulated by \simulator{}
  (Fig.~\ref{alg:sim:push}, line~\ref{alg:sim:push:report:updates}). In the real
  world, this happens only after receiving a valid (\textsc{revokeAndAck})
  message from the channel counterparty and after first having sent a
  corresponding (\textsc{commitmentSigned}) message
  (Fig.~\ref{alg:protocol:pay:revokeAndAck},
  line~\ref{alg:protocol:pay:revokeAndAck:ensure} and
  Fig.~\ref{alg:protocol:pay:commitmentSigned},
  lines~\ref{alg:protocol:pay:commit:nomark}
  and~\ref{alg:protocol:pay:commit:mark}), which happens only after receiving
  (\textsc{commit}) from \environment. In the ideal world a simulation of the
  same events can only happen in the exact same case, i.e. when \environment{}
  sends an identical (\textsc{commit}) to the same player. Indeed, \fpaynet{}
  simply forwards this message to \simulator{} (Fig.~\ref{alg:fpaynet:daemon},
  line~\ref{alg:fpaynet:daemon:commit}), who in turn simply simulates the
  response to the message with the simulated ITI that corresponds to the player
  that would receive the message in the real world (Fig.~\ref{alg:sim:push},
  line~\ref{alg:sim:push:commit}). We conclude that the \texttt{updatesToReport}
  sent to \environment{} in either the real or the ideal world are always
  identical.

  Lastly, in the ideal world, whenever (\textsc{read}) is sent to \ledger{} and
  a reply is received, the function \texttt{checkClosed}
  (Fig.\ref{alg:fpaynet:close:func}) is called with the reply of the \ledger{}
  as argument. This function does not generate new messages, but may cause the
  \fpaynet{} to halt. We will now prove that this never happens.

  \fpaynet{} halts in line~\ref{alg:fpaynet:close:func:dsforgery} of
  Fig.~\ref{alg:fpaynet:close:func} in case a channel is closed without using a
  commitment transaction. Similarly to event $E$ in the proof of
  Lemma~\ref{lemma:pay}, this event is a subset of $P$ and thus is impossible to
  happen given that we assume $\neg P$.

  \fpaynet{} halts in line~\ref{alg:fpaynet:close:func:malicious} of
  Fig.~\ref{alg:fpaynet:close:func} in case a malicious closure by the
  counterparty was successful, in spite of the fact that \alice{} polled in time
  to apply the punishment. A (\textsc{poll}) message to \alice{} within the
  prescribed time frame (line~\ref{alg:fpaynet:close:func:ifmalicious}) would
  cause \fpaynet{} to alert \simulator{} (Fig.~\ref{alg:fpaynet:poll},
  line~\ref{alg:fpaynet:poll:send}), who in turn would submit the punishment
  transaction in time to prevent the counterparty from spending the delayed
  payment (Fig.~\ref{alg:protocol:poll:closedch},
  lines~\ref{alg:protocol:poll:mal:tx}-\ref{alg:protocol:poll:mal:submit}).
  Therefore the only way for a malicious counterparty to spend the delayed
  output before \alice{} has the time to punish is by spending the punishment
  output themself. This however can never happen, since this event would be a
  subset of either $R$, if $\mathtt{remoteCom}_n$ (i.e. the counterparty closed
  the channel) is in $\Sigma_{\alice}$, or $Q$, if $\mathtt{localCom}_n$ is in
  $\Sigma_{\alice}$ (i.e. \alice{} closed the channel).

  \fpaynet{} halts in line~\ref{alg:fpaynet:close:func:idle} of
  Fig.~\ref{alg:fpaynet:close:func} in case \environment{} has asked for the
  channel to close, but too much time has passed since. This event cannot happen
  for two reasons. First, regarding elements in \texttt{pendingClose}(\alice),
  because \fpaynet{} forwards a (\textsc{forceCloseChannel}) message to
  \simulator{} (Fig.~\ref{alg:fpaynet:close},
  line~\ref{alg:fpaynet:close:unilateral:send}) for every element that it adds
  to \texttt{pendingClose} (Fig~\ref{alg:fpaynet:close},
  line~\ref{alg:fpaynet:close:unilateral:mark}) and this causes \simulator{} to
  submit the commitment transaction to \ledger{}
  (Fig.~\ref{alg:protocol:close:unilateral},
  line~\ref{alg:protocol:close:unilateral:submit}). This transaction is
  necessarily valid, because there is no other transaction that spends the
  funding transaction of the channel, according to the first check of
  line~\ref{alg:fpaynet:close:func:ifidle} of Fig.~\ref{alg:fpaynet:close:func}.
  \fpaynet{} halts in this case only if it is sure that the chain has grown by
  \tochain{} blocks, and thus if the closing transaction had been submitted when
  (\textsc{forceCloseChannel}) was received, it should have been necessarily
  included (Proposition~\ref{prop:tochain}). Second, elements added to
  \texttt{pendingClose}(\alice) as a reaction to a (\textsc{closeChannel})
  message are built with an infinite waiting time (Fig.~\ref{alg:fpaynet:close},
  line~\ref{alg:fpaynet:close:coop:mark}). Third, every element added to
  \texttt{closedChannels} (Fig.~\ref{alg:protocol:close:unilateral},
  line~\ref{alg:protocol:close:unilateral:report} and
  Fig.~\ref{alg:protocol:poll:closedch}, line~\ref{alg:protocol:poll:report})
  corresponds to a submission of a closing transaction for the same channel
  (Fig.~\ref{alg:protocol:close:unilateral},
  line~\ref{alg:protocol:close:unilateral:submit}), or to a channel for which
  the closing transaction is already in the ledger state
  (Fig.~\ref{alg:protocol:poll:closedch},
  line~\ref{alg:protocol:poll:closedch:loop}). In both cases, the transaction
  has been submitted at least \tochain{} blocks earlier, thus again by
  Proposition~\ref{prop:tochain} it is impossible for the transaction not to be
  in the ledger state. Therefore \fpaynet{} cannot halt in
  line~\ref{alg:fpaynet:close:func:idle} of Fig.~\ref{alg:fpaynet:close:func}.
  We deduce that, given $\neg P \wedge \neg Q \wedge \neg R$, the execution of
  \texttt{checkClosed} by \fpaynet{} does not contribute any increase to the
  probability of distinguishability. Put otherwise, given $\neg P \wedge \neg Q
  \wedge \neg R$, it is $\textsc{Exec}^{\fpaynet{}_{\mathrm{, Pay}},
  \ledger}_{\simulator_{\mathrm{LN - Reg - Open - Pay}}, \environment} =
  \textsc{Exec}^{\fpaynet, \ledger}_{\simulator, \environment}$.

  \fpaynet{} halts in line~\ref{alg:fpaynet:close:func:balance} of
  Fig.~\ref{alg:fpaynet:close:func} in case all \alice's channels are closed
  on-chain and either \alice's off-chain balance is not equal to zero, or if her
  on-chain balance is not the expected one, as reported by \simulator. This
  event can never happen for the following reasons. Firstly, as we have seen,
  \simulator{} reports all updates with a (\textsc{newUpdate}) message
  (Fig.~\ref{alg:sim:push}, line~\ref{alg:sim:push:report:updates}) and a
  (\textsc{resolvePays}) message; upon receiving the latter and given that it
  doesn't halt, \fpaynet{} updates \texttt{offChainBalance}(\alice) if she is
  the payer or payee of one of the resolved payments
  (Fig.~\ref{alg:fpaynet:resolvepay:honestpayer},
  lines~\ref{alg:fpaynet:resolvepay:corr:credit},~\ref{alg:fpaynet:resolvepay:debit}
  and~\ref{alg:fpaynet:resolvepay:credit}). Secondly, upon closure of each
  channel, \fpaynet{} would have halted if the closing balance were not the
  expected one (Fig.~\ref{alg:fpaynet:close:func},
  line~\ref{alg:fpaynet:close:func:ifmalicious}), an event that cannot happen as
  we have already proven. Lastly, upon each channel opening and closing,
  \fpaynet{} updates \texttt{offChainBalance}(\alice) and
  \texttt{onChainBalance}(\alice) to reflect the event
  (Fig.~\ref{alg:fpaynet:open:negotiate},
  lines~\ref{alg:fpaynet:channelOpened:offchain}
  and~\ref{alg:fpaynet:channelOpened:onchain} and
  Fig.~\ref{alg:fpaynet:close:func},
  lines~\ref{alg:fpaynet:close:func:alice:credit}
  or~\ref{alg:fpaynet:close:func:bob:credit} respectively). Therefore, it is
  impossible for \fpaynet{} to halt here.

  Similarly to the previous proof, if we allow for forgeries again, i.e. if we
  allow the event $P \vee Q \vee R \vee S$, we observe that $\Pr[P \vee Q \vee R
  \vee S] \leq nm \cdot \mathrm{E \mhyphen ds}(k) + 3np \cdot \mathrm{E \mhyphen
  ibs}(k) + nmp \cdot \mathrm{E \mhyphen share}(k) + \mathrm{E \mhyphen prf}(k)
  + nm \cdot \mathrm{E \mhyphen master}(k)$, where $n$ is the number of players,
  $m$ is the maximum channels a player can open and $p$ is the maximum number of
  updates a player can perform. We thus deduce that
  \begin{gather*}
    \forall k \in \mathbb{N}, \text{ PPT } \environment, \\
    |\Pr[\textsc{Exec}^{\fpaynet{}_{\mathrm{, Pay}},
    \ledger}_{\simulator_{\mathrm{LN - Reg - Open - Pay}}, \environment} = 1] -
    \ifelseieee{\\ -}{}
    \Pr[\textsc{Exec}^{\fpaynet, \ledger}_{\simulator, \environment} = 1]| \leq
    \\
    nm \cdot \mathrm{E \mhyphen ds}(k) + 3np \cdot \mathrm{E \mhyphen ibs}(k) +
    \\
    nmp \cdot \mathrm{E \mhyphen share}(k) + \mathrm{E \mhyphen prf}(k) + nm
    \cdot \mathrm{E \mhyphen master}(k) \enspace.
  \end{gather*}
\end{proof}
