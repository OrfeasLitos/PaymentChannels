\section{Identity Based Signatures primitive}
\label{sec:ibs}
  As we mentioned previously, LN uses a custom construction to derive three new
  keys on each update. We abstract this construction using a slight modification
  to the previously established Identity Based Signatures
  primitive~\cite{ibspaterson,ibsshamir}. Our version augments the scheme with
  explicit verification keys, which are generated together with the signing
  keys. Furthermore a new key derivation algorithm is introduced, which returns
  only the verification key of an identity, given its label. We furthermore
  prove that the custom construction used in LN realizes the primitive.

  The five algorithms used by an Identity Based Signatures scheme (with our
  modification) are:
  \begin{itemize}
    \item $(mpk, msk) \leftarrow \textsc{Setup}(1^k)$
    \item $(pk_l, sk_l) \leftarrow \textsc{KeyDer}(mpk, msk, l)$
    \item $pk_l \leftarrow \textsc{PubKeyDer}(mpk, l)$
    \item $\sigma \leftarrow \textsc{SignIBS}(m, sk_l)$
    \item $\{0, 1\} \leftarrow \textsc{VerifyIBS}(\sigma, m, pk_l)$
  \end{itemize}
  Observe that $mpk$ is not part of the input to \textsc{SignIBS} and
  \textsc{VerifyIBS}. In our case, this input is not needed. In fact, because
  of the underlying similarity of these two algorithms with their
  counterparts from standard Digital Signatures, such an input would rather
  complicate the exposition.

  We demand that the following holds for a scheme to have correctness:
  \begin{itemize}
    \item $\forall k \in \mathbb{N}, l \in \mathcal{L},$ \\
    $\Pr[(mpk, msk) \gets \textsc{Setup}(1^k),$ \\
    $(pk_1, sk_1) \gets \textsc{KeyDer}\left(mpk, msk, l\right),$ \\
    $pk_2 \gets \textsc{PubKeyDer}\left(mpk, l\right),$ \\
    $pk_1 = pk_2] = 1$

    \item $\forall k \in \mathbb{N}, m \in \mathcal{M},$ \\
    $\Pr[(mpk, msk) \gets \textsc{Setup}(1^k),$ \\
    $(pk, sk) \gets \textsc{KeyDer}\left(mpk, msk, l\right),$ \\
    $\textsc{VerifyIBS}(\textsc{SignIBS}(m, sk), m, pk) = 1] = 1$
  \end{itemize}

  \begin{figure}[!htbp]
    \begin{gamebox}{$\mathsf{IBS \mhyphen EUF \mhyphen
    CMA}^{\adversary}\left(1^k\right)$}
      \begin{algorithmic}[1]
        \State $(mpk, msk) \gets \textsc{Setup}(1^k)$
        \State $i, j \gets 0$
        \State $(\mathtt{aux}_0, \mathrm{response}) \gets
        \adversary(\textsc{init}, mpk)$
        \While{response can be parsed as $(m, l)$ or $l$}
          \If{response can be parsed as $(m, l)$}
            \State $i \gets i + 1$
            \State store $(m, l)$ as $(m, l)_i$
            \State $(pk, sk) \gets \textsc{KeyDer}(mpk, msk, l)$
            \State $\sigma \gets \textsc{SignIBS}(m, sk)$
            \State $(\mathtt{aux}_{i+j}, \mathrm{response}) \gets
            \adversary(\textsc{signature}, \mathtt{aux}_{i+j-1}, \sigma)$
          \Else \ \Comment{response can be parsed as $l$}
            \State $j \gets j + 1$
            \State store $l$ as $l_j$
            \State $(pk, sk) \gets \textsc{KeyDer}(mpk, msk, l)$
            \State $(\mathtt{aux}_{i+j}, \mathrm{response}) \gets
            \adversary(\textsc{keypair}, \mathtt{aux}_{i+j-1}, (pk, sk))$
          \EndIf
        \EndWhile
        \State parse response as $(m^*, l^*, \sigma^*)$
        \If{$(m^*, l^*) \notin \{(m, l)_1, \dots, (m, l)_i\} \wedge l^* \notin
        \{l_1, \dots, l_j\} \wedge \textsc{VerifyIBS}(\sigma^*, m^*,
        \textsc{PubKeyDer}(mpk, l^*)) = 1$}
          \State \Return 1
        \Else
          \State \Return 0
        \EndIf
      \end{algorithmic}
    \end{gamebox}
    \caption{}
    \label{game:ibs}
  \end{figure}
  \begin{definition}
    \label{def:ibs:secure}
    An Identity Based Signatures scheme is \emph{\textsf{IBS-EUF-CMA}-secure}
    if
    \begin{gather*}
      \forall k \in \mathbb{N}, \forall \adversary \in \mathtt{PPT},
      \ifelseieee{\\}{}
      \Pr\left[\mathsf{IBS \mhyphen EUF \mhyphen
      CMA}^{\adversary}\left(1^k\right) = 1\right] = \mathrm{negl}\left(k\right)
      \ifelseieee{\\}{}
      \text{or equivalently} \\
      \forall k \in \mathbb{N}, \mathrm{E \mhyphen ibs}(k) =
      \mathrm{negl}\left(k\right) \enspace, \\
      \text{where } \mathrm{E \mhyphen ibs}(k) = \underset{\adversary \in
      \mathtt{PPT}}{\sup}\{\Pr[\mathsf{IBS \mhyphen EUF \mhyphen
      CMA}^{\adversary}\left(1^k\right) = 1]\} \enspace.
    \end{gather*}
  \end{definition}

  \subsection{Construction}
    We here define the particular construction for Identity Based Signatures
    used in LN and prove its security.

    Parameters: hash function $\mathcal{H}$, group generator $G$
    \begin{algorithmic}[0]
      \State \textsc{Setup}($1^k$, rand):
      \Indent
        \State \Return ($G \cdot \mathrm{rand}$, rand)
      \EndIndent
    \end{algorithmic}

    \begin{algorithmic}[0]
      \State \textsc{KeyDer}($mpk, msk, l$):
      \Indent
        \State $pk \gets mpk + \mathcal{H}\left(l \concat mpk\right) \cdot G$
        \State $sk \gets msk + \mathcal{H}\left(l \concat mpk\right)$
        \State \Return $(pk, sk)$
      \EndIndent
    \end{algorithmic}

    \begin{algorithmic}[0]
      \State \textsc{PubKeyDer}($mpk, l$):
      \Indent
        \State \Return $mpk + \mathcal{H}\left(l \concat mpk\right) \cdot G$
      \EndIndent
    \end{algorithmic}

    \begin{algorithmic}[0]
      \State \textsc{SignIBS}($m, sk_l$):
      \Indent
        \State \Return \textsc{SignDS}($m, sk_l$)
      \EndIndent
    \end{algorithmic}

    \begin{algorithmic}[0]
      \State \textsc{VerifyIBS}($\sigma, m, pk_l$):
      \Indent
        \State \Return \textsc{VerifyDS}($\sigma, m, pk_l$)
      \EndIndent
    \end{algorithmic}

    \begin{lemma}
      \label{lemma:ibs}
      The construction above is \textsf{IBS-EUF-CMA}-secure in the Random Oracle
      model under the assumption that the underlying signature scheme is
      strongly \textsf{EUF-CMA}-secure and the range of the Random Oracle
      coincides with that of the underlying signature scheme signing keys.
    \end{lemma}

    \begin{proof}
      Let $k \in \mathbb{N}, \mathcal{B}$ PPT algorithm such that
      \begin{equation*}
        \Pr\left[\mathsf{IBS \mhyphen EUF \mhyphen
        CMA}^{\mathcal{B}}\left(1^k\right) = 1\right] = a >
        \mathrm{negl}\left(k\right) \enspace.
      \end{equation*}
      We construct a PPT distinguisher \adversary{}
      (Fig.~\ref{proof:ibs:distinguisher}) such that
      \begin{equation*}
        \Pr\left[\mathsf{EUF \mhyphen CMA}^{\adversary}\left(1^k\right) =
        1\right] > \mathrm{negl}\left(k\right)
      \end{equation*}
      that breaks the assumption, thus proving Lemma~\ref{lemma:ibs}.

      \begin{figure}[!htbp]
        \begin{algobox}{$\adversary\left(vk\right)$}
          \begin{algorithmic}[1]
            \State $k \overset{\$}{\gets} U\left[1, T\left(\mathcal{B}\right) +
            T\left(\adversary\right)\right]$
            \Comment{$T\left(M\right)$ is the maximum running time of $M$}
            \Indent
              \State Random Oracle: for every first-seen query $q$ from
              $\mathcal{B}$ set $\mathcal{H}\left(q\right)$ to a random value
              \State \Return $\mathcal{H}\left(q\right)$ to $\mathcal{B}$
            \EndIndent
            \State $\left(mpk, msk\right) \gets
            \textsc{Setup}\left(1^k\right)$
            \Indent
              \State Random Oracle: Let $q$ be the $k$th first-seen query from
              $\mathcal{B}$ or \adversary:
              \If{$q = \left(l \concat mpk\right)$}
                \State set $\mathcal{H}\left(l \concat mpk\right)$ to $\left(vk
                - mpk\right) \cdot G^{-1}$
              \Else
                \State set $\mathcal{H}\left(q\right)$ to a random value
              \EndIf
              \State \Return $\mathcal{H}\left(q\right)$ to $\mathcal{B}$ or
              \adversary
            \EndIndent
            \State $i \gets 0$
            \State $j \gets 0$
            \State $\left(\mathtt{aux}_0, \mathrm{response}\right) \gets
            \mathcal{B}\left(\textsc{init}, mpk\right)$
            \While{response can be parsed as $(m, l)$ or $l$}
              \If{response can be parsed as $(m, l)$}
                \State $i \gets i + 1$
                \State store $(m, l)$ as $(m, l)_i$
                \State $(pk, sk) \gets \textsc{KeyDer}(mpk, msk, l)$
                \State $\sigma \gets \textsc{SignIBS}(m, sk)$
                \State $\left(\mathtt{aux}_{i+j}, \mathrm{response}\right) \gets
                \mathcal{B}\left(\textsc{signature}, \mathtt{aux}_{i+j-1},
                \sigma\right)$
              \Else \ \Comment{response can be parsed as $l$}
                \State $j \gets j + 1$
                \State store $l$ as $l_j$
                \State $(pk, sk) \gets \textsc{KeyDer}(mpk, msk, l)$
                \State $\left(\mathtt{aux}_{i+j}, \mathrm{response}\right) \gets
                \mathcal{B}\left(\textsc{keypair}, \mathtt{aux}_{i+j-1}, (pk,
                sk)\right)$
              \EndIf
            \EndWhile
            \State parse response as $\left(m^*, l^*, \sigma^*\right)$
            \If{$vk = \textsc{PubKeyDer}(mpk, l^*) \wedge \mathcal{B}$ wins the
            \textsf{IBS-EUF-CMA} game} \Comment{\adversary{} won the
            \textsf{EUF-CMA} game}
            \label{proof:ibs:distinguisher:won}
              \State \Return $\left(m^*, \sigma^*\right)$
            \Else
              \State \Return \textsc{fail}
            \EndIf
          \end{algorithmic}
        \end{algobox}
        \caption{}
        \label{proof:ibs:distinguisher}
      \end{figure}

      Let $Y$ be the range of the random oracle. The modified random oracle used
      in Fig.~\ref{proof:ibs:distinguisher} is indistinguishable from the
      standard random oracle by PPT algorithms since the statistical distance of
      the standard random oracle from the modified one is at most
      $\frac{1}{2|Y|} < \mathrm{negl}\left(k\right)$ as they differ in at most
      one element.

      Let $E$ denote the event in which neither \textsc{KeyDer}($mpk, msk, l^*$)
      or \textsc{PubKeyDer}($mpk, l^*$) is invoked. In that case the value
      $\mathcal{H}\left(l \concat mpk\right)$ is decided after $\mathcal{B}$
      terminates (Fig.~\ref{proof:ibs:distinguisher},
      line~\ref{proof:ibs:distinguisher:won}) and thus
      \begin{equation}
        \begin{gathered}
          \Pr[vk \in \textsc{KeyDer}\left(mpk, msk, l^*\right) \vee \\
          vk = \textsc{PubKeyDer}\left(mpk, l^*\right) | E] <
          \mathrm{negl}\left(k\right) \Rightarrow \\
          \Pr[(vk \in \textsc{KeyDer}\left(mpk, msk, l^*\right) \vee \\
          vk = \textsc{PubKeyDer}\left(mpk, l^*\right)) \wedge E] <
          \mathrm{negl}\left(k\right) \Rightarrow \\
          \Pr\left[vk = \textsc{PubKeyDer}\left(mpk, l^*\right) \wedge E\right]
          < \mathrm{negl}\left(k\right) \enspace.
        \end{gathered}
        \label{proof:ibs:nocomb}
      \end{equation}
      It is
      \begin{gather*}
        \left(\mathcal{B} \text{ wins}\right) \rightarrow \left(vk =
        \textsc{PubKeyDer}\left(mpk, l^*\right)\right) \Rightarrow \\
        \Pr\left[\mathcal{B} \text{ wins}\right] \leq \Pr\left[vk =
        \textsc{PubKeyDer}\left(mpk, l^*\right)\right] \Rightarrow \\
        \Pr\left[\mathcal{B} \text{ wins} \wedge E\right] \leq
        \ifelseieee{\\ \leq}{}
        \Pr\left[vk = \textsc{PubKeyDer}\left(mpk, l^*\right) \wedge E\right]
        \overset{\left(\ref{proof:ibs:nocomb}\right)}{\Rightarrow} \\
        \Pr\left[\mathcal{B} \text{ wins} \wedge E\right] <
        \mathrm{negl}\left(k\right) \enspace.
      \end{gather*}

      But we know that $\Pr\left[\mathcal{B} \text{ wins}\right] =
      \Pr\left[\mathcal{B} \text{ wins} \wedge E\right] + \Pr\left[\mathcal{B}
      \text{ wins} \wedge \neg E\right]$ and $\Pr\left[\mathcal{B} \text{
      wins}\right] = a$ by the assumption, thus
      \begin{equation}
        \label{proof:ibs:nohash}
        \Pr\left[\mathcal{B} \text{ wins} \wedge \neg E\right] > a -
        \mathrm{negl}\left(k\right) \enspace.
      \end{equation}

      We now focus at the event $\neg E$. Let $F$ the event in which the call of
      to \textsc{KeyDer}($mpk, msk, l^*$) or \textsc{PubKeyDer}($mpk, l^*$)
      results in the $k$th
      invocation of the Random Oracle. Since $k$ is chosen uniformly at random
      and using Proposition~\ref{prop:distrib},
      $\Pr\left[F | \neg E\right] = \frac{1}{T\left(\mathcal{B}\right) +
      T\left(\adversary\right)}$. Observe that $\Pr\left[F | E\right] = 0
      \Rightarrow \Pr\left[F\right] = \Pr\left[F | \neg E\right] =
      \frac{1}{T\left(\mathcal{B}\right) + T\left(\adversary\right)}$.

      In the case where the event $\left(F \wedge \mathcal{B} \text{ wins}
      \wedge \neg E\right)$ holds, it is
      \ifelseieee{
        \begin{gather*}
          \textsc{PubKeyDer}\left(mpk, l^*\right) = \\
          mpk + \mathcal{H}\left(l^* \concat mpk\right) \cdot G = \\
          mpk + (vk - mpk) \cdot G^{-1} \cdot G = vk
        \end{gather*}
      }{
        \begin{equation*}
          \textsc{PubKeyDer}\left(mpk, l^*\right) = mpk + \mathcal{H}\left(l^*
          \concat mpk\right) \cdot G = mpk + (vk - mpk) \cdot G^{-1} \cdot G = vk
        \end{equation*}
      }
      Since $F$ holds, the $k$th invocation of the Random Oracle queried
      $\mathcal{H}\left(l^* \concat mpk\right)$.
      Therefore it is $\textsc{PubKeyDer}\left(mpk, l^*\right) = vk$. This means
      that the verification is successful: $\textsc{VerifyIBS}\left(\sigma^*,
      m^*, vk\right) = 1$. We conclude that, if $\left(F \wedge
      \mathcal{B} \text{ wins} \wedge \neg E\right)$, \adversary{} wins the
      \textsf{EUF-CMA} game. A final observation is that the probability that
      the events $\left(\mathcal{B} \text{ wins} \wedge \neg E\right)$ and $F$
      are almost independent, thus
      \begin{gather*}
        \Pr\left[F \wedge \mathcal{B} \text{ wins} \wedge \neg E\right] =
        \ifelseieee{\\ =}{}
        \Pr\left[F\right] \Pr\left[\mathcal{B} \text{ wins} \wedge \neg E\right]
        \pm \mathrm{negl}\left(k\right)
        \overset{\left(\ref{proof:ibs:nohash}\right)}{=} \\
        \frac{a - \mathrm{negl}\left(k\right)}{T\left(\mathcal{A}\right) +
        T\left(\mathcal{B}\right)} \pm \mathrm{negl}\left(k\right) >
        \mathrm{negl}\left(k\right)
      \end{gather*}
    \end{proof}
