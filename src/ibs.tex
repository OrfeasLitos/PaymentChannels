\section{Identity Based Signatures primitive}
  As we mentioned previously, LN uses a custom construction to derive three new
  keys on each update. We abstract this construction using a slight modification
  to the previously established Identity Based Signatures
  primitive~\cite{ibspaterson,ibsshamir}. Our version augments the scheme with
  explicit verification keys, which are generated together with the signing
  keys. Furthermore a new key derivation algorithm is introduced, which returns
  only the verification key of an identity, given its label. We furthermore
  prove that the custom construction used in LN realizes the primitive.
  \TODO{TODO last bit}

  The five algorithms used by an Identity Based Signatures scheme (with our
  modification) are:
  \begin{itemize}
    \item $(mpk, msk) \leftarrow \textsc{Setup}(1^k)$
    \item $(pk_l, sk_l) \leftarrow \textsc{KeyDer}(mpk, msk, l)$
    \item $pk_l \leftarrow \textsc{PubKeyDer}(mpk, l)$
    \item $\sigma \leftarrow \textsc{SignIBS}(m, sk_l)$
    \item $\{0, 1\} \leftarrow \textsc{VerifyIBS}(\sigma, m, pk_l)$
  \end{itemize}
  Observe that $mpk$ is not part of the input to \textsc{SignIBS} and
  \textsc{VerifyIBS}. In our case, this input is not needed. In fact, because
  of the underlying similarity of these two algorithms with their
  counterparts from standard Digital Signatures, such an input would rather
  complicate the exposition.

  We demand that the following holds for a scheme to have correctness:
  \begin{itemize}
    \item $\forall k \in \mathbb{N}, l \in \mathcal{L},$ \\
    $\Pr[(mpk, msk) \gets \textsc{Setup}(1^k),$ \\
    $(pk_1, sk_1) \gets \textsc{KeyDer}\left(mpk, msk, l\right),$ \\
    $pk_2 \gets \textsc{PubKeyDer}\left(mpk, l\right),$ \\
    $pk_1 = pk_2] = 1$

    \item $\forall k \in \mathbb{N}, m \in \mathcal{M},$ \\
    $\Pr[(mpk, msk) \gets \textsc{Setup}(1^k),$ \\
    $(pk, sk) \gets \textsc{KeyDer}\left(mpk, msk, l\right),$ \\
    $\textsc{VerifyIBS}(\textsc{SignIBS}(m, sk), m, pk) = 1] = 1$
  \end{itemize}

  \begin{figure}[!htbp]
    \begin{gamebox}{$\mathsf{IBS \mhyphen EUF \mhyphen
    CMA}^{\adversary}\left(1^k\right)$}
      \begin{algorithmic}[1]
        \State $(mpk, msk) \gets \textsc{Setup}(1^k)$
        \State $i, j \gets 0$
        \State $(\mathtt{aux}_0, \mathrm{response}) \gets
        \adversary(\textsc{init}, mpk)$
        \While{response can be parsed as $(m, l)$ or $l$}
          \If{response can be parsed as $(m, l)$}
            \State $i \gets i + 1$
            \State store $(m, l)$ as $(m, l)_i$
            \State $(pk, sk) \gets \textsc{KeyDer}(mpk, msk, l)$
            \State $\sigma \gets \textsc{SignIBS}(m, sk)$
            \State $(\mathtt{aux}_{i+j}, \mathrm{response}) \gets
            \adversary(\textsc{signature}, \mathtt{aux}_{i+j-1}, \sigma)$
          \Else \ \Comment{response can be parsed as $l$}
            \State $j \gets j + 1$
            \State store $l$ as $l_j$
            \State $(pk, sk) \gets \textsc{KeyDer}(mpk, msk, l)$
            \State $(\mathtt{aux}_{i+j}, \mathrm{response}) \gets
            \adversary(\textsc{keypair}, \mathtt{aux}_{i+j-1}, (pk, sk))$
          \EndIf
        \EndWhile
        \State parse response as $(m^*, l^*, \sigma^*)$
        \If{$(m^*, l^*) \notin \{(m, l)_1, \dots, (m, l)_i\} \wedge l^* \notin
        \{l_1, \dots, l_j\} \wedge \textsc{VerifyIBS}(\sigma^*, m^*,
        \textsc{PubKeyDer}(mpk, l^*)) = 1$}
          \State \Return 1
        \Else
          \State \Return 0
        \EndIf
      \end{algorithmic}
    \end{gamebox}
    \caption{}
    \label{game:ibs}
  \end{figure}
  \begin{definition}
    \label{def:ibs:secure}
    An Identity Based Signatures scheme is \emph{\textsf{IBS-EUF-CMA}-secure}
    if
    \begin{equation*}
      \forall k \in \mathbb{N}, \forall \adversary \in \mathtt{PPT},
      \Pr\left[\mathsf{IBS \mhyphen EUF \mhyphen
      CMA}^{\adversary}\left(1^k\right) = 1\right] =
      \mathit{negl}\left(k\right) \enspace.
    \end{equation*}
  \end{definition}

  Let $\mathrm{E \mhyphen ibs}(k) = \underset{\adversary \in
  \mathtt{PPT}}{\sup}\{\Pr[\mathsf{IBS \mhyphen EUF \mhyphen
  CMA}^{\adversary}\left(1^k\right) = 1]\}$. Then
  Definition~\ref{def:ibs:secure} is equivalent to the following:

  \begin{definition}
    \label{def:ibs:secure:sup}
    An Identity Based Signatures scheme is \emph{\textsf{IBS-EUF-CMA}-secure}
    if
    \begin{equation*}
      \forall k \in \mathbb{N}, \mathrm{E \mhyphen ibs}(k) =
      \mathit{negl}\left(k\right) \enspace.
    \end{equation*}
  \end{definition}

  \subsection{Construction}
    We here define the particular construction for Identity Based Signatures
    used in LN and prove its security.
    Parameters: $\mathcal{H}, G$
    \begin{algorithmic}[0]
      \State \textsc{Setup}($1^k$, rand):
      \Indent
        \State \Return (rand, $G \cdot \mathrm{rand}$)
      \EndIndent
    \end{algorithmic}

    \begin{algorithmic}[0]
      \State \textsc{KeyDer}($mpk, msk, l$):
      \Indent
        \State $pk \gets mpk + \mathcal{H}\left(l \concat mpk\right) \cdot G$
        \State $sk \gets msk + \mathcal{H}\left(l \concat mpk\right) \cdot G$
        \State \Return $(pk, sk)$
      \EndIndent
    \end{algorithmic}

    \begin{algorithmic}[0]
      \State \textsc{PubKeyDer}($mpk, l$):
      \Indent
        \State \Return $mpk + \mathcal{H}\left(l \concat mpk\right) \cdot G$
      \EndIndent
    \end{algorithmic}

    \begin{algorithmic}[0]
      \State \textsc{SignIBS}($m, sk_l$):
      \Indent
        \State \Return \textsc{SignDS}($m, sk_l$)
      \EndIndent
    \end{algorithmic}

    \begin{algorithmic}[0]
      \State \textsc{VerifyIBS}($\sigma, m, pk_l$):
      \Indent
        \State \Return \textsc{VerifyDS}($\sigma, m, pk_l$)
      \EndIndent
    \end{algorithmic}

    \begin{theorem}
      \label{theorem:ibs}
      The construction above is \textsf{IBS-EUF-CMA}-secure in the Random Oracle
      model under the assumption that the underlying signature scheme is
      strongly \textsf{EUF-CMA}-secure and the range of the Random Oracle
      coincides with that of the underlying signature scheme signing keys.
    \end{theorem}
