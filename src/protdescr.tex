\section{Overview of our Lightning Protocol $\Pi_{\mathrm{LN}}$}
  In order to prove that software adhering to the LN specification is secure
  with respect to the guarantees given by \fpaynet, it is necessary to define a
  concrete protocol that implements LN in the UC model. To that end we define
  the formal protocol $\Pi_{\mathrm{LN}}$, an overview of which is given here.
  The full definition can be found in Appendix~\ref{appendix:protocol}.

  For the rest of this section, we will assume that \alice, \bob{} and
  \charlie{} ITIs honestly execute $\Pi_{\mathrm{LN}}$. Similarly to the ideal
  world, the main functions of $\Pi_{\mathrm{LN}}$ are triggered when it
  receives (\textsc{openChannel}), (\textsc{pay}) and (\textsc{closeChannel})
  from \environment{}. These three messages along with (\textsc{getNews})
  informally correspond to actions that a ``human user'' would instruct the
  system to perform. (\textsc{register}) and (\textsc{toppedUp}) are sent by
  \environment{} for player intialization. The rest of the messages sent from
  \environment{} prompt $\Pi_{\mathrm{LN}}$ to perform actions that a software
  implementation would spontaneously perform periodically. All messages sent
  between \alice, \bob{} and \charlie{} correspond to messages specified by LN.
  For clarity of exposition, we avoid mentioning the exact name and contents of
  every message. We refer the reader to the formal definition of
  $\Pi_{\mathrm{LN}}$ for further details.

  \subsection{Registration}
    Before \alice{} can use the network, \environment{} first has to send her a
    (\textsc{register}, delay, relayDelay) message. She generates her
    persistent key and sends it back to \environment{}. The latter may choose to
    add some funds to this key and then send (\textsc{toppedUp}) to \alice, who
    checks her state in \ledger{} and records her on-chain balance.

  \subsection{Channel opening}
    When she receives (\textsc{openChannel}, \alice, \bob, $x$, \textit{tid})
    from \environment, \alice{} initiates the message sequence needed to open a
    channel with \bob, funded by her with $x$ coins. She first generates and
    sends to \bob{} some keys and her timelock delay, who also generates some
    keys and sends them back along with his timelock delay. \alice{} then builds
    the funding transaction using \bob's keys and sends its signature back to
    \bob. He again mirrors \alice's steps, sending back his signature. Both
    parties can now unilaterally spend the funding transaction, so \alice{}
    submits it to \ledger.

    At a later point \environment{} may send (\textsc{checkForNew}, \alice,
    \bob, \textit{tid}) to \alice. She then checks if her state in \ledger{}
    contains the funding transaction with the temporary ID \textit{tid} and in
    that case she generates a new ``commitment'' key for the next update and
    sends it to \bob. \bob{} also confirms that his state contains the funding
    transaction, generates his next commitment key and sends it back to \alice.
    The channel is now open. Both parties keep a note to give \environment{}
    a receipt of the new channel the next time they receive (\textsc{getNews}).

  \subsection{Channel closing}
    When she receives (\textsc{closeChannel}, \texttt{receipt}, \textit{tid})
    from \environment, \alice{} checks that \texttt{receipt} corresponds to the
    latest state of the channel and submits to \ledger{} the latest commitment
    transaction, along with all the relevant HTLC transactions. It also takes a
    note to give \environment{} a receipt of the closed channel the next time
    she receives (\textsc{getNews}).

  \subsection{Performing Payments}
    When she receives (\textsc{pay}, \charlie, $x$,
    $\overrightarrow{\mathtt{path}}$) from \environment, \alice{} attempts to
    pay \charlie{} $x$ coins, using the provided
    $\overrightarrow{\mathtt{path}}$. Let us assume that the path is \alice{} --
    \bob{} -- \charlie. \alice{} asks \charlie{} for an invoice with the HTLC
    hash, to which \charlie{} reacts by choosing a random preimage and sending
    back to \alice{} its hash. \alice{} then prepares a Sphinx~\cite{sphinx}
    onion packet with the relevant information for each party on the
    $\overrightarrow{\mathtt{path}}$ and sends it to \bob{} along with the hash.
    \bob{} peels his layer of the onion and, after performing sanity checks, he
    takes a note of this pending HTLC. He does not yet forward the onion to
    \charlie, because \alice{} is not yet committed to paying \bob. The latter
    happens if \alice{} subsequently receives (\textsc{commit},
    $\mathit{pchid}_1$) from \environment, where $\mathit{pchid}_1$ is the ID of
    the \alice{} -- \bob{} channel. She then sends \bob{} all the signatures
    needed to make the new commitment transaction spendable, who replies with
    the secret commitment key of the old commitment transaction (thus revoking
    it), along with the public commitment key of the future commitment
    transaction (to allow \alice{} to prepare the next update, when that
    happens). LN demands that before \bob{} forwards the onion, he also should
    commit to the new channel version (that includes the HTLC) to \alice. This
    happens if he receives a (\textsc{commit}) message from \environment, which
    causes a similar exchange as above, but with the roles swapped. Now that
    both parties have the HTLC in their commitment transaction and all past
    commitment transactions are revoked, they consider this HTLC ``irrevocably
    committed''.

    \bob{} may then receive (\textsc{pushAdd}, $\mathit{pchid}_1$) from
    \environment. Given that the HTLC is irrevocably committed, \bob{} sends the
    onion to \charlie, who in turn peels it, recognizes that the payment is for
    him and that indeed he knows the preimage (since he generated it himself)
    and waits for the HTLC between him and \bob{} to be irrevocably committed.
    After both \bob{} and \charlie{} receive (\textsc{commit}), \charlie{}
    awaits for a (\textsc{pushFulfill}, \textit{pchid}) message from
    \environment. If it arrives, \charlie{} sends the preimage to \bob, who
    sends it back to \alice. Once more every party has to receive a
    (\textsc{commit}) message for each of the channels it participates in order
    to remove the HTLC and update the definitive balance of each player to the
    righteous value after the payment is complete. After this last update, each
    party keeps a note to inform \environment{} about the new balance when it
    receives (\textsc{getNews}).

    Observe that while locked up in an HTLC, funds do not belong to either
    player; they are rather in a temporary, transitive state. If one party
    learns the preimage, the funds become theirs, whereas if it does not learn
    the preimage after some time, the other party is entitled to the funds. Also
    observe that within the UC framework the necessary messages \textsc{commit},
    \textsc{pushFulfill} and \textsc{pushAdd} may never arrive, but in a
    correct software implementation the corresponding actions happen
    automatically, without waiting for a prompt by the user.

  \subsection{Polling}
    Lastly, \environment{} may send (\textsc{poll}) to \alice. She then reads
    her state in \ledger{} and checks for closed channels. If she finds
    maliciously closed channels (closed using old commitments), she punishes the
    counterparty and takes all the funds in the channel. If she finds in an
    honestly closed channel a preimage of an HTLC that she has previously
    signed and for which she is an intermediary, she records it and prepares to
    send it when she receives (\textsc{pushFulfill}). Finally, if she finds an
    honestly closed channel with an HTLC output for which she knows the
    preimage, she spends it immediately. For every closed channel she finds, she
    keeps a note to report it to \environment{} the next time she receives
    (\textsc{getNews}).

  \begin{remark}[Differences between LN and $\Pi_{\mathrm{LN}}$]

    In LN, a custom construction for generating a new secret during each channel
    update is used. It reduces the amount of space needed to maintain a channel
    from $O(n)$ to $O(\log n)$ in the number of updates. As far as we know, its
    security has not been formally analyzed. In the current paper we use instead
    a PRF~\cite{prf}.

    As mentioned earlier, LN uses a custom construction that takes advantage of
    elliptic curve homomorphic properties in order to derive any number of
    keypairs by combining a single ``basepoint'' with different ``labels''. We
    instead use the Identity Based Signature
    primitive~\cite{ibsshamir,ibspaterson} in this work to abstract the
    properties provided by the construction.

    additionally, we have chosen to simplify the protocol in a number of ways in
    order to keep the analysis tractable. In particular LN defines several
    additional messages that signal various types of errors in transmission. It
    also specifies exactly how message retransmission should happen upon
    reconnection, specifically for the case of connection failure while updating
    a channel. This allows for a more robust system by excluding many cases of
    accidental channel closures. What is more, an LN user can change their
    ``delay'' and ``relayDelay'' parameters even after registration, which is
    not the case in $\Pi_{\mathrm{LN}}$.

    In order to incentivize users to act as intermediaries or check for channel
    closures on behalf of others, LN provides for fees for these two roles.
    Furthermore, in order to reduce transaction size, it specifies exact rules
    for prunning outputs of too low value (known as ``dust outputs''). In the
    current analysis we do not consider these features.

    Last but not least, LN makes it possible for parties to cooperatively close
    a channel, thus avoiding the need to wait for the expiry of timelocks and
    reducing the size of the transactions that are added to the blockchain. As
    we mentioned earlier, we do not analyze this part of the specification.
  \end{remark}
