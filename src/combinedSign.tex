\section{Combined Signatures primitive}
  \label{appendix:combinedsign}

  The seven algorithms used by a Combined Signatures scheme are:
  \begin{itemize}
    \item $\left(mpk, msk\right) \gets \textsc{MasterKeyGen}\left(1^k\right)$
    \item $\left(pk, sk\right) \gets \textsc{KeyShareGen}\left(1^k\right)$
    \item $cpk_l \gets \textsc{CombinePubKey}\left(mpk, pk\right)$
    \item $\left(cpk_l, csk_l\right) \gets \textsc{CombineKey}\left(mpk,
    msk, pk, sk\right)$
    \item $\{0, 1\} \gets \textsc{TestKey}\left(pk, sk\right)$
    \item $\sigma \gets \textsc{SignCS}\left(m, csk\right)$
    \item $\left\{0, 1\right\} \gets \textsc{VerifyCS}\left(\sigma, m,
    cpk\right)$
  \end{itemize}
  We demand that these three properties hold for a scheme to have correctness:
  \begin{itemize}
    \item $\forall k \in \mathbb{N},$ \\
    $\Pr[\left(pk, sk\right) \gets \textsc{KeyShareGen}\left(1^k\right),$ \\
    $\textsc{TestKey}(pk, sk) = 1] = 1$

    I.e. \textsc{KeyShareGen}() must always generate a valid keypair.

    \item $\forall k \in \mathbb{N},$ \\
    $\Pr[(mpk, msk) \gets \textsc{MasterKeyGen}\left(1^k\right),$ \\
    $\left(pk, sk\right) \gets \textsc{KeyShareGen}\left(1^k\right),$ \\
    $\left(cpk_1, csk_1\right) \gets \textsc{CombineKey}\left(mpk, msk, pk,
    sk\right),$ \\
    $cpk_2 \gets \textsc{CombinePubKey}\left(mpk, pk\right),$ \\
    $cpk_1 = cpk_2] = 1$

    I.e. for suitable input, \textsc{CombinePubKey}() and \textsc{CombineKey}()
    produce the same public key.

    \item $\forall k \in \mathbb{N}, m \in \mathcal{M},$ \\
    $\Pr[(mpk, msk) \gets \textsc{MasterKeyGen}\left(1^k\right),$ \\
    $\left(pk, sk\right) \gets \textsc{KeyShareGen}\left(1^k\right),$ \\
    $\left(cpk, csk\right) \gets \textsc{CombineKey}\left(mpk, msk, pk,
    sk\right),$ \\
    $\textsc{VerifyCS}(\textsc{SignCS}(m, csk), m, cpk) = 1] = 1$

    I.e. for suitable input, honestly generated signatures always verify
    correctly.
  \end{itemize}
  Last but not least, the security of the scheme is defined in terms of two
  games:
  \begin{figure}[!htbp]
    \begin{gamebox}{$\mathsf{share \mhyphen
    EUF}^{\adversary}\left(1^k\right)$}
      \begin{algorithmic}[1]
        \State $\left(\mathtt{aux}, mpk, n\right) \gets
        \adversary\left(\textsc{init}\right)$
        \For{$i \gets 1$ to $n$}
          \State $\left(pk_i, sk_i\right) \gets
          \textsc{KeyShareGen}\left(1^k\right)$
        \EndFor
        \State $\left(cpk^*, pk^*, m^*, \sigma^*\right) \gets
        \adversary\left(\textsc{keys}, \mathtt{aux}, pk_1, \dots, pk_n\right)$
        \If{$pk^* \in \left\{pk_1, \dots, pk_n\right\} \wedge$
        $cpk^* = \textsc{CombinePubKey}\left(mpk, pk^*\right) \wedge$
        $\textsc{VerifyCS}\left(\sigma^*, m^*, cpk^*\right) = 1$}
          \State \Return 1
        \Else
          \State \Return 0
        \EndIf
      \end{algorithmic}
    \end{gamebox}
    \caption{Key Share: Existential Unforgeability -- Key-Only Attack}
    \label{game:comb:share}
  \end{figure}
  \begin{definition}
    \label{def:share:secure}
    A Combined Signatures scheme is \emph{\textsf{share-EUF}-secure} if
    \begin{gather*}
      \forall k \in \mathbb{N}, \forall \adversary \in \mathtt{PPT},
      \Pr\left[\mathsf{share \mhyphen EUF}^{\adversary}\left(1^k\right) =
      1\right] = \mathit{negl}\left(k\right) \text{or equivalently} \\
      \forall k \in \mathbb{N}, \mathrm{E \mhyphen share}(k) =
      \mathit{negl}\left(k\right) \enspace, \\
      \text{where } \mathrm{E \mhyphen share}(k) = \underset{\adversary \in
      \mathtt{PPT}}{\sup}\{\Pr[\mathsf{share \mhyphen
      EUF}^{\adversary}\left(1^k\right) = 1]\} \text{ (cf.
      Fig~\ref{game:comb:share})} \enspace.
    \end{gather*}
  \end{definition}

  \begin{figure}[!htbp]
    \begin{gamebox}{$\mathsf{master \mhyphen EUF \mhyphen
    CMA}^{\adversary}\left(1^k\right)$}
      \begin{algorithmic}[1]
        \State $\left(mpk, msk\right) \gets
        \textsc{MasterKeyGen}\left(1^k\right)$
        \State $i \gets 0$
        \State $\left(\mathtt{aux}_i, \mathrm{response}\right) \gets
        \adversary\left(\textsc{init}, mpk\right)$
        \While{response can be parsed as $\left(pk, sk, m\right)$}
          \State $i \gets i + 1$
          \State store $pk, sk, m$ as $pk_i, sk_i, m_i$
          \State $\left(cpk_i, csk_i\right) \gets
          \textsc{CombineKey}\left(mpk, msk, pk_i, sk_i\right)$
          \State $\sigma_i \gets \textsc{SignCS}\left(m_i, csk_i\right)$
          \State $\left(\mathtt{aux}_i, \mathrm{response}\right) \gets
          \adversary\left(\textsc{signature}, \mathtt{aux}_{i-1},
          \sigma_i\right)$
        \EndWhile
        \State parse response as $\left(cpk^*, pk^*, m^*, \sigma^*\right)$
        \If{$m^* \notin \left\{m_1, \dots, m_i\right\} \wedge$
        $cpk^* = \textsc{CombinePubKey}\left(mpk, pk^*\right) \wedge$
        $\textsc{VerifyCS}\left(\sigma^*, m^*, cpk^*\right) = 1$}
          \State \Return 1
        \Else
          \State \Return 0
        \EndIf
      \end{algorithmic}
    \end{gamebox}
    \caption{Master Key: Existential Unforgeability -- Chosen Message Attack}
    \label{game:comb:master}
  \end{figure}
  \begin{definition}
    \label{def:master:secure}
    A Combined Signatures scheme is \emph{\textsf{master-EUF-CMA}-secure} if
    \begin{gather*}
      \forall k \in \mathbb{N}, \forall \adversary \in \mathtt{PPT},
      \Pr\left[\mathsf{master \mhyphen EUF \mhyphen
      CMA}^{\adversary}\left(1^k\right) = 1\right] =
      \mathit{negl}\left(k\right) \text{or equivalently} \\
      \forall k \in \mathbb{N}, \mathrm{E \mhyphen master}(k) =
      \mathit{negl}\left(k\right) \enspace, \\
      \text{where } \mathrm{E \mhyphen master}(k) = \underset{\adversary \in
      \mathtt{PPT}}{\sup}\{\Pr[\mathsf{master \mhyphen
      EUF-CMA}^{\adversary}\left(1^k\right) = 1]\} \text{ (cf.
      Fig~\ref{game:comb:master})} \enspace.
    \end{gather*}
  \end{definition}

  \begin{definition}
    A Combined Signatures scheme is \emph{\textsf{combine-EUF}-secure} if it
    is both \textsf{share-EUF}-secure and \textsf{master-EUF-CMA}-secure.
  \end{definition}

  In conclusion, a collection of algoritms is said to be a secure Combined
  Signatures scheme if it conforms to the syntax of the seven aforementioned
  algorithms, it satisfies the three correctness properties and provides
  existential unforgeability against key-only attacks with respect to key shares
  and existential unforgeability against chosen message attacks with respect to
  master keys.

  \subsection{Construction}
    We here define the particular construction for Combined Signatures used in
    LN and prove its security.

    Parameters: hash function $\mathcal{H}$, group generator $G$
    \begin{algorithmic}[0]
      \State \textsc{MasterKeyGen}($1^k$, rand):
      \Indent
        \State \Return ($G \cdot \mathrm{rand}$, rand)
      \EndIndent
    \end{algorithmic}

    \begin{algorithmic}[0]
      \State \textsc{KeyShareGen}($1^k$, rand):
      \Indent
        \State \Return ($G \cdot \mathrm{rand}$, rand)
      \EndIndent
    \end{algorithmic}

    \begin{algorithmic}[0]
      \State \textsc{CombinePubKey}($mpk, pk$):
      \Indent
        \State \Return $mpk \cdot \mathcal{H}\left(mpk \concat pk\right) + pk
        \cdot \mathcal{H}\left(pk \concat mpk\right)$
      \EndIndent
    \end{algorithmic}

    \begin{algorithmic}[0]
      \State \textsc{CombineKey}($mpk, msk, pk, sk$):
      \Indent
        \State \Return $(\textsc{CombinePubKey}(mpk, pk), msk \cdot
        \mathcal{H}\left(mpk \concat pk\right) + sk \cdot \mathcal{H}\left(pk
        \concat mpk\right))$
      \EndIndent
    \end{algorithmic}

    \begin{algorithmic}[0]
      \State \textsc{TestKey}($pk, sk$):
      \Indent
        \If{$pk = G \cdot sk$}
          \State \Return 1
        \Else
          \State \Return 0
        \EndIf
      \EndIndent
    \end{algorithmic}

    \begin{algorithmic}[0]
      \State \textsc{SignCS}($m, csk$):
      \Indent
        \State \Return \textsc{SignDS}($m, csk$)
      \EndIndent
    \end{algorithmic}

    \begin{algorithmic}[0]
      \State \textsc{VerifyCS}($\sigma, m, cpk$):
      \Indent
        \State \Return \textsc{VerifyDS}($\sigma, m, cpk$)
      \EndIndent
    \end{algorithmic}

    \begin{lemma}
      \label{lemma:comb:share}
      The construction above is \textsf{share-EUF}-secure in the Random Oracle
      model under the assumption that the underlying signature scheme is
      strongly \textsf{EUF-CMA}-secure and the range of the Random Oracle
      coincides with that of the underlying signature scheme signing keys.
    \end{lemma}

    \begin{proof}
      Let $k \in \mathbb{N}, \mathcal{B}$ PPT algorithm such that
      \begin{equation*}
        \Pr\left[\mathsf{share \mhyphen EUF}^{\mathcal{B}}\left(1^k\right) =
        1\right] = a > \mathrm{negl}\left(k\right) \enspace.
      \end{equation*}
      We construct a PPT distinguisher \adversary{}
      (Fig.~\ref{proof:comb:share:distinguisher}) such that
      \begin{equation*}
        \Pr\left[\mathsf{EUF \mhyphen CMA}^{\adversary}\left(1^k\right) =
        1\right] > \mathrm{negl}\left(k\right)
      \end{equation*}
      that breaks the assumption, thus proving Lemma~\ref{lemma:comb:share}.

      \begin{figure}[!htbp]
        \begin{algobox}{$\adversary\left(vk\right)$}
          \begin{algorithmic}[1]
            \State $j \overset{\$}{\gets} U\left[1,
            T\left(\mathcal{B}\right)\right]$
            \Comment{$T\left(M\right)$ is the maximum running time of $M$}
            \Indent
              \State Random Oracle: for every first-seen query $q$ from
              $\mathcal{B}$ set $\mathcal{H}\left(q\right)$ to a random value
              \State \Return $\mathcal{H}\left(q\right)$ to $\mathcal{B}$
            \EndIndent
            \State $\left(\mathtt{aux}, mpk, n\right) \gets
            \adversary\left(\textsc{init}\right)$
            \For{$i \gets 1$ to $n$}
              \State $\left(pk_i, sk_i\right) \gets
              \textsc{KeyShareGen}\left(1^k\right)$
            \EndFor
            \Indent
              \State Random Oracle: Let $q$ be the $j$th first-seen query from
              $\mathcal{B}$:
              \If{$q = \left(mpk \concat x\right)$}
              \label{proof:comb:share:ro:start}
                \If{$\mathcal{H}\left(x \concat mpk\right)$ unset}
                  \State set $\mathcal{H}\left(x \concat mpk\right)$ to a random
                  value
                \EndIf
                \State set $\mathcal{H}\left(mpk \concat x\right)$ to $\left(vk
                - x \cdot \mathcal{H}\left(x \concat mpk\right)\right) \cdot
                mpk^{-1}$
              \ElsIf{$q = \left(x \concat mpk\right)$}
                \If{$\mathcal{H}\left(mpk \concat x\right)$ unset}
                  \State set $\mathcal{H}\left(mpk \concat x\right)$ to a random
                  value
                \EndIf
                \State set $\mathcal{H}\left(x \concat mpk\right)$ to $\left(vk
                - mpk \cdot \mathcal{H}\left(mpk \concat x\right)\right) \cdot
                x^{-1}$
                \label{proof:comb:share:ro:end}
              \Else
                \State set $\mathcal{H}\left(q\right)$ to a random value
              \EndIf
              \State \Return $\mathcal{H}\left(q\right)$ to $\mathcal{B}$
            \EndIndent
            \State $\left(cpk^*, pk^*, m^*, \sigma^*\right) \gets
            \mathcal{B}\left(\textsc{keys}, \mathtt{aux}, pk_1, \dots,
            pk_n\right)$
            \If{$vk = cpk^* \wedge \mathcal{B}$ wins the \textsf{share-EUF}
            game} \Comment{\adversary{} won the \textsf{EUF-CMA} game}
            \label{proof:comb:share:distinguisher:won}
              \State \Return $\left(m^*, \sigma^*\right)$
            \Else
              \State \Return \textsc{fail}
            \EndIf
          \end{algorithmic}
        \end{algobox}
        \caption{}
        \label{proof:comb:share:distinguisher}
      \end{figure}

      Let $Y$ be the range of the random oracle. The modified random oracle used
      in Fig.~\ref{proof:comb:share:distinguisher} is indistinguishable from the
      standard random oracle by PPT algorithms since the statistical distance of
      the standard random oracle from the modified one is at most
      $\frac{1}{2|Y|} < \mathit{negl}\left(k\right)$ as they differ in at most
      one element.

      Let $E$ denote the event in which $\mathcal{B}$ does not invoke
      \textsc{CombinePubKey} to produce $cpk^*$. In that case the values
      $\mathcal{H}\left(pk^* \concat mpk\right)$ and $\mathcal{H}\left(mpk
      \concat pk^*\right)$ are decided after $\mathcal{B}$ terminates
      (Fig.~\ref{proof:comb:share:distinguisher},
      line~\ref{proof:comb:share:distinguisher:won}) and thus
      \begin{equation}
        \begin{gathered}
          \Pr\left[cpk^* = \textsc{CombinePubKey}\left(mpk, pk^*\right) |
          E\right] = \frac{1}{|Y|} < \mathit{negl}\left(k\right) \Rightarrow \\
          \Pr\left[cpk^* = \textsc{CombinePubKey}\left(mpk, pk^*\right)
          \wedge E\right] < \mathit{negl}\left(k\right) \enspace.
        \end{gathered}
        \label{proof:comb:share:nocomb}
      \end{equation}
      It is
      \begin{gather*}
        \left(\mathcal{B} \text{ wins}\right) \rightarrow \left(cpk^* =
        \textsc{CombinePubKey}\left(mpk, pk^*\right)\right) \Rightarrow \\
        \Pr\left[\mathcal{B} \text{ wins}\right] \leq \Pr\left[cpk^* =
        \textsc{CombinePubKey}\left(mpk, pk^*\right)\right] \Rightarrow \\
        \Pr\left[\mathcal{B} \text{ wins} \wedge E\right] \leq \Pr\left[cpk^* =
        \textsc{CombinePubKey}\left(mpk, pk^*\right) \wedge E\right]
        \overset{\left(\ref{proof:comb:share:nocomb}\right)}{\Rightarrow} \\
        \Pr\left[\mathcal{B} \text{ wins} \wedge E\right] <
        \mathit{negl}\left(k\right) \enspace.
      \end{gather*}

      But we know that $\Pr\left[\mathcal{B} \text{ wins}\right] =
      \Pr\left[\mathcal{B} \text{ wins} \wedge E\right] + \Pr\left[\mathcal{B}
      \text{ wins} \wedge \neg E\right]$ and $\Pr\left[\mathcal{B} \text{
      wins}\right] = a$ by the assumption, thus
      \begin{equation}
        \label{proof:comb:share:nohash}
        \Pr\left[\mathcal{B} \text{ wins} \wedge \neg E\right] > a -
        \mathit{negl}\left(k\right) \enspace.
      \end{equation}

      We now focus at the event $\neg E$. Let $F$ the event in which the call of
      $\mathcal{B}$ to \textsc{CombinePubKey} to produce $cpk^*$ results in the
      $j$th invocation of the Random Oracle. Since $j$ is chosen uniformly at
      random and using Proposition~\ref{prop:distrib}, $\Pr\left[F | \neg
      E\right] = \frac{1}{T\left(\mathcal{B}\right)}$. Observe that $\Pr\left[F
      | E\right] = 0 \Rightarrow \Pr\left[F\right] = \Pr\left[F | \neg E\right]
      = \frac{1}{T\left(\mathcal{B}\right)}$.

      In the case where the event $\left(F \wedge \mathcal{B} \text{ wins}
      \wedge \neg E\right)$ holds, it is
      \begin{gather*}
        cpk^* = \textsc{CombinePubKey}\left(mpk, pk^*\right) = \\
        mpk \cdot \mathcal{H}\left(mpk \concat pk^*\right) + pk^* \cdot
        \mathcal{H}\left(pk^* \concat mpk\right)
      \end{gather*}
      Since $F$ holds, the $j$th invocation of the Random Oracle queried either
      the value $\mathcal{H}\left(mpk \concat pk^*\right)$ or
      $\mathcal{H}\left(pk^* \concat mpk\right)$. In either case
      (Fig.~\ref{proof:comb:share:distinguisher},
      lines~\ref{proof:comb:share:ro:start}-\ref{proof:comb:share:ro:end}), it
      is $cpk^* = vk$. This means that $\textsc{VerifyCS}\left(\sigma^*, m^*,
      vk\right) = 1$. We conclude that in the event $\left(F \wedge \mathcal{B}
      \text{ wins} \wedge \neg E\right)$, \adversary{} wins the \textsf{EUF-CMA}
      game. A final observation is that the probability that the events
      $\left(\mathcal{B} \text{ wins} \wedge \neg E\right)$ and $F$ are almost
      independent, thus
      \begin{gather*}
        \Pr\left[F \wedge \mathcal{B} \text{ wins} \wedge \neg E\right] =
        \Pr\left[F\right] \Pr\left[\mathcal{B} \text{ wins} \wedge \neg E\right]
        \pm \mathit{negl}\left(k\right)
        \overset{\left(\ref{proof:comb:share:nohash}\right)}{=} \\
        \frac{a - \mathit{negl}\left(k\right)}{T\left(\mathcal{B}\right)} \pm
        \mathit{negl}\left(k\right) > \mathit{negl}\left(k\right)
      \end{gather*}
    \end{proof}

    \begin{lemma}
      \label{lemma:comb:master}
      The construction above is \textsf{master-EUF-CMA}-secure in the Random
      Oracle model under the assumption that the underlying signature scheme is
      strongly \textsf{EUF-CMA}-secure and the range of the Random Oracle
      coincides with that of the underlying signature scheme signing keys.
    \end{lemma}

    \begin{proof}
      Let $k \in \mathbb{N}, \mathcal{B}$ PPT algorithm such that
      \begin{equation*}
        \Pr\left[\mathsf{master \mhyphen EUF \mhyphen
        CMA}^{\mathcal{B}}\left(1^k\right) = 1\right] = a >
        \mathrm{negl}\left(k\right) \enspace.
      \end{equation*}
      We construct a PPT distinguisher \adversary{}
      (Fig.~\ref{proof:comb:master:distinguisher}) such that
      \begin{equation*}
        \Pr\left[\mathsf{EUF \mhyphen CMA}^{\adversary}\left(1^k\right) =
        1\right] > \mathrm{negl}\left(k\right)
      \end{equation*}
      that breaks the assumption, thus proving Lemma~\ref{lemma:comb:master}.

      \begin{figure}[!htbp]
        \begin{algobox}{$\adversary\left(vk\right)$}
          \begin{algorithmic}[1]
            \State $j \overset{\$}{\gets} U\left[1, T\left(\mathcal{B}\right) +
            T\left(\adversary\right)\right]$
            \Comment{$T\left(M\right)$ is the maximum running time of $M$}
            \Indent
              \State Random Oracle: for every first-seen query $q$ from
              $\mathcal{B}$ set $\mathcal{H}\left(q\right)$ to a random value
              \State \Return $\mathcal{H}\left(q\right)$ to $\mathcal{B}$
            \EndIndent
            \State $\left(mpk, msk\right) \gets
            \textsc{MasterKeyGen}\left(1^k\right)$
            \Indent
              \State Random Oracle: Let $q$ be the $j$th first-seen query from
              $\mathcal{B}$ or \adversary:
              \If{$q = \left(mpk \concat x\right)$}
                \If{$\mathcal{H}\left(x \concat mpk\right)$ unset}
                  \State set $\mathcal{H}\left(x \concat mpk\right)$ to a random
                  value
                \EndIf
                \State set $\mathcal{H}\left(mpk \concat x\right)$ to $\left(vk
                - x \cdot \mathcal{H}\left(x \concat mpk\right)\right) \cdot
                mpk^{-1}$
              \ElsIf{$q = \left(x \concat mpk\right)$}
                \If{$\mathcal{H}\left(mpk \concat x\right)$ unset}
                  \State set $\mathcal{H}\left(mpk \concat x\right)$ to a random
                  value
                \EndIf
                \State set $\mathcal{H}\left(x \concat mpk\right)$ to $\left(vk
                - mpk \cdot \mathcal{H}\left(mpk \concat x\right)\right) \cdot
                x^{-1}$
              \Else
                \State set $\mathcal{H}\left(q\right)$ to a random value
              \EndIf
              \State \Return $\mathcal{H}\left(q\right)$ to $\mathcal{B}$ or
              \adversary
            \EndIndent
            \State $i \gets 0$
            \State $\left(\mathtt{aux}_i, \mathrm{response}\right) \gets
            \mathcal{B}\left(\textsc{init}, mpk\right)$
            \While{response can be parsed as $\left(pk, sk, m\right)$}
              \State $i \gets i + 1$
              \State store $pk, sk, m$ as $pk_i, sk_i, m_i$
              \State $\left(cpk_i, csk_i\right) \gets
              \textsc{CombineKey}\left(mpk, msk, pk_i, sk_i\right)$
              \State $\sigma_i \gets \textsc{SignCS}\left(m_i, csk_i\right)$
              \State $\left(\mathtt{aux}_i, \mathrm{response}\right) \gets
              \mathcal{B}\left(\textsc{signature}, \mathtt{aux}_{i-1},
              \sigma_i\right)$
            \EndWhile
            \State parse response as $\left(cpk^*, pk^*, m^*, \sigma^*\right)$
            \If{$vk = cpk^* \wedge \mathcal{B}$ wins the \textsf{master-EUF-CMA}
            game} \Comment{\adversary{} won the \textsf{EUF-CMA} game}
            \label{proof:comb:master:distinguisher:won}
              \State \Return $\left(m^*, \sigma^*\right)$
            \Else
              \State \Return \textsc{fail}
            \EndIf
          \end{algorithmic}
        \end{algobox}
        \caption{}
        \label{proof:comb:master:distinguisher}
      \end{figure}

      The modified random oracle used in
      Fig.~\ref{proof:comb:master:distinguisher} is indistinguishable from the
      standard random oracle for the same reasons as in the proof of
      Lemma~\ref{lemma:comb:share}.

      Let $E$ denote the event in which \textsc{CombinePubKey} is not invoked to
      produce $cpk^*$. In that case the values $\mathcal{H}\left(pk^* \concat
      mpk\right)$ and $\mathcal{H}\left(mpk \concat pk^*\right)$ are decided
      after $\mathcal{B}$ terminates
      (Fig.~\ref{proof:comb:master:distinguisher},
      line~\ref{proof:comb:master:distinguisher:won}) and thus
      \begin{equation}
        \begin{gathered}
          \Pr\left[cpk^* = \textsc{CombinePubKey}\left(mpk, pk^*\right) |
          E\right] < \mathit{negl}\left(k\right) \Rightarrow \\
          \Pr\left[cpk^* = \textsc{CombinePubKey}\left(mpk, pk^*\right) \wedge
          E\right] < \mathit{negl}\left(k\right) \enspace.
        \end{gathered}
        \label{proof:comb:master:nocomb}
      \end{equation}
      We can reason like in the proof of Lemma~\ref{lemma:comb:share} to deduce
      that
      \begin{equation}
        \label{proof:comb:master:nohash}
        \Pr\left[\mathcal{B} \text{ wins} \wedge \neg E\right] > a -
        \mathit{negl}\left(k\right) \enspace.
      \end{equation}

      We now focus at the event $\neg E$. Let $F$ the event in which the call of
      to \textsc{CombinePubKey} that produces $cpk^*$ results in the $j$th
      invocation of the Random Oracle. Since $j$ is chosen uniformly at random
      and using Proposition~\ref{prop:distrib}, $\Pr\left[F | \neg E\right] =
      \frac{1}{T\left(\mathcal{B}\right) + T\left(\adversary\right)}$. Observe
      that $\Pr\left[F | E\right] = 0 \Rightarrow \Pr\left[F\right] = \Pr\left[F
      | \neg E\right] = \frac{1}{T\left(\mathcal{B}\right) +
      T\left(\adversary\right)}$.

      Once more we can reason in the same fashion as in the proof of
      Lemma~\ref{lemma:comb:share} to deduce that
      \begin{gather*}
        \Pr\left[F \wedge \mathcal{B} \text{ wins} \wedge \neg E\right] =
        \Pr\left[F\right] \Pr\left[\mathcal{B} \text{ wins} \wedge \neg E\right]
        \pm \mathit{negl}\left(k\right)
        \overset{\left(\ref{proof:comb:master:nohash}\right)}{=} \\
        \frac{a - \mathit{negl}\left(k\right)}{T\left(\mathcal{B}\right) +
        T\left(\adversary\right)} \pm \mathit{negl}\left(k\right) >
        \mathit{negl}\left(k\right)
      \end{gather*}
    \end{proof}

    \begin{theorem}
      \label{thm:combsign}
      The construction above is \textsf{combine-EUF}-secure in the Random Oracle
      model under the assumption that the underlying signature scheme is
      strongly \textsf{EUF-CMA}-secure.
    \end{theorem}

    \begin{proof}
      The construction is \textsf{combine-EUF}-secure as a consequence of
      Lemma~\ref{lemma:comb:share}, Lemma~\ref{lemma:comb:master} and the
      definition of \textsf{combine-EUF}-security.
    \end{proof}
