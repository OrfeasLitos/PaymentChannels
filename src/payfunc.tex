\section{Payment Network Functionality}
  \label{appendix:payfunc}
  \begin{figure}[H]
    \begin{systembox}{\fpaynet{} -- interface}
      \begin{itemize}
        \item from $\environment$:
        \begin{itemize}
          \item (\textsc{register}, delay, relayDelay)
          \item (\textsc{toppedUp})
          \item (\textsc{openChannel}, \alice, \bob, $x$, \textit{tid})
          \item (\textsc{checkForNew}, \alice, \bob, \textit{tid})
          \item (\textsc{pay}, \bob, $x, \overrightarrow{\mathtt{path}}$,
          \textit{expayid})
          \item (\textsc{closeChannel}, \texttt{receipt}, \textit{pchid})
          \item (\textsc{forceCloseChannel}, \texttt{receipt}, \textit{pchid})
          \item (\textsc{poll})
          \item (\textsc{pushFulfill}, \textit{pchid})
          \item (\textsc{pushAdd}, \textit{pchid})
          \item (\textsc{commit}, \textit{pchid})
          \item (\textsc{fulfillOnChain})
          \item (\textsc{getNews})
        \end{itemize}
        \item to $\environment$:
        \begin{itemize}
          \item (\textsc{register}, \alice, \texttt{delay}(\alice),
          \texttt{relayDelay}(\alice), pubKey)
          \item (\textsc{registered})
          \item (\textsc{news}, \texttt{newChannels}, \texttt{closedChannels},
          \texttt{updatesToReport})
        \end{itemize}
        \item from $\simulator$:
        \begin{itemize}
          \item (\textsc{registerDone}, \alice, pubKey)
          \item (\textsc{corrupted}, \alice)
          \item (\textsc{channelAnnounced}, \alice, $p_{\alice, F}, p_{\bob,
          F}$, \textit{fchid}, \textit{pchid}, \textit{tid})
          \item (\textsc{update}, \texttt{receipt}, \texttt{diffs}, \alice)
          \item (\textsc{closedChannel}, \texttt{channel}, \alice)
          \item (\textsc{resolvePays}, \texttt{charged})
        \end{itemize}
        \item to $\simulator$:
        \begin{itemize}
          \item (\textsc{register}, \alice, delay, relayDelay)
          \item (\textsc{openChannel}, \alice, \bob, $x$, \textit{fchid},
          \textit{tid})
          \item (\textsc{channelOpened}, \alice, \textit{fchid})
          \item (\textsc{pay}, \alice, \bob, $x,
          \overrightarrow{\mathtt{path}}$, \textit{expayid}, \textit{payid},
          \textsc{state}, $\Sigma$)
          \item (\textsc{continue})
          \item (\textsc{closeChannel}, \textit{fchid}, \alice)
          \item (\textsc{forceCloseChannel}, \textit{fchid}, \alice)
          \item (\textsc{poll}, $\Sigma_{\alice}$, \alice)
          \item (\textsc{pushFulfill}, \textit{pchid}, \alice)
          \item (\textsc{pushAdd}, \textit{pchid}, \alice)
          \item (\textsc{commit}, \textit{pchid}, \alice)
          \item (\textsc{fulfillOnChain}, $t$, \alice)
        \end{itemize}
      \end{itemize}
    \end{systembox}
    \caption{}
    \label{alg:fpaynet:interface}
  \end{figure}

  All players need to register in order to use channels. The registration of
  \alice{} works as follows: \alice{} inputs her desired delay and relayDelay
  that will be used for all her future channels. The first denotes how often she
  has to check the blockchain for revoked commitments and the second defines the
  minimum time distance between incoming and outgoing CLTV expiries. \fpaynet{}
  then informs \simulator{}, who sends back a long-lived public key for \alice.
  This key represents \alice's account, from where \fpaynet{} can get coins to
  open new channels on her behalf and to place coins of closed channels. The key
  is sent to \alice{} who moves some initial funds to it and notifies
  \fpaynet{}. She is now registered. The exact logic is found in
  Fig.~\ref{alg:fpaynet:support}, which also contains the actions of \fpaynet{}
  related to corruptions.

  Additionally, the procedure \texttt{checkClosed}() is called after
  \textsc{read}ing from \ledger, with the received state $\Sigma$ as input. This
  call happens every time \fpaynet{} \textsc{read}s from \ledger. The formal
  definition of \texttt{checkClosed}() can be found in
  Fig.~\ref{alg:fpaynet:close:func}, along with a discussion of its purpose.

  \begin{figure}[H]
    \begin{systembox}{\fpaynet{} -- registration and corruption}
      \begin{algorithmic}[1]
        \State Initialisation:
        \Indent
          \State $\mathtt{channels}, \mathtt{pendingPay}, \mathtt{pendingOpen}
          \gets \emptyset$
          \State $\texttt{pendingDiffs}, \mathtt{corrupted}, \Sigma \gets
          \emptyset$
        \EndIndent
        \Statex

        \State Upon receiving $\left(\textsc{register}, \mathrm{delay},
        \mathrm{relayDelay}\right)$ from \alice:
        \Indent
          \State $\mathtt{delay}\left(\alice\right) \gets \mathrm{delay}$
          \Comment{Must check chain at least once every \texttt{delay}(\alice)
          blocks}
          \State $\mathtt{relayDelay}\left(\alice\right) \gets
          \mathrm{relayDelay}$
          \State $\mathtt{updatesToReport}\left(\alice\right),
          \mathtt{newChannels}\left(\alice\right) \gets \emptyset$
          \State $\mathtt{polls}\left(\alice\right) \gets \emptyset$
          \State $\mathtt{focs}\left(\alice\right) \gets \emptyset$
          \State send (\textsc{read}) to \ledger{} as \alice{}, store reply to
          $\Sigma_{\alice}$, add $\Sigma_{\alice}$ to $\Sigma$ and add largest
          block number to \texttt{polls}(\alice)
          \label{alg:fpaynet:support:read}
          \State \texttt{checkClosed}($\Sigma_{\alice}$)
          \State send $\left(\textsc{register}, \alice, \mathrm{delay},
          \mathrm{relayDelay}\right)$ to \simulator
        \EndIndent
        \Statex

        \State Upon receiving $\left(\textsc{registerDone}, \alice,
        \mathrm{pubKey}\right)$ from \simulator:
        \Indent
          \State $\mathtt{pubKey}\left(\alice\right) \gets \mathrm{pubKey}$
          \State send (\textsc{register}, \alice, \texttt{delay}(\alice),
          \texttt{relayDelay}(\alice), pubKey) to \alice
        \EndIndent
        \Statex

        \State Upon receiving (\textsc{toppedUp}) from \alice:
        \Indent
          \State send (\textsc{read}) to \ledger{} as \alice{} and store reply
          to $\Sigma_{\alice}$
          \State \texttt{checkClosed}($\Sigma_{\alice}$)
          \State assign the sum of all output values that are exclusively
          spendable by \alice{} to \texttt{onChainBalance}
          \State send (\textsc{registered}) to \alice
        \EndIndent
        \Statex

        \State Upon receiving any message ($M$) except for (\textsc{register})
        or (\textsc{toppedUp}) from \alice:
        \Indent
          \If{if haven't received (\textsc{register}) and (\textsc{toppedUp})
          from \alice{} (in this order)}
            \State send (\textsc{invalid}, $M$) to \alice{} and ignore message
          \EndIf
          \label{alg:fpaynet:support:unreg}
        \EndIndent
        \Statex

        \State Upon receiving $\left(\textsc{corrupted}, \alice\right)$ from
        \simulator:
        \Indent
          \State add \alice{} to \texttt{corrupted}
          \State for the rest of the execution, upon receiving any message for
          \alice{}, bypass normal execution and simply forward it to
          \simulator
        \EndIndent
      \end{algorithmic}
    \end{systembox}
    \caption{}
    \label{alg:fpaynet:support}
  \end{figure}

  The process of \alice{} opening a channel with \bob{} is as follows: First
  \alice{} asks \fpaynet{} to open and \fpaynet{} informs \simulator.
  \simulator{} provides the necessary keys and IDs for the new channel to
  \fpaynet. \alice{} asks \fpaynet{} to check if \ledger{} contains the funding
  transaction from \alice's point of view. If it does, \fpaynet{} activates
  \simulator, who in turn returns control to \fpaynet{}. Now \fpaynet{} checks
  that the funding transaction is in the \ledger{} also from \bob's point of
  view and in case it does, it notifies \simulator. \simulator{} then confirms
  that to \fpaynet{} that the channel is open and \fpaynet{} finally stores the
  channel as open. This last exchange is needed to match the real-world
  interaction.

  \begin{figure}[H]
    \begin{systembox}{\fpaynet{} -- open}
      \begin{algorithmic}[1]
        \State Upon receiving $\left(\textsc{openChannel}, \alice, \bob, x,
        \mathit{tid}\right)$ from \alice:
        \Indent
          \State ensure \textit{tid} hasn't been used by \alice{} for opening
          another channel before
          \label{alg:fpaynet:open:valid}
          \State choose unique channel ID \textit{fchid}
          \State $\mathtt{pendingOpen}\left(\mathit{fchid}\right) \gets
          \left(\alice, \bob, x, \mathit{tid}\right)$
          \State send $\left(\textsc{openChannel}, \alice, \bob, x,
          \mathit{fchid}, \mathit{tid}\right)$ to \simulator
        \EndIndent
        \Statex

        \State Upon receiving (\textsc{channelAnnounced}, \alice, $p_{\alice,
        F}$, $p_{\bob, F}$, \textit{fchid}, \textit{pchid}, \textit{tid}) from
        \simulator:
        \Indent
          \State ensure that there is a \texttt{pendingOpen}(\textit{fchid})
          entry with temporary id \textit{tid}
          \label{alg:fpaynet:announced:valid}
          \State add $p_{\alice, F}, p_{\bob, F}$, \textit{pchid} and  mark
          ``\alice{} announced'' to \texttt{pendingOpen}(\textit{fchid})
          \label{alg:fpaynet:announced:add}
        \EndIndent
        \Statex

        \State Upon receiving (\textsc{checkForNew}, \alice, \bob, \textit{tid})
        from \alice:
        \Indent
          \State ensure there is a matching \texttt{channel} in
          \texttt{pendingOpen}(\textit{fchid}), marked with ``\alice{}
          announced''
          \label{alg:fpaynet:checkForNew:valid}
          \State $\left(\mathrm{funder}, \mathrm{fundee}, x, p_{\alice, F},
          p_{\bob, F}\right) \gets
          \mathtt{pendingOpen}\left(\mathit{fchid}\right)$
          \State send (\textsc{read}) to \ledger{} as \alice{} and store reply
          to $\Sigma_{\alice}$
          \State \texttt{checkClosed}($\Sigma_{\alice}$)
          \label{alg:fpaynet:checkForNew:read:alice}
          \State ensure that there is a TX $F \in \Sigma_{\alice}$ with a
          $\left(x, \left(p_{\mathrm{funder}, F} \wedge p_{\mathrm{fundee},
          F}\right)\right)$ output
          \label{alg:fpaynet:checkForNew:included}
          \State mark \texttt{channel} with ``waiting for
          \textsc{fundingLocked}''
          \label{alg:fpaynet:checkForNew:mark}
          \State send (\textsc{fundingLocked}, \alice, $\Sigma_{\alice}$,
          \textit{fchid}) to \simulator
          \label{alg:fpaynet:checkForNew:sim}
        \EndIndent
        \Statex

        \State Upon receiving (\textsc{fundingLocked}, \textit{fchid}) from
        \simulator:
        \Indent
          \State ensure a \texttt{channel} is in
          \texttt{pendingOpen}(\textit{fchid}), marked with ``waiting for
          \textsc{fundingLocked}'' and replace mark with ``waiting for
          \textsc{channelOpened}''
          \State send (\textsc{read}) to \ledger{} as \bob{} and store reply
          to $\Sigma_{\bob}$
          \State \texttt{checkClosed}($\Sigma_{\bob}$)
          \label{alg:fpaynet:checkForNew:read:bob}
          \State ensure that there is a TX $F \in \Sigma_{\bob}$ with a
          $\left(x, \left(p_{\mathrm{funder}, F} \wedge p_{\mathrm{fundee},
          F}\right)\right)$ output
          \State add \texttt{receipt}(\texttt{channel}) to
          \texttt{newChannels}(\bob)
          \label{alg:fpaynet:fundingLocked:report}
          \State send (\textsc{fundingLocked}, \bob, $\Sigma_{\bob}$,
          \textit{fchid}) to \simulator
          \label{alg:fpaynet:fundingLocked:sim}
        \EndIndent
        \Statex

        \State Upon receiving (\textsc{channelOpened}, \textit{fchid}) from
        \simulator:
        \Indent
          \State ensure a \texttt{channel} is in
          \texttt{pendingOpen}(\textit{fchid}), marked with ``waiting for
          \textsc{channelOpened}'' and remove mark
          \State $\mathtt{offChainBalance}\left(\mathrm{funder}\right) \gets
          \mathtt{offChainBalance}\left(\mathrm{funder}\right) + x$
          \label{alg:fpaynet:channelOpened:offchain}
          \State $\mathtt{onChainBalance}\left(\mathrm{funder}\right) \gets
          \mathtt{onChainBalance}\left(\mathrm{funder}\right) - x$
          \label{alg:fpaynet:channelOpened:onchain}
          \State $\mathtt{channel} \gets \left(\mathrm{funder}, \mathrm{fundee},
          x, 0, 0, \mathit{fchid}, \mathit{pchid}\right)$
          \State add \texttt{channel} to \texttt{channels}
          \State add \texttt{receipt}(\texttt{channel}) to
          \texttt{newChannels}(\alice)
          \label{alg:fpaynet:channelOpened:report}
          \State clear \texttt{pendingOpen}(\textit{fchid}) entry
        \EndIndent
      \end{algorithmic}
    \end{systembox}
    \caption{}
    \label{alg:fpaynet:open}
  \end{figure}

  When instructed to perform a payment, \fpaynet{} simply takes note of the
  message and forwards it to \simulator. It also remembers to inform the payer
  that the payment has been completed when \simulator{} says so. Observe here
  that \fpaynet{} trusts \simulator{} to correctly carry out channel updates.
  While counterintuitive, it allows \fpaynet{} to ignore the details of channel
  updates, signatures, key and transaction handling. Nevertheless, as we will
  see \fpaynet{} keeps track of requested and ostensibly carried out updates and
  ensures that upon channel closure the balances are as expected, therefore
  ensuring funds security.

  \begin{figure}[H]
    \begin{systembox}{\fpaynet{} -- pay}
      \begin{algorithmic}[1]
        \State Upon receiving $\left(\textsc{pay}, \bob, x,
        \overrightarrow{\mathtt{path}}, \mathit{expayid}\right)$ from \alice:
        \Indent
          \State choose unique payment ID \textit{payid}
          \State add $\left(\alice, \bob, x, \overrightarrow{\mathtt{path}},
          \mathit{payid}\right)$ to \texttt{pendingPay}
          \State send $\left(\textsc{pay}, \alice, \bob, x,
          \overrightarrow{\mathtt{path}}, \mathit{expayid}, \mathit{payid},
          \textsc{state}, \Sigma\right)$ to \simulator
          \label{alg:fpaynet:pay:send}
        \EndIndent
        \Statex

        \State Upon receiving (\textsc{update}, \texttt{receipt},
        \texttt{diffs}, \alice) from \simulator:
        \label{alg:fpaynet:update}
        \Indent
          \State append (\texttt{receipt}, \texttt{diffs}) to
          \texttt{updatesToReport}(\alice) \Comment{trust \simulator{} here,
          check on \textsc{resolvePays}}
          \label{alg:fpaynet:update:add}
          \State append \texttt{diffs} to \texttt{pendingDiffs}
          \State send (\textsc{continue}) to \simulator
        \EndIndent
      \end{algorithmic}
    \end{systembox}
    \caption{}
    \label{alg:fpaynet:pay}
  \end{figure}

  The message \textsc{resolvePays}, sent by \simulator{}, is supposed to contain
  a list of resolved payments, along with who was charged for each payment after
  all. For each entry there are four ``happy paths'' that do not lead to
  \fpaynet{} halting (\fpaynet{} halts when it cannot uphold its security
  guarantees anymore): if the payment failed and no balance is changed, if the
  charged player is the one who initiated the payment, if the charged player is
  corrupted or if she has not checked the blockchain at the right times, i.e.
  was negligent (as discussed in Section~\ref{sec:ov-paynet} and formally
  defined in Figures~\ref{alg:fpaynet:resolvepay}
  and~\ref{alg:fpaynet:resolvepay:haltcond:abs}). In case the payment was
  completed in a legal manner, the balance of all channels involved is updated
  accordingly (Fig.~\ref{alg:fpaynet:resolvepay:loops}). Conversely, \fpaynet{}
  halts if the charged player was not on the payment path
  (Fig.~\ref{alg:fpaynet:resolvepay},
  l.~\ref{alg:fpaynet:resolvepay:halt:path}), if a signature forgery has taken
  place (Fig.~\ref{alg:fpaynet:resolvepay},
  l.~\ref{alg:fpaynet:resolvepay:halt:ds}), if the charged player has not been
  negligent (Fig.~\ref{alg:fpaynet:resolvepay},
  ll.~\ref{alg:fpaynet:resolvepay:halt:rel}
  and~\ref{alg:fpaynet:resolvepay:halt:abs}), or if any one of the individual
  channel updates needed to carry out the whole payment has not been previously
  reported with an \textsc{update} message by \simulator{}
  (Fig.~\ref{alg:fpaynet:resolvepay:loops},
  l.~\ref{alg:fpaynet:resolvepay:loops:halt:do}).

  \begin{figure}[H]
    \begin{systembox}{\fpaynet{} -- resolve payments}
      \begin{algorithmic}[1]
        \State Upon receiving $\left(\textsc{resolvePays},
        \mathtt{charged}\right)$ from \simulator:
        \Comment{after first sending \textsc{pay}, \textsc{pushFulfill},
        \textsc{pushAdd}, \textsc{commit}}
        \Indent
          \ForAll{$\alice \text{ keys} \in \texttt{charged}$}
            \ForAll{$\left(\dave, \mathit{payid}\right) \in
            \mathtt{charged}\left(\alice\right)$}
              \State retrieve $\left(\alice, \bob, x,
              \overrightarrow{\mathtt{path}}, \mathit{payid}\right)$ and remove
              it from \texttt{pendingPay}
              \If{$\dave{} = \bot$} \Comment{Payment failed}
                \State continue with next iteration of inner loop
              \ElsIf{$\dave{} \notin \overrightarrow{\mathtt{path}}$}
                \State halt \Comment{Only players on path may be charged}
                \label{alg:fpaynet:resolvepay:halt:path}
              \ElsIf{$\dave{} \in \mathtt{corrupted}$}
                \State run code of Fig.~\ref{alg:fpaynet:resolvepay:loops}
                \State $\mathtt{offChainBalance}\left(\bob\right) \gets
                \mathtt{offChainBalance}\left(\bob\right) + x$
                \label{alg:fpaynet:resolvepay:corr:credit}
              \Else \ \Comment{\dave{} honest}
                \State send (\textsc{read}) to \ledger{} as \dave{} and store
                reply to $\Sigma_{\dave}$
                \State \texttt{checkClosed}($\Sigma_{\dave}$)
                \If{$\Sigma_{\dave}$ contains a tx that is not a
                $\mathtt{localCom}_n$ or a $\mathtt{remoteCom}_n$ and spends a
                funding tx for an open \texttt{channel} that contains \dave{}}
                \label{alg:fpaynet:resolvepay:haltcond:ds}
                  \State halt \Comment{DS forgery}
                  % TODO: decide how to determine forgeries on pubKey(Alice)
                  \label{alg:fpaynet:resolvepay:halt:ds}
                \ElsIf{$\Sigma_{\dave}$ contains in block $h_{\mathtt{tx}}$ an
                old $\mathtt{remoteCom}_m$ that does not contain the HTLC and a
                tx that spends the delayed output of $\mathtt{remoteCom}_m$}
                  \If{\texttt{polls}(\dave) contains an element in
                  $[h_{\mathtt{tx}}, h_{\mathtt{tx}} +
                  \mathtt{delay}\left(\dave\right) - 1]$}
                  \label{alg:fpaynet:resolvepay:haltcond:rel}
                    \State halt \Comment{\dave{} \textsc{poll}ed, but successful
                    malicious closure}
                    \label{alg:fpaynet:resolvepay:halt:rel}
                  \Else
                    \State $\mathtt{negligent}(\dave) \gets \true$
                  \EndIf
                \ElsIf{$\dave{} \neq \alice$}
                  \State calculate \texttt{IncomingCltvExpiry},
                  \texttt{OutgoingCltvExpiry} of \dave{} (as in
                  Fig.~\ref{alg:protocol:pay:invoice},
                  l.~\ref{alg:protocol:pay:invoice:cltv})
                  \If{$\Sigma_{\dave}$ does not contain an old
                  $\mathtt{remoteCom}_m$}
                    \If{failure condition of
                    Fig.~\ref{alg:fpaynet:resolvepay:haltcond:abs}
                    is true}
                      \State halt \Comment{\dave{} \textsc{poll}ed and fulfilled,
                      but charged}
                      \label{alg:fpaynet:resolvepay:halt:abs}
                    \Else
                      \State $\mathtt{negligent}(\dave) \gets \true$
                    \EndIf
                  \EndIf
                \EndIf
                \State run code of Fig.~\ref{alg:fpaynet:resolvepay:loops}
                \State $\mathtt{offChainBalance}\left(\dave\right) \gets
                \mathtt{offChainBalance}\left(\dave\right) - x$
                \label{alg:fpaynet:resolvepay:debit}
                \State $\mathtt{offChainBalance}\left(\bob\right) \gets
                \mathtt{offChainBalance}\left(\bob\right) + x$
                \label{alg:fpaynet:resolvepay:credit}
              \EndIf
            \EndFor
          \EndFor
        \EndIndent
      \end{algorithmic}
    \end{systembox}
    \caption{$r$, \texttt{windowSize} as in Proposition~\ref{prop:tochain}}
    \label{alg:fpaynet:resolvepay}
  \end{figure}

  \begin{figure}[H]
    \begin{titlebox}{\normalfont Absolute delay failure
    condition}{roundbox}{normal}
      \begin{algorithmic}[1,nonumber]
        \State $\mathtt{IncomingCltvExpiry} - \mathtt{OutgoingCltvExpiry} <
        \mathtt{relayDelay}(\alice) + \tochain{} \: \vee$
        \State (\texttt{polls}(\dave) contains two elements in
        $[\mathtt{OutgoingCltvExpiry}, \mathtt{IncomingCltvExpiry} -
        \tochain{}]$ that have a difference of at least $\tochain{} \: \wedge$
        \State \texttt{focs}(\dave) contains $\mathtt{IncomingCltvExpiry} -
        \tochain{} \ \wedge$
        \State the element in \texttt{polls}(\dave) was added before the element
        in \texttt{focs}(\dave))
      \end{algorithmic}
    \end{titlebox}
    \caption{}
    \label{alg:fpaynet:resolvepay:haltcond:abs}
  \end{figure}

  \begin{figure}[H]
    \begin{titlebox}{\normalfont Loop over payment hops for update and
    check}{roundbox}{normal}
      \begin{algorithmic}[1]
        \ForAll{open $\mathtt{channel}\text{s} \in
        \overrightarrow{\mathtt{path}}$ that are not in any
        \texttt{closedChannels}, starting from the one where \dave{} pays}
        \label{alg:fpaynet:resolvepay:loops:update:loop}
          \State in the first iteration, \texttt{payer} is \dave. In
          subsequent iterations, \texttt{payer} is the unique player that
          has received but has not given. The other \texttt{channel} party
          is \texttt{payee}
          \If{\texttt{payer} has $x$ or more in \texttt{channel}}
            \State update \texttt{channel} to the next version and
            transfer $x$ from \texttt{payer} to \texttt{payee}
            \label{alg:fpaynet:resolvepay:loops:update:do}
          \Else
            \State revert all updates done in this loop
          \EndIf
        \EndFor
        \ForAll{updated \texttt{channel}s in the previous loop}
        \label{alg:fpaynet:resolvepay:loops:halt:loop}
          \State ensure that a corresponding element has been added to
          the \texttt{updatesToReport} of each honest counterparty,
          otherwise halt
          \label{alg:fpaynet:resolvepay:loops:halt:do}
        \EndFor
        % TODO: modify proof to pass this test
        \If{$\dave{} = \alice{} \wedge{} \alice{} \notin \mathtt{corrupted}
        \wedge (\mathit{payid}, -x) \notin \mathtt{pendingDiffs}(\alice)$}
        \Comment{payer not informed}
        \label{alg:fpaynet:resolvepay:loops:indiffs:debit:cond}
          \State halt
          \label{alg:fpaynet:resolvepay:loops:indiffs:debit:halt}
        \EndIf
        \State remove (\textit{payid}, $-x$) from \texttt{pendingDiffs}(\alice)
        \If{$\bob{} \notin \mathtt{corrupted} \wedge (\mathit{payid}, x) \notin
        \mathtt{pendingDiffs}(\bob)$} \Comment{payee not informed}
        \label{alg:fpaynet:resolvepay:loops:indiffs:credit:cond}
          \State halt
          \label{alg:fpaynet:resolvepay:loops:indiffs:credit:halt}
        \EndIf
        \State remove (\textit{payid}, $x$) from \texttt{pendingDiffs}(\bob)
        % TODO end
      \end{algorithmic}
    \end{titlebox}
    \caption{}
    \label{alg:fpaynet:resolvepay:loops}
  \end{figure}

  Similarly to payment instructions, when \fpaynet{} receives a message
  instructing it to close a channel (Fig.~\ref{alg:fpaynet:close}), it takes a
  note of the pending closure, it stops serving any more requests for this
  channel and it forwards the request to \simulator. In turn \simulator{}
  notifies \fpaynet{} of a closed channel with the corresponding message, upon
  which \fpaynet{} takes a note to inform the corresponding player. Depending on
  whether the message instructed for a unilateral or a cooperative close,
  \fpaynet{} will either put or not a time limit respectively to the service of
  the request. In particular, in case of cooperative close, the time limit is
  infinity (l.~\ref{alg:fpaynet:close:coop:mark}). As we will see, in case a
  unilateral close request was made and the time limit for servicing it is
  reached, \fpaynet{} halts (Fig.~\ref{alg:fpaynet:close:func},
  l.~\ref{alg:fpaynet:close:func:idle}). Once more \fpaynet{} trusts \simulator,
  but later checks that the chain contains the correct transactions with
  \texttt{checkClosed}() (Fig.~\ref{alg:fpaynet:close:func}).

  \begin{figure}[H]
    \begin{systembox}{\fpaynet{} -- close}
      \begin{algorithmic}[1]
        \State Upon receiving (\textsc{closeChannel}, \texttt{receipt},
        \textit{pchid}) from \alice
        \Indent
          \State ensure that there is a $\mathtt{channel} \in \mathtt{channels}
          : \mathtt{receipt}\left(\mathtt{channel}\right) = \mathtt{receipt}$
          with ID \textit{pchid}
          \label{alg:fpaynet:close:coop:ensure}
          \State retrieve \textit{fchid} from \texttt{channel}
          \label{alg:fpaynet:close:coop:retrieve}
          \State add (\textit{fchid}, \texttt{receipt}(\texttt{channel}),
          $\infty$) to \texttt{pendingClose}(\alice)
          \label{alg:fpaynet:close:coop:mark}
          \State do not serve any other (\textsc{pay}, \textsc{closeChannel})
          message from \alice{} for this channel
          \label{alg:fpaynet:close:coop:noserve}
          \State send (\textsc{closeChannel}, \texttt{receipt}, \textit{pchid},
          \alice) to \simulator
          \label{alg:fpaynet:close:coop:send}
        \EndIndent
        \Statex

        \State Upon receiving (\textsc{forceCloseChannel}, \texttt{receipt},
        \textit{pchid}) from \alice
        \Indent
          \State retrieve \textit{fchid} from \texttt{channel}
          \label{alg:fpaynet:close:unilateral:retrieve}
          \State add (\textit{fchid}, \texttt{receipt}(\texttt{channel}),
          $\bot$) to \texttt{pendingClose}(\alice)
          \label{alg:fpaynet:close:unilateral:mark}
          \State do not serve any other (\textsc{pay}, \textsc{closeChannel},
          \textsc{forceCloseChannel}) message from \alice{} for this channel
          \label{alg:fpaynet:close:unilateral:noserve}
          \State send (\textsc{forceCloseChannel}, \texttt{receipt},
          \textit{pchid}, \alice) to \simulator
          \label{alg:fpaynet:close:unilateral:send}
        \EndIndent
        \Statex

        \State Upon receiving (\textsc{closedChannel}, \texttt{channel}, \alice)
        from \simulator:
        \Indent
          \State remove any (\textit{fchid} of channel,
          \texttt{receipt}(\texttt{channel}), $\infty$) from
          \texttt{pendingClose}(\alice)
          \State add (\textit{fchid} of channel,
          \texttt{receipt}(\texttt{channel}), $\bot$) to
          \texttt{closedChannels}(\alice)
          \Comment{trust \simulator{} here, check on \texttt{checkClosed}()}
          \label{alg:fpaynet:closedChannel:report}
          \State send (\textsc{continue}) to \simulator
        \EndIndent
        \Statex
      \end{algorithmic}
    \end{systembox}
    \caption{}
    \label{alg:fpaynet:close}
  \end{figure}

  After every \textsc{read} \fpaynet{} sends to \ledger{} and its response is
  received, \texttt{checkClosed}() (Fig.~\ref{alg:fpaynet:close:func}) is
  called. \fpaynet{} checks the input state $\Sigma$ for transactions that close
  channels and, in case no security violation has taken place, it updates the
  on- and off-chain balances of the player accordingly
  (ll.~\ref{alg:fpaynet:close:func:happy:start}-\ref{alg:fpaynet:close:func:happy:end}).
  The possible security violations are: signature forgery
  (l.~\ref{alg:fpaynet:close:func:dsforgery}), malicious closure even though the
  player was not negligent (l.~\ref{alg:fpaynet:close:func:malicious}), no
  closing transaction in $\Sigma$ even though the player asked for channel
  closure a substantial amount of time before
  (l.~\ref{alg:fpaynet:close:func:idle}) and incorrect on- or off-chain balance
  after the closing of all of the player's channels
  (l.~\ref{alg:fpaynet:close:func:balance}).

  \begin{figure}[H]
    \begin{systembox}{\fpaynet{} -- \texttt{checkClosed}()}
      \begin{algorithmic}[1]
        \Function{checkClosed}{$\Sigma_{\alice}$} \Comment{Called after every
        (\textsc{read}), ensures requested closes eventually happen}
        \label{alg:fpaynet:close:func:start}
          \If{there is any closing/commitment transaction in $\Sigma_{\alice}$
          with no corresponding entry in $\mathtt{pendingClose}(\alice) \cup
          \mathtt{closedChannels}(\alice)$}
          \label{alg:fpaynet:close:func:ifunnotified}
            \State add $(\mathit{fchid}, \mathtt{receipt}, \bot)$ to
            \texttt{closedChannels}(\alice), where \textit{fchid} is the ID of
            the corresponding \texttt{channel}, \texttt{receipt} comes from the
            latest \texttt{channel} state
            \label{alg:fpaynet:close:func:unnotified}
          \EndIf
          \ForAll{entries $(\mathit{fchid}, \mathtt{receipt}, h) \in
          \mathtt{pendingClose}(\alice) \cup \mathtt{closedChannels}(\alice)$}
            \If{there is a closing/commitment transaction in $\Sigma_{\alice}$
            for open \texttt{channel} with ID \textit{fchid} with a balance that
            corresponds to \texttt{receipt}}
            \label{alg:fpaynet:close:func:happy:start}
              \State let $x, y$ \alice{}'s and \texttt{channel} counterparty
              \bob{}'s balances respectively
              \State reduce \texttt{offChainBalance}(\alice) by $x$
              \State increase \texttt{onChainBalance}(\alice) by $x$
              \label{alg:fpaynet:close:func:alice:credit}
              \State reduce \texttt{offChainBalance}(\bob) by $y$
              \State increase \texttt{onChainBalance}(\bob) by $y$
              \label{alg:fpaynet:close:func:bob:credit}
              \State remove \texttt{channel} from \texttt{channels} \& entry
              from \texttt{pendingClose}(\alice)
              \If{there is an (\textit{fchid}, \_, \_) entry in
              \texttt{pendingClose}(\bob)}
                \State remove it from \texttt{pendingClose}(\bob)
              \EndIf
              \label{alg:fpaynet:close:func:happy:end}
            \ElsIf{there is a tx in $\Sigma_{\alice}$ that is not a
            closing/commitment tx and spends the funding tx of the
            \texttt{channel} with ID \textit{fchid}}
            \label{alg:fpaynet:close:func:ifdsforgery}
              \State halt \Comment{DS forgery}
              \label{alg:fpaynet:close:func:dsforgery}
            \ElsIf{there is a commitment transaction in block of height $h$ in
            $\Sigma_{\alice}$ for open \texttt{channel} with ID \textit{fchid}
            with a balance that does not correspond to the \texttt{receipt} and
            the delayed output has been spent by the counterparty}
              \If{\texttt{polls}(\alice) contains an entry in $[h, h
              + \mathtt{delay}(\alice) - 1]$}
              \label{alg:fpaynet:close:func:ifmalicious}
                \State halt
                \label{alg:fpaynet:close:func:malicious}
              \Else
                \State $\mathtt{negligent}(\alice) \gets \true$
              \EndIf
            \ElsIf{there is no such closing/commitment transaction $\wedge \ h =
            \bot$}
              \State assign largest $\Sigma_{\alice}$ block no. to entry's $h$
            \ElsIf{there is no such closing/commitment transaction $\wedge \ h
            \neq \bot \ \wedge$ (largest block number of $\Sigma_{\alice}) \geq
            h + \tochain{}$}
            \label{alg:fpaynet:close:func:ifidle}
              \State halt
              \label{alg:fpaynet:close:func:idle}
            \EndIf
          \EndFor
          \If{\alice{} has no open channels in $\Sigma_{\alice} \wedge
          \mathtt{negligent}(\alice) = \false \wedge
          (\mathtt{offChainBalance}(\alice) \neq 0 \: \vee$
          \texttt{onChainBalance}(\alice) is not equal to the total funds
          exclusively spendable by \alice{} in $\Sigma_{\alice}$)}
            \State halt
            \label{alg:fpaynet:close:func:balance}
          \EndIf
        \EndFunction
        \label{alg:fpaynet:close:func:end}
      \end{algorithmic}
    \end{systembox}
    \caption{}
    \label{alg:fpaynet:close:func}
  \end{figure}

  \textsc{poll} is a request that every player has to make to \fpaynet{}
  periodically (once every \texttt{delay} blocks, as set on registration) in
  order to remain non-negligent. In a software implementation, such a request
  would be automatically sent at safe time intervals. When receiving
  \textsc{poll} (Fig.~\ref{alg:fpaynet:poll}), \fpaynet{} checks the ledger for
  maliciously closed channels and halts in case of a forgery
  (l.~\ref{alg:fpaynet:poll:dsforgery}) or in case of a successful malicious
  closing of a channel whilst the offended player was non-negligent
  (l.~\ref{alg:fpaynet:poll:haspolled:halt}). If on the other hand a channel has
  been closed maliciously but the offended player did not \textsc{poll} in time,
  she is marked as negligent (l.~\ref{alg:fpaynet:poll:negligent}).

  \begin{figure}[H]
    \begin{systembox}{\fpaynet{} -- poll}
      \begin{algorithmic}[1]
        \State Upon receiving (\textsc{poll}) from \alice:
        \Indent
          \State send (\textsc{read}) to \ledger{} as \alice{} and store reply
          to $\Sigma_{\alice}$
          \State add largest block number in $\Sigma_{\alice}$ to
          \texttt{polls}(\alice)
          \label{alg:fpaynet:poll:height}
          \State \texttt{checkClosed}($\Sigma_{\alice}$)
          \label{alg:fpaynet:poll:read}
          \If{$\exists \mathtt{channel} \in \Sigma_{\alice}$ that contains
          \alice{} and is closed by a tx that is not a commitment transaction}
          \label{alg:fpaynet:poll:ifdsforgery}
            \State halt \Comment{DS forgery}
            \label{alg:fpaynet:poll:dsforgery}
          \EndIf
          \ForAll{$\mathtt{channel}\text{s} \in \Sigma_{\alice}$ that contain
          \alice{} and are maliciously closed by a remote commitment \texttt{tx}
          (one with an older channel version than the irrevocably committed one)
          in block with height $h_{\mathtt{tx}}$}
          \label{alg:fpaynet:poll:malicious}
            \If{the delayed output (of the counterparty) has been spent}
              \If{\texttt{polls}(\alice) has an element in
              $\left[h_{\mathtt{tx}}, h_{\mathtt{tx}} +
              \mathtt{delay}\left(\alice\right) - 1\right]$}
              \label{alg:fpaynet:poll:haspolled}
                \State halt \Comment{\alice{} wasn't negligent but couldn't
                punish}
                \label{alg:fpaynet:poll:haspolled:halt}
              \Else
                \State $\mathtt{negligent}(\alice) \gets \true$
                \label{alg:fpaynet:poll:negligent}
              \EndIf
            \EndIf
          \EndFor
          \State send (\textsc{poll}, $\Sigma_{\alice}$, \alice) to \simulator
          \label{alg:fpaynet:poll:send}
        \EndIndent
      \end{algorithmic}
    \end{systembox}
    \caption{}
    \label{alg:fpaynet:poll}
  \end{figure}

  The last part of \fpaynet{} (Fig.~\ref{alg:fpaynet:daemon}) contains some
  additional ``daemon'' messages that help various processes carry on.
  \textsc{pushFulfill}, \textsc{pushAdd} and \textsc{commit} are simply
  forwarded to \simulator. They exist because the ``token of execution'' in the
  protocol does not follow the strict order required by UC, and thus some
  additional messages are needed for the protocol to carry on. In other words,
  they are needed due to the incompatibility of the serial execution of UC and
  the asynchronous nature of LN.

  \textsc{FulfillOnChain} has to be sent by a multi-hop payment intermediary
  that has not been paid by the previous player off-chain in order to close the
  channel. The request is noted and forwarded to \simulator. \textsc{getNews}
  requests from \fpaynet{} information on newly opened, closed and updated
  channels.

  \begin{figure}[H]
    \begin{systembox}{\fpaynet{} -- daemon messages}
      \begin{algorithmic}[1]
        \State Upon receiving (\textsc{pushFulfill}, \textit{pchid}) from
        \alice:
        \Indent
          \State send (\textsc{pushFulfill}, \textit{pchid}, \alice,
          \textsc{state}, $\Sigma$) to \simulator
          \label{alg:fpaynet:daemon:fulfill}
        \EndIndent
        \Statex

        \State Upon receiving (\textsc{pushAdd}, \textit{pchid}) from \alice:
        \Indent
          \State send (\textsc{pushAdd}, \textit{pchid}, \alice, \textsc{state},
          $\Sigma$) to \simulator
          \label{alg:fpaynet:daemon:add}
        \EndIndent
        \Statex

        \State Upon receiving (\textsc{commit}, \textit{pchid}) from \alice:
        \Indent
          \State send (\textsc{commit}, \textit{pchid}, \alice, \textsc{state},
          $\Sigma$) to \simulator
          \label{alg:fpaynet:daemon:commit}
        \EndIndent
        \Statex

        \State Upon receiving (\textsc{fulfillOnChain}) from \alice:
        \Indent
          \State send (\textsc{read}) to \ledger{} as \alice{}, store reply to
          $\Sigma_{\alice}$ and assign largest block number to $t$
          \label{alg:fpaynet:daemon:foc:read}
          \State add $t$ to \texttt{focs}(\alice)
          \State \texttt{checkClosed}($\Sigma_{\alice}$)
          \State send (\textsc{fulfillOnChain}, $t$, \alice) to \simulator
        \EndIndent
        \Statex

        \State Upon receiving (\textsc{getNews}) from \alice:
        \label{alg:fpaynet:getnews}
        \Indent
          \State clear \texttt{newChannels}(\alice),
          \texttt{closedChannels}(\alice), \texttt{updatesToReport}(\alice) and
          send them to \alice{} with message name \textsc{news}, stripping
          \textit{fchid} and h from \texttt{closedChannels}(\alice)
          \label{alg:fpaynet:getnews:send}
        \EndIndent
      \end{algorithmic}
    \end{systembox}
    \caption{}
    \label{alg:fpaynet:daemon}
  \end{figure}
