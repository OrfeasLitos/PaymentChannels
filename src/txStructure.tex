\section{Transaction Structure}
  \label{appendix:txstructure}
  A well-formed transaction contains:
  \begin{itemize}
    \item A list of inputs
    \item A list of outputs
    \item An arbitrary payload (optional)
  \end{itemize}
  Each input must be connected to a single valid, previously unconnected
  (unspent) output in the state.

  We assume a one-way, collision-free hash function $\mathcal{H}$ that creates
  the id of each transaction.

  A well-formed output contains:
  \begin{itemize}
    \item A value in coins
    \item A list of spending methods. An input that spends this output must
    specify exactly one of the available spending methods.
  \end{itemize}

  A well-formed spending method contains any combination of the following:
  \begin{itemize}
    \item Public keys in disjunctive normal form. An input that spends using
    this spending method must contain signatures made with the private keys that
    correspond to the public keys of one of the conjunctions. If empty, no
    signatures are needed.
    \item Absolute locktime in block height, transaction height or time. The
    output can be spent by an input to a transaction that is added to the state
    after the specified block height, transaction height or time.
    \item Relative locktime in block height, transaction height or time. The
    output can be spent by an input that is added to the state after the current
    output has been part of the state for the specified number of blocks,
    transactions or time.
    \item Hashlock value. The output can be spent by an input that contains a
    preimage that hashes to the hashlock value. If empty, the input does not
    need to specify a preimage.
  \end{itemize}
  If both the absolute and the relative locktime are empty, output can be spent
  immediately after being added to the state.

  A well-formed input contains:
  \begin{itemize}
    \item A reference to the output and the spending method it spends
    \item A set of signatures that correspond to one of the conjunctions of
    public keys in the referred spending method (if needed)
    \item A preimage that hashes to the hashlock value of the referred spending
    method (if needed)
  \end{itemize}

  Lastly, the sum of coins of the outputs referenced by the inputs of the
  transaction (to-be-spent outputs) should be greater than or equal to the sum
  of coins of the outputs of the transaction.

  We say that an unspent output is currently exclusively spendable by a player
  $Alice$ with a public key $pk$ and a hash list $hl$ if for each spending
  method one of the following two holds:
  \begin{itemize}
    \item It still has a locktime that has not expired and thus is currently
    unspendable, or
    \item The only specified public key is $pk$ and if there is a hashlock, its
    hash is contained in $hl$.
  \end{itemize}
  If an output is exclusively spendable, we say that its coins are exclusively
  spendable.
