\section{Lightning Network}
  This construction is the first to achieve a functional model for payment channels. It is
  designed for bitcoin and requires some new opcodes and removing the malleability of
  transactions to function properly~\cite{lightning}.

  \subsection{Simple two-party channel}
    The basic construction is as follows. Suppose that $Alice$ and $Bob$ want to create a
    payment channel that contains 1 BTC consisting of 0.5 BTC from each party. To achieve
    this, they follow these steps (see also section 3.1.2 and Figure 4 in 3.3.2
    in~\cite{lightning}):
    \begin{enumerate}
      \item Either party (say $Alice$) creates a "Funding" transaction ($F$) with an input
      of 0.5 BTC from her and 0.5 BTC from $Bob$, and a 2-of-$\left\{Alice, Bob\right\}$
      multisig as output; she then sends $F$ to $Bob$. This transaction is not yet signed
      nor broadcast. $F$ needs to be signed by both parties to be valid.

      \item $Alice$ creates, signs and sends to $Bob$ a "Commitment" transaction ($C1b$)
      that spends $F$ and has the following outputs:
      \begin{enumerate}
        \item 0.5 BTC that can be spent by $Alice$ immediately when $C1b$ is broadcast.
        \item 0.5 BTC that can be spent by either party, but $Bob$ can spend it only after
        a specified amount of blocks (say $n$) have been mined on top of $C1b$, whereas
        $Alice$ can spend it only if $Bob$ provides her with a "Breach Remedy" transaction
        (explained later) signed by him. This output is called "Revocable Sequence
        Maturity Contract" (RSMC).
      \end{enumerate}
      Furthermore, $Alice$ creates, signs and sends a "Revocable Delivery" transaction
      ($RD1b$) that pays the first of the two outputs of $C1b$ to $Bob$, but will be
      accepted by the network if it is in the mempool only after $n$ blocks have been
      mined on top of $C1b$.

      $Bob$ similarly creates, signs and sends $C1a$ and $RD1a$ to $Alice$.

      \item After $Alice$ receives the signed $C1a$ and $RD1a$ from $Bob$, she verifies
      that they are both valid and correctly spend $F$. Given that everything works out
      right, she signs $F$ and sends it to $Bob$.

      Observe that she is not running the risk of $Bob$ refusing to cooperate in signing
      $F$ and thus keeping her 0.5 BTC locked because she has the ability to sign and
      broadcast the (already signed by $Bob$) $C1a$ and $RD1a$ and thus get her money back
      $n$ confirmations after $C1a$ is confirmed (that is when $RD1a$ is confirmed). Thus
      $Alice$ need not trust $Bob$ in any way.

      $Bob$ similarly verifies that $C1b$ and $RD1b$ have the correct structure, along
      with $Alice$'s signature on $F$. He then signs $F$ and broadcasts it. Note that he
      does not have to trust $Alice$ either.
    \end{enumerate}

    After initially setting up the channel, $Alice$ and $Bob$ can update it as follows
    (see also section 3.3.4 and Figures 7, 8 in~\cite{lightning}):
    \begin{enumerate}
      \item Both $Alice$ and $Bob$ follow exactly the same steps as before to create
      $C2a$, $C2b$, $RD2a$ and $RD2b$; the only difference these transactions have to
      their counterparts from the previous state of the channel is that, instead of 0.5
      BTC for each player, they contain the new agreed balance of the channel (e.g. 0.4
      BTC for $Alice$ and 0.6 BTC for $Bob$).

      \item $Alice$ creates, signs and sends to $Bob$ a so-called "Breach Remedy"
      transaction ($BR1a$). This transaction lets $Bob$ redeem the RSMC output of $C1a$ as
      soon as $C1a$ is broadcast. $Bob$ similarly creates, signs and sends $BR1b$ to
      $Alice$.
    \end{enumerate}
    Note that this effectively disincentivises $Alice$ from ever broadcasting $C1a$, since
    in such case $Bob$ will have a window of $n$ blocks during which he can claim the
    entire sum in $C1a$, 1 BTC, for himself. $Alice$ had better purge $C1a$ after $BR1a$
    is sent to $Bob$. Similarly $Bob$ is incentivised to refrain from ever broadcasting
    $C1b$.

    This arrangement creates a situation where both players can be confident that the
    state of the channel is the one expressed by $C2a$, $C2b$, $RD2a$ and $RD2b$, thus
    they can assume that $Alice$ has just paid $Bob$ 0.1 BTC. No trust between the two
    players was needed all along. There are only two caveats: First, both players must
    periodically check the blockchain to ensure that the other party has not broadcast an
    old Commitment transaction. Second, in case of an uncooperative counterparty, one has
    to wait a prespecified amount of time before releasing their funds, which may be
    undesirable.

    Thus, the necessary number of blocks mined on top of a Confirmation transaction for a
    subsequent Revocable Delivery to be valid (previously called $n$) must be carefully
    chosen in a way that does not lock up the funds for a long time in case of a dispute
    and at the same time does not require that the parties check the blockchain too often
    for a malicious broadcast of an already invalidated Commitment transaction.

    $Alice$ can outsource the task of the periodic check to a dedicated service by sending
    it all the previous Breach Remedy transactions. To incentivise the service to
    cooperate, $Alice$ can pay a fee to it as an output of these transactions. Note that
    $Alice$ does not need to trust the service, since the only thing it can do is to
    broadcast a Branch Remedy transaction that was created by $Alice$; she never discloses
    any of her private keys to it.

    Finally, the parties can cooperatively close the channel without having to wait $n$
    blocks as follows: When both parties have agreed to closing the channel, $Alice$
    creates, signs and sends to $Bob$ an "Exercise Settlement" transaction ($ES$) that
    spends the Funding transaction and has two simple outputs, each paying to the
    respective party the sum of the last agreed Commitment transaction. Following the
    previous example, this transaction would pay 0.4 BTC to $Alice$ and 0.6 BTC to $Bob$.
    $Bob$ can then also sign and broadcast the transaction to close the channel.

    Once $Alice$ has sent $ES$, she considers the channel as closed. If $Bob$ does not
    broadcast $ES$, we have a dispute and she has to broadcast the latest Commitment
    transaction and wait for her funds to be unlocked.

  \subsection{Payments depending on preimage knowledge (HTLC)}
    Multi-hop payments can take place between players (e.g. $Alice$ and $Dave$) who do not
    share a simple channel (i.e. an on-chain Funding transaction), but share simple
    channels with intermediate nodes (e.g. $Alice$ with $Bob$, $Bob$ with $Carol$ and
    $Carol$ with $Dave$).

    To enable the creation of multi-hop channels, so-called "Hashed Timelock Contracts"
    (HTLC) are used. An HTLC is an additional output in a Commitment transaction which can
    be redeemed by either $Alice$ or $Bob$; $Alice$ can redeem it after a specified number
    of additional blocks, say $m$, have been mined after the creation (\textit{not} the
    broadcast) of the Commitment transaction, whereas $Bob$ can redeem it at any time, but
    only if he produces the preimage $R$ of a hash specified in the HTLC output (see also
    section 4.2 and Figure 12 in~\cite{lightning}).

    More specifically, consider $C2a, C2b$ where, contrary to the example in the previous
    subsection, $Alice$ has paid the 0.1 BTC to an HTLC instead of directly to $Bob$.
    $Bob$ should be able to redeem the 0.1 BTC only if he knows the preimage $R$ before
    the $m$ blocks have been mined. In addition to $RD2a$ and $RD2b$, six additional
    transactions have to be signed and exchanged.
    \begin{enumerate}
      \item $Alice$ signs and sends an "HTLC Execution Delivery" transaction ($HED1a$) to
      $Bob$. $HED1a$ pays the HTLC output of $C2a$ to $Bob$, only if he knows the required
      preimage $R$. Only $Bob$ can broadcast the transaction.
      \item $Bob$ signs and sends a so-called "HTLC Timeout Delivery" transaction
      ($HTD1b$) to $Alice$. $HTD1b$ pays the HTLC output of $C2b$ to $Alice$, only after
      $m$ blocks have been mined from the time $C2b$ was created. Only $Alice$ can
      broadcast this transaction.
      \item $Alice$ signs and sends an "HTLC Execution" transaction ($HE1b$) to $Bob$.
      $HE1b$ pays the HTLC output of $C2b$ to $Bob$, only if he knows the required
      preimage $R$. Only $Bob$ can broadcast this transaction. Its single output is an
      RSMC with duration $n$, spendable by $Bob$.
      \item $Alice$ signs and sends an "HTLC Execution Revocable Delivery" transaction
      ($HERD1b$) to $Bob$. This transaction spends the RSMC output of $HE1b$. $Bob$ can
      broadcast this transaction after $n$ blocks have been mined on top of $HE1b$.
      \item $Bob$ signs and sends an "HTLC Timeout" transaction ($HT1a$) to $Alice$.
      $HT1a$ pays the HTLC output of $C2a$ to $Alice$, only after $m$ blocks have been
      mined from the time $HT1a$ was created. Its single output is an RSMC with duration
      $n$, spendable by $Alice$.
      \item $Bob$ signs and sends an "HTLC Timeout Revocable Delivery" transaction
      ($HTRD1b$) to $Alice$. This transaction spends the RSMC output of $HT1b$. $Alice$
      can broadcast this transaction after $n$ blocks have been mined on top of $HE1b$.
    \end{enumerate}
    Note that once again, no trust is necessary in the process described above. The
    RSMC outputs of $HT1a$ and $HE1b$ are necessary for future invalidation according to
    the "Breach Remedy" method. More details can be found in Figure 14 of section 4.3. In
    case of common desire to close the channel, they can be cooperatively closed using the
    "Exercise Settlement" method.
