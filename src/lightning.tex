\section{Lightning Network}
  This construction is the first to achieve a functional model for payment channels. It
  is designed for bitcoin and requires some new opcodes and removing the malleability of
  transactions to function properly.

  The basic construction is as follows. Suppose that $Alice$ and $Bob$ want to create a
  payment channel that contains 1 BTC consisting of 0.5 BTC from each party. To achieve
  this, they follow these steps (see also section 3.1.2 and Figure 4 in 3.3.2 in
  \cite{lightning}):
  \begin{enumerate}
    \item Either party (say $Alice$) creates a transaction called $F$ with an input of 0.5
    BTC from her and 0.5 BTC from $Bob$, and a 2-of-$\left\{Alice, Bob\right\}$ multisig
    as output; she then sends $F$ to $Bob$. This transaction is not yet signed nor
    broadcast. $F$ needs to be signed by both parties to be valid.

    \item $Alice$ creates, signs and sends to $Bob$ a transaction ($C1b$) that spends $F$
    and has the following outputs:
    \begin{enumerate}
      \item 0.5 BTC that can be spent by either party, but only after a specified amount
      of blocks (say $n$) have been mined on top of $C1b$.
      \item 0.5 BTC that can be spent by $Alice$ immediately when $C1b$ is broadcast.
    \end{enumerate}
    Furthermore, $Alice$ creates, signs and sends a "Revocable Delivery" transaction
    ($RD1b$) that pays the first of the two outputs to $Bob$, but will be accepted by the
    network if it is in the mempool only after $n$ blocks have been mined on top of $C1b$.

    $Bob$ similarly creates, signs and sends $C1a$ and $RD1a$ to $Alice$.

    \item After $Alice$ receives the signed $C1a$ and $RD1a$ from $Bob$, she verifies that
    they are both valid and correctly spend $F$. Given that everything works out right,
    she signs $F$ and sends it to $Bob$.

    Observe that she is not running the risk of $Bob$ refusing to cooperate in signing $F$
    and thus keeping her 0.5 BTC locked because she has the ability to sign and broadcast
    the (already signed by $Bob$) $C1a$ and $RD1a$ and thus get her money back $n$
    confirmations after $C1a$ is confirmed (that is when $RD1a$ is confirmed). Thus
    $Alice$ need not trust $Bob$ in any way.

    $Bob$ similarly verifies that $C1b$ and $RD1b$ have the correct structure, along with
    $Alice$'s signature on $F$. He then signs $F$ and broadcasts it. Note that he does not
    have to trust $Alice$ either.
  \end{enumerate}
