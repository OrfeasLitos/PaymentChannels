\section{Lightning Network}
  This construction is the first to achieve a functional model for payment channels. It
  is designed for bitcoin and requires some new opcodes and removing the malleability of
  transactions to function properly.

  The basic construction is as follows. Suppose that $Alice$ and $Bob$ want to create a
  payment channel that contains 1 BTC consisting of 0.5 BTC from each party. To achieve
  this, they follow these steps (see also section 3.1.2 and Figure 4 in 3.3.2 in
  \cite{lightning}):
  \begin{enumerate}
    \item Either party (say $Alice$) creates a "Funding" transaction ($F$) with an input
    of 0.5 BTC from her and 0.5 BTC from $Bob$, and a 2-of-$\left\{Alice, Bob\right\}$
    multisig as output; she then sends $F$ to $Bob$. This transaction is not yet signed
    nor broadcast. $F$ needs to be signed by both parties to be valid.

    \item $Alice$ creates, signs and sends to $Bob$ a "Commitment" transaction ($C1b$)
    that spends $F$ and has the following outputs:
    \begin{enumerate}
      \item 0.5 BTC that can be spent by $Alice$ immediately when $C1b$ is broadcast.
      \item 0.5 BTC that can be spent by either party, but $Bob$ can spend it only after
      a specified amount of blocks (say $n$) have been mined on top of $C1b$, whereas
      $Alice$ can spend it only if $Bob$ provides her with a "Breach Remedy" transaction
      (explained later) signed by him. This output is called "Revocable Sequence Maturity
      Contract" (RSMC).
    \end{enumerate}
    Furthermore, $Alice$ creates, signs and sends a "Revocable Delivery" transaction
    ($RD1b$) that pays the first of the two outputs to $Bob$, but will be accepted by the
    network if it is in the mempool only after $n$ blocks have been mined on top of $C1b$.

    $Bob$ similarly creates, signs and sends $C1a$ and $RD1a$ to $Alice$.

    \item After $Alice$ receives the signed $C1a$ and $RD1a$ from $Bob$, she verifies that
    they are both valid and correctly spend $F$. Given that everything works out right,
    she signs $F$ and sends it to $Bob$.

    Observe that she is not running the risk of $Bob$ refusing to cooperate in signing $F$
    and thus keeping her 0.5 BTC locked because she has the ability to sign and broadcast
    the (already signed by $Bob$) $C1a$ and $RD1a$ and thus get her money back $n$
    confirmations after $C1a$ is confirmed (that is when $RD1a$ is confirmed). Thus
    $Alice$ need not trust $Bob$ in any way.

    $Bob$ similarly verifies that $C1b$ and $RD1b$ have the correct structure, along with
    $Alice$'s signature on $F$. He then signs $F$ and broadcasts it. Note that he does not
    have to trust $Alice$ either.
  \end{enumerate}

  After initially setting up the channel, $Alice$ and $Bob$ can update it as follows:
  \begin{enumerate}
    \item Both $Alice$ and $Bob$ follow exactly the same steps as before to create $C2a$,
    $C2b$, $RD2a$ and $RD2b$; the only difference these transactions have to their
    counterparts from the previous state of the channel is that, instead of 0.5 BTC for
    each player, they contain the new agreed balance of the channel (e.g. 0.4 BTC for
    $Alice$ and 0.6 BTC for $Bob$).

    \item $Alice$ creates, signs and sends to $Bob$ a so-called "Breach Remedy"
    transaction ($BR1a$). This transaction lets $Bob$ redeem the RSMC output of $C1a$ as
    soon as $C1a$ is broadcast. $Bob$ similarly creates, signs and sends $BR1b$ to
    $Alice$.
  \end{enumerate}
  Note that this effectively disincentivises $Alice$ from ever broadcasting $C1a$,
  since then $Bob$ will have a window of $n$ blocks during which he can claim the entire
  sum in $C1a$, 1 BTC, for himself. $Alice$ had better purge $C1a$ after $BR1a$ is sent to
  $Bob$. Similarly $Bob$ is incentivised to refrain from ever broadcasting $C1b$.

  This arrangement creates a situation where both players can be confident that the state
  of the channel is the one expressed by $C2a$, $C2b$, $RD2a$ and $RD2b$, thus they can
  assume that $Alice$ has just paid $Bob$ 0.1 BTC. No trust between the two players was
  needed all along. The only caveat is that both players must periodically check the
  blockchain to ensure that the other party has not broadcast an old Commitment
  transaction.

  $Alice$ can outsource the task of the periodic check to a dedicated service by sending
  it all the previous Breach Remedy transactions. To incentivise the service to cooperate,
  $Alice$ can pay a fee to it as an output of these transactions. Note that $Alice$ does
  not need to trust the service, since the only thing it can do is to broadcast a Branch
  Remedy transaction that was created by $Alice$; she never discloses any of her private
  keys to it.
